\ifdefined\tool\else
\documentclass[uplatex, 10pt, a4j, dvipdfmx]{jsarticle}

\usepackage{packages/abstract}
\usepackage{enumitem}
\usepackage[dvipdfmx]{graphicx}
\usepackage{url}
\usepackage[dvipdfmx, hidelinks]{hyperref}
\usepackage{pxjahyper}

\newcommand{\tool}{2VSG}
\newcommand{\VDM}{VDM\texttt{++}}
\newcommand{\AbstractStandalone}{}

% make abstract コマンドでの発表要旨のPDF生成に必要な記述項目
\setAbstractTitle{画面遷移システムを対象とした\\\VDM 仕様作成ツール\tool の\\GUI操作への拡張}
\setAbstractPresenter{宮原 嵩尭}

\begin{document}
\AbstractHeader
\begin{AbstractBody}
\fi

近年、ソフトウェア開発において、システムの大規模化および複雑化が進んでおり、ソフトウェアのバグが社会にもたらす影響は甚大なものになっている。
特に、開発の上流工程で、仕様記述に曖昧さを含んでいる自然言語を用いることが、ソフトウェアにバグが混入する原因の1つとなっている。

そこで、自然言語の曖昧性によるソフトウェアへのバグの混入を回避し、厳密な仕様書を作成する方法の1つに、形式手法(Formal Method)を用いた仕様記述が挙げられる。

形式手法の1つにVDM(Vienna Development Method)があり、
VDMをオブジェクト指向に拡張した形式仕様記述言語として\VDM がある。
\VDM は、システムの状態遷移や制約を論理的に表現できるという特徴を持つため、画面を状態、画面状態の変化を遷移として捉える画面遷移システムにおいて、
遷移条件や状態遷移に伴う制約を明確に記述する手法として用いられる。

しかし、\VDM による仕様記述を行うためには、\VDM の構文や記述規則に関する事前知識が必要であるという課題があり、\VDM の導入コストを高めている。

この課題を解決するため、画面遷移システムを対象とした、\VDM 仕様の作成支援をするツールとして\tool(2vdm-spec-generator) がある。\tool は、Markdown形式の仕様記述から\VDM 仕様を生成できる。
\tool を用いることで、自然言語に近い記述から\VDM 仕様を生成できるため、\VDM 仕様の導入コストを低減できる。

しかし、\tool は、特定の記述ルールに則ってMarkdown形式の仕様を記述する必要がある。そのため、\VDM 仕様の作成に時間がかかるという課題がある。

そこで本研究では、画面遷移システムの\VDM 仕様の作成にかかる時間の削減を目的として、\tool に対してGUI操作による\VDM 仕様の作成を可能にする拡張を行う。
本研究では、\tool をGUI操作による\VDM 仕様の作成を可能とするため、GUI操作で実際に操作するCondition Transition Map(CTM)を提案する。
CTMは、画面遷移システムにおけるボタンとイベントの関係を視覚的に表し、画面要素、ボタン要素、イベント要素、分岐イベント要素、タイムアウト要素、遷移先のないイベント要素の6つの構成要素からなる。

拡張後の\tool は、以下の3つの機能を持つ。

\begin{itemize}
    \item ページ遷移機能 \\
    ページ遷移機能は、\tool の起動時に表示する「スタートページ」、「Markdown仕様記述ページ」、および、本研究の拡張で新たに追加する「GUI操作による\VDM 仕様編集ページ」の3つのページ間の遷移を可能にする機能である。
    
    \item 描画機能 \\
    描画機能は、「GUI操作による\VDM 仕様編集ページ」において、「フォルダツリー表示領域」、「CTM領域」、「\VDM 仕様表示領域」、および、「操作ボタン領域」の4つの領域に適した描画を可能にする機能である。

    \item GUI操作による仕様編集機能 \\
    GUI操作による仕様編集機能は、ユーザによるメニューバーの操作、操作ボタンの操作、CTM領域のクリックイベントに応じて、CTMの要素の追加、削除、編集を行う。さらに、CTMの要素の追加、削除、編集を行うことで、編集対象としているMarkdown仕様、および、\VDM 仕様を作成、編集する機能である。
\end{itemize}

適用例を用いて、拡張後の\tool が、正しく動作し、GUI操作で画面遷移システムの\VDM 仕様を生成できることを確認した。

次に、評価実験によって、拡張後の\tool が\VDM 仕様作成時間を削減できることを確認した。具体的には、既存の\tool と拡張後の\tool を用いて、画面遷移システムの2つの画面の仕様を対象とした\VDM 仕様を作成し、その作成時間を比較する実験を行った。
その結果、拡張後の\tool は、既存の\tool に比べて、\VDM 仕様の作成時間を削減できることを確認した。

以上より、拡張後の\tool は、画面遷移システムの\VDM 仕様の作成にかかる時間の削減が達成できたといえる。

\ifdefined\AbstractStandalone
\end{AbstractBody}
\end{document}
\fi
