\ifdefined\tool\else
\documentclass[uplatex, 10pt, a4j, dvipdfmx]{jsarticle}

\usepackage{packages/abstract}
\usepackage{enumitem}
\usepackage[dvipdfmx]{graphicx}
\usepackage{url}
\usepackage[dvipdfmx, hidelinks]{hyperref}
\usepackage{pxjahyper}

\newcommand{\tool}{TOOL}
\newcommand{\AbstractStandalone}{}

% make abstract コマンドでの発表要旨のPDF生成に必要な記述項目
\setAbstractTitle{長ったらしいタイトルを リアルタイムに描画するツール \tool{}の実装と評価}
\setAbstractPresenter{執筆者ネーム}

\begin{document}
\AbstractHeader
\begin{AbstractBody}
\fi

帳票の電子化は、スキャナやカメラで帳票を撮影することによって実現できる。
簡単に電子化できるというメリットがある一方で、帳票に記入した内容は、人が目視で確認する必要があるというデメリットがある。
効率的に記入内容を管理する方法の1つとして、電子フォームを用いて、その記入内容をデータとして保存する方法がある。

電子フォームを自動で作成するために、複数のツールが開発されている。
既存の電子フォーム作成ツールである、i-ReporterやCreate!Formは、帳票のExcelファイル、もしくは、画像ファイルを入力として、電子フォームを作成できる。
帳票を紙媒体で管理している場合にも、電子フォームを作成できるが、マウスのドラッグ操作で記入欄を配置する必要があり、記入欄の配置作業に時間がかかるという課題がある。

そこで本論文では、電子フォーム作成にかかる時間の削減を目的とし、記入欄の配置作業に着目して、記入欄自動検出およびラベル割付ツール\tool{}(Fields detection and Labels Assignment Tool)を開発する。

\tool{}は、以下の3つの機能を持つ。

\begin{itemize}
    \item 領域データ自動取得機能 \\
    領域データ自動取得機能は、帳票の画像(以下、帳票画像と呼ぶ)を入力として、帳票画像内にある矩形、および、下線で示された記入欄のデータをそれぞれ矩形領域データ、下線部領域データとして自動で取得する機能である。
    \item ラベル割付機能 \\
    ラベル割付機能は、帳票画像を入力として、領域データ自動取得機能で取得した矩形領域データ、および、下線部領域データにラベルを割り付ける機能である。
    ラベルを割り付けることで、バリデーションチェックに必要な情報を付与できる。
    \item 結果出力機能 \\
    結果出力機能は、帳票画像を入力として、領域データ自動取得機能で取得した領域データと、ラベル割付機能で取得したラベルを参照し、領域データとラベルの組をまとめたJSONファイルと、取得した領域とラベルを描画することによって、取得した領域を強調表示した領域強調画像を出力する機能である。
	領域強調画像を出力することで、出力するJSONファイルの内容を、目視で確認しやすくする。
	本機能で出力する領域強調画像は、帳票画像に、矩形、および、下線を、ランダムな色で描画する。
\end{itemize}

適用例を用いて、\tool{}が、領域データの取得と、ラベル割付が正常に動作することを確認した。
また、出力であるJSONファイルの内容と領域強調画像が正しいことと、それらが対応していることを確認した。

次に、\tool{}の有用性を評価するため、記入欄の配置作業にかかる時間を計測する評価実験を行った。
2枚の帳票画像について、GUIツールのみを用いた場合と比較して、\tool{}を使用した場合は、それぞれ平均で1分48秒(約43.25\%)、2分36秒(約45.87\%)短いことから、記入欄の配置作業にかかる時間を削減できることを確認した。

最後に、\tool{}を、帳票構造を解析する関連研究と比較した。
帳票構造を解析する関連研究のうち、事前準備が必要な関連研究と比較して、\tool{}は、帳票構造テンプレートの登録や、学習モデルの準備などの事前準備が不要であり、新たな種類の帳票にも、事前準備なしで対応できる。
また、帳票構造を解析する関連研究のうち、事前準備が不要な関連研究と比較して、電子文書と電子化文書の両方の帳票画像に対応しており、取得した領域データに対して、バリデーションチェックに必要な情報をラベルとして割り付けできる。

以上より、\tool{}は、記入欄の配置作業にかかる時間を削減できたことから、電子フォーム作成にかかる時間の削減に有用であると言える。

\ifdefined\AbstractStandalone
\end{AbstractBody}
\end{document}
\fi
