\chapter{はじめに}\label{cha:Introduction}
近年、ソフトウェア開発において、システムの大規模化および複雑化が進んでおり、ソフトウェアのバグが社会にもたらす影響は甚大なものになっている\cite{shippai-mizuho}。\\
特に、開発の上流工程では、自然言語による仕様記述を用いることがバグ混入原因の一つとなっている\cite{ipa-report}。
自然言語は本質的に曖昧さを含むため、解釈の違いによる誤解や抜け漏れが生じやすいという問題がある。
このような曖昧さは、実装段階でのバグ混入の原因となり、システム障害の発生や社会的影響の拡大につながる可能性がある。\\
この問題に対処する手法として、形式手法による仕様記述が挙げられる。
形式手法の一つであるVienna Development Method(VDM)は、
形式仕様記述言語を用いてシステムの振る舞いを厳密に定義できる手法であり、そのオブジェクト指向拡張であるVDM\texttt{++}は、
状態遷移や操作を明確に記述できる仕様記述言語として知られている。
VDM\texttt{++}は、曖昧さを排除した仕様記述を可能とし、信頼性の高いソフトウェア開発を支援することが可能である。\\
特に、画面遷移システムでは、状態遷移や条件分岐が頻繁に発生するため、
仕様の誤解や記述漏れが起こりやすい。
そのため、画面遷移やイベントの振る舞いを厳密に定義できるVDM\texttt{++}は有効な手段であると考えられる。\\
一方で、VDM\texttt{++}による仕様記述には、以下の2つの課題が存在する。
\begin{enumerate}
    \item VDM\texttt{++}の事前知識が必要であること
    \item 作成者によって仕様の粒度が異なること
\end{enumerate}
そこで、これらの課題を解決するために、VDM\texttt{++}仕様記述支援ツールである\toolFullName \cite{2vdm-spec-generator}が存在する。
\tool は、画面一覧や画面ロジックをMarkdown上で記述し、定義された変換ルールに基づいてVDM\texttt{++}仕様を生成することで、
VDM\texttt{++}仕様作成の効率化を図るものである。\\
しかし、\tool には、以下の3つの課題が存在する。
\begin{enumerate}
  \item 変換ルール特有の記述規則を理解する必要があること
  \item ボタンとイベントの対応漏れに気づきにくいこと
  \item プロジェクト規模が大きくなるにつれて、テキストベースの仕様管理では、画面遷移や条件分岐の全体像を把握することが困難になること
\end{enumerate}
そこで、本研究では、これらの課題を解決するために、\tool のGUI操作への拡張を行う。
本拡張では、GUI操作によって画面、ボタン、イベント、タイムアウトなどの要素を視覚的に編集可能とし、
画面遷移や条件分岐を直感的に把握できるようにする。
GUI上で編集された内容はMarkdown形式へと反映し、VDM\texttt{++}仕様へと変換することで、仕様作成の効率化、および保守性の向上を目指す。

以下、本論文の構成は次のとおりである。

第\ref{cha:Preparation}章では、\tool のGUI操作への拡張を行う際に必要となる前提知識について説明する。

第\ref{cha:Function}章では、拡張した\tool の機能について詳細に説明する。

第\ref{cha:Implementation}章では、拡張した\tool の実装について説明する。

第\ref{cha:Indication}章では、適用例を用いて拡張した\tool が正しく動作することを検証する。

第\ref{cha:Evaluation}章では、拡張した\tool について考察する。

第\ref{cha:Conclusion}章では、本研究のまとめと今後の課題を示す。
