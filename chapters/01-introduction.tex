\chapter{はじめに}\label{cha:Introduction}
近年、ソフトウェア開発において、システムの大規模化および複雑化が進んでおり、ソフトウェアのバグが社会にもたらす影響は甚大なものになっている\cite{shippai-mizuho}。

特に、開発の上流工程では、自然言語による仕様記述を用いることがバグ混入原因の一つとなっている\cite{ipa-report}。

自然言語はもともと曖昧さを含んでいるため、プログラマが自然言語による仕様書を仕様作成者の意図していない意味で読み取ることがある。
プログラマによる、仕様作成者の意図してない実装は、仕様作成当初の意図とは異なるソフトウェアを開発することになり、ソフトウェアにバグを持たせる可能性が高くなる。
そのため、自然言語による仕様書を用いたソフトウェア開発は、ソフトウェアの品質を低下させる危険性を帯びている。

この問題を解決するための方法の一つに

以下、本論文の構成は次のとおりである。

第\ref{cha:Preparation}章では、\tool のGUI操作への拡張を行う際に必要となる前提知識について説明する。

第\ref{cha:Function}章では、拡張した\tool の機能について詳細に説明する。

第\ref{cha:Implementation}章では、拡張部分の実装について説明する。

第\ref{cha:Indication}章では、適用例を用いて拡張した\tool が正しく動作することを検証する。

第\ref{cha:Evaluation}章では、拡張した\tool について考察する。

第\ref{cha:Conclusion}章では、本研究のまとめと今後の課題を示す。
