\chapter{はじめに}\label{cha:Introduction}
近年、ソフトウェア開発において、システムの大規模化および複雑化が進んでおり、ソフトウェアのバグが社会にもたらす影響は甚大なものになっている\cite{shippai-mizuho}。

特に、開発の上流工程では、自然言語による仕様記述を用いることがバグ混入原因の一つとなっている\cite{ipa-report}。

自然言語はもともと曖昧さを含んでいるため、プログラマが自然言語による仕様書を解釈する際に、仕様作成者の意図していない意味で解釈してしまうことがある。
プログラマによる、仕様作成者の意図してないソフトウェアの実装は、仕様作成当初の意図とは異なるソフトウェアを開発することになり、ソフトウェアにバグを混入させるリスクが高くなる。
そのため、自然言語による仕様書を用いたソフトウェア開発は、ソフトウェアの品質を低下させる危険性を帯びている。

ソフトウェア開発の上流工程において自然言語の曖昧さを回避し、誤解の挿入リスクを低減する方法の1つに、形式手法(Formal Method)\cite{formal-method}を用いた仕様記述が挙げれる。
形式手法は、ソフトウェアの仕様を数学的に厳密に表すことができるため、仕様作成者の意図をプログラマが正しく読み取ることができる。形式手法の1つにVDM(Vienna Development Method)\cite{vdm-1}がある。
VDMをオブジェクト指向に拡張した形式仕様記述言語として\VDM\cite{VDM++}がある。
\VDM は、状態と操作を厳密に定義できる形式仕様記述言語である。クラス、不変条件、事前条件、および、事後条件を用いることで、システムの状態遷移や制約を論理的に表現できるという特徴を持つ。このため、画面を状態、画面状態の変化を遷移として捉える画面遷移システムにおいて、遷移条件や状態遷移に伴う制約を明確に記述する手法として、\VDM が用いられる場合がある。

しかし、\VDM による仕様記述を行うためには、VDM++の構文や記述規則に関する事前知識が必要である。また、仕様記述者の理解度や経験により、記述粒度にばらつきが生じやすいという課題がある。
これらは、\VDM の導入における学習コストを高めている。

このような課題に対し、VDM++仕様を作成支援をするツールとして\tool がある。\tool は、Markdown形式の仕様記述からVDM++仕様を生成することができる。\tool を用いることで、自然言語に近い記述からVDM++仕様を生成することができるため、VDM++の導入における学習コストを低減することができる。

しかし、\tool では、特定の記述ルールに則ってMarkdown仕様を記述する必要があり、そのMarkdown仕様を作成する際の支援機能が不十分であり、\VDM 仕様作成に時間がかかってしまうという課題がある。

そこで本研究では、\tool にGUI操作による\VDM 仕様作成を可能にする拡張を行う。GUI操作による仕様作成を可能にすることで、仕様記述時の記述ミスを低減し、画面遷移構造を視覚的に把握しながら\VDM 仕様を作成できる環境を実現することで、\VDM 仕様作成時の支援機能を強化し、\VDM 仕様作成にかかる時間を短縮することを目的とする。

以下、本論文の構成は次のとおりである。

第\ref{cha:Preparation}章では、\tool のGUI操作への拡張を行う際に必要となる前提知識について説明する。

第\ref{cha:Function}章では、拡張した\tool の機能について詳細に説明する。

第\ref{cha:Implementation}章では、拡張部分の実装について説明する。

第\ref{cha:Indication}章では、適用例を用いて拡張した\tool が正しく動作することを検証する。

第\ref{cha:Evaluation}章では、拡張した\tool について考察する。

第\ref{cha:Conclusion}章では、本研究のまとめと今後の課題を示す。
