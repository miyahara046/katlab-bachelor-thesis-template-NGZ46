\chapter{はじめに}\label{cha:Introduction}

本テンプレートは、KatLab における修士論文執筆のための LaTeX 環境である。
このテンプレートを使用することで、Docker 環境で簡単に LaTeX 文書をコンパイルし、PDF を生成できる。

\section{このテンプレートの使い方}

\subsection{環境構築}
初回セットアップは以下のコマンドで行う:
\begin{verbatim}
make setup
\end{verbatim}

本テンプレートは Docker を使用しており、環境の再現性を確保している\cite{kimura-docker}。

\subsection{PDF の生成}
以下のコマンドで paper.pdf を生成できる:
\begin{verbatim}
make
\end{verbatim}

このコマンドを実行すると、\verb|chapters/| ディレクトリ内の .tex ファイルの変更を自動で監視し、変更があれば自動的に paper.pdf を再生成する。

\subsection{ファイル構成}
本テンプレートは以下のように構成されている:
\begin{itemize}
  \item \verb|paper.tex|: メインファイル(この内容は通常編集不要)
  \item \verb|chapters/|: 各章の .tex ファイルを配置するディレクトリ
  \item \verb|images/|: 画像ファイルを配置するディレクトリ
  \item \verb|paper.bib|: 参考文献データベース
\end{itemize}

\section{章立ての参照}

他の章を参照する場合は、\verb|\ref{}| コマンドを使用する。
例えば、第\ref{cha:Preparation}章では基本的な \LaTeX の書き方について説明している。

\section{本論文の構成}

以下、本論文の構成は次のとおりである。

第\ref{cha:Preparation}章では、\tool のGUI操作への拡張を行う際に必要となる前提知識について説明する。

第\ref{cha:Function}章では、拡張した\tool の機能について詳細に説明する。

第\ref{cha:Implementation}章では、拡張した\tool の実装について説明する。

第\ref{cha:Indication}章では、適用例を用いて拡張した\tool が正しく動作することを検証する。

第\ref{cha:Evaluation}章では、拡張した\tool について考察する。

第\ref{cha:Conclusion}章では、本研究のまとめと今後の課題を示す。
