\chapter{はじめに}\label{cha:Introduction}
近年、ソフトウェア開発において、システムの大規模化および複雑化が進んでおり、ソフトウェアのバグが社会にもたらす影響は甚大なものになっている\cite{shippai-mizuho}。
特に、開発の上流工程では、仕様記述に自然言語を用いることが、バグが混入する原因の1つとなっている\cite{ipa-report}。

自然言語はもともと曖昧さを含んでいるため、プログラマが自然言語による仕様書を解釈する際に、仕様作成者の意図していない意味で解釈してしまうことがある。
プログラマによる、仕様作成者の意図していないソフトウェアの実装は、仕様作成当初の意図とは異なるソフトウェアを開発することになり、ソフトウェアにバグが混入するリスクが高くなる。
そのため、自然言語による仕様書を用いたソフトウェア開発は、ソフトウェアの品質を低下させる可能性がある。

ソフトウェア開発の上流工程において、自然言語の曖昧さを回避し、厳密な仕様書を作成する方法の1つに、形式手法(Formal Method)\cite{formal-method}を用いた仕様記述が挙げられる。
形式手法は、ソフトウェアの仕様を数学的に厳密に表すことができるため、仕様作成者の意図をプログラマが正しく読み取ることができる。

形式手法の1つにVDM(Vienna Development Method)\cite{vdm-1}がある。
VDMをオブジェクト指向に拡張した形式仕様記述言語として\VDM\cite{VDM++}がある。
\VDM は、状態と操作を厳密に定義できる形式仕様記述言語である。クラス、不変条件、事前条件、および、事後条件を用いることで、システムの状態遷移や制約を論理的に表現できるという特徴を持つ。このため、画面を状態、画面状態の変化を遷移として捉える画面遷移システムにおいて、遷移条件や状態遷移に伴う制約を明確に記述する手法として、\VDM が用いられる場合がある。

しかし、\VDM による仕様記述を行うためには、\VDM の構文や記述規則に関する事前知識が必要であるという課題があり、
\VDM の導入コストを高めている。

この課題に対し、画面遷移システムを対象とした\VDM 仕様の作成支援ツール\tool(2vdm-spec-generator) がある\cite{2vdm-spec-generator}。\tool は、Markdown\cite{markdown}形式の仕様記述からVDM++仕様を生成できる。\tool を用いることで、自然言語に近い記述から\VDM 仕様を生成できるため、\VDM の導入における学習コストを低減することができる。

しかし、\tool は、特定の記述ルールに則ってMarkdown形式の仕様を記述する必要がある。このため、そのMarkdown形式の仕様を作成する際に時間がかかってしまうことで、\VDM 仕様の作成に時間がかかるという課題がある。

そこで本研究では、画面遷移システムの\VDM 仕様の作成にかかる時間の削減を目的として、\tool を拡張し、GUI操作による\VDM 仕様の作成を可能にする。GUI操作による\VDM 仕様の作成を可能にすることで、仕様記述時の記述ミスを低減し、画面遷移を視覚的に把握しながら\VDM 仕様を作成する。

本研究では、\tool をGUI操作による\VDM 仕様の作成を可能とするため、GUI操作で実際に操作するCondition Transition Map(CTM)を提案する。CTMは、画面遷移システムにおけるボタンとイベントの関係を表す。CTMは、以下の6つの構成要素を持つ。
\begin{itemize}
  \item 画面要素:システムの画面を表す要素
  \item ボタン要素:ある画面遷移システムにおいて、ユーザが操作可能なボタン(以降、本論文では、有効ボタンと呼ぶ)を表す要素
  \item イベント要素:ある画面遷移システムにおいて、対象の有効ボタンの押下時に発生する動作
  \item 分岐イベント要素:ある画面遷移システムにおいて、対象の有効ボタン押下時に条件付きで発生する動作を表す要素
  \item タイムアウト要素:ある画面遷移システムにおいて、一定時間内に特定の操作が行われなかった場合に発生する動作を表す要素
  \item 遷移先のないイベント要素:イベント動作に対応する画面要素が画面仕様に存在しない場合に表示する要素
\end{itemize}

本研究で拡張する\tool はこれらの要素を追加、編集、削除することでCTMを編集し、その編集したCTMに基づいて、画面遷移システムの\VDM 仕様を自動で生成する。

本研究では、株式会社フルタイムシステム\cite{fts}が開発し、実際に運用している画面遷移システムの画面の仕様を参考に、適用例で拡張した\tool が正しく動作するかを確認し、考察で評価実験を行うことで\VDM 仕様の作成にかかる時間を削減できていることを確認する。

本論文の構成を、以下に示す。

第\ref{cha:Preparation}章では、\tool のGUI操作への拡張を行う際に必要となる前提知識について説明する。

第\ref{cha:Function}章では、拡張した\tool の機能について詳細に説明する。

第\ref{cha:Implementation}章では、拡張部分の実装について説明する。

第\ref{cha:Indication}章では、適用例を用いて、拡張した\tool が正しく動作することを検証する。

第\ref{cha:Evaluation}章では、拡張した\tool について考察する。

第\ref{cha:Conclusion}章では、本研究のまとめと今後の課題を示す。
