\chapter{研究の準備}\label{cha:Preparation}
本章では、本研究を進めるにあたって必要となる基礎知識について説明する。
\section{画面遷移システム}\label{sec:ScreenTransitionSystem}

画面遷移システムとは、画面遷移図を工程成果物として持つことが可能なシステムの総称である\cite{screen-system}。
本研究における画面遷移システムは、以下をすべて満たすシステムである。
\begin{itemize}
\item システムが複数の画面を有する。
\item 各画面は、その画面内での動作や入力状況の違いを表す複数の内部状態を持ち、それぞれは区別可能な状態として定義する。
\item 各画面における状態は、トリガに応じて他の状態へ遷移する。この遷移は、以下のいずれかの形式を取る。
\begin{itemize}
\item 同一画面内の別状態への変化
\item 別の画面への遷移
\item 条件付きで変化先が決定される、内部状態または別の画面への遷移
\end{itemize}
\end{itemize}

本研究では、設計対象とする画面遷移システムの仕様を記述し、
それをもとに画面遷移を可視化およびVDM\texttt{++}仕様へ変換する。

\section{VDM\texttt{++}}\label{sec:VDM++}

形式手法の1つにVDM(Vienna Development Method)がある\cite{vdm-1}。
VDMは1970年代にウィーンのIBM研究所で考案され、1990年代前半にかけて開発された形式手法である。
1996年には、形式仕様記述言語VDM-SLがISO標準となっている\cite{vdm-1}。
VDM++とは、VDM-SLにオブジェクト指向拡張を施した形式仕様記述言語である\cite{VDM++}。
VDMには、VDMTools\cite{VDMTools}やOverture IDE\cite{Overture}などの支援ツールが揃っており、
仕様の検証を他の形式手法よりも比較的行いやすいという利点がある。
VDM\texttt{++}は、クラス定義において、型定義や関数定義などをブロックで定義する。
\VDM のファイルは .vdmppという拡張子を持つ。

VDM\texttt{++}における各定義について、以下で説明する。
\begin{itemize}
    \item 型定義

    型定義では、プログラムにおけるデータ型を定義する。VDM\texttt{++}における型は、大きく分けて
    基本型と合成型がある。基本型は型の最小構成要素となる定義で、合成型は基本型の組み合わせによってできる型である。

    \item 定数定義

    定数定義では、プログラムにおける定数を定義する。

    \item インスタンス変数定義

    インスタンス変数定義では、各オブジェクトがオブジェクト内に保持する属性について定義する。

    \item 関数定義

    関数定義では、オブジェクトの振る舞いの一部を定義する。また、VDM\texttt{++}の関数は入力引数の値だけで戻り値が定まり、インスタンス変数を参照することができない。
    VDM\texttt{++}の関数は、関数が何をすべきかの特性のみを記述する陰関数定義と、引数からどのように結果を計算すべきかアルゴリズムを示す陽関数定義の2つの定義スタイルがある。

    \item 操作定義

    操作定義では、オブジェクトの振る舞いの一部を定義する。VDM\texttt{++}の操作は入力引数を受け取った後に結果を返し、そのオブジェクト内の、あるいは参照しているオブジェクトのインスタンス変数を参照することができる。
    VDM\texttt{++}の操作は、操作の特性を記述する陰操作定義と、計算アルゴリズムを示す陽操作定義の2つの定義スタイルがある。
\end{itemize}

本研究では、画面遷移システムの仕様をVDM\texttt{++}で出力する。

\section{Markdown}\label{sec:markdown}

Markdownは、2004年にJohn Gruber氏によって開発された軽量マークアップ言語である\cite{markdown}。
Markdownを使用することで、プレーンテキストを使って簡単に文書を記述できる。Markdownのファイルは .mdという拡張子を持つ。
Markdownの主な記法を、表\ref{tab:Markdown_Example}に示す。

本研究では、
Markdownで記述した画面遷移システムの仕様を解析することで、本研究で提案するCTMの表示、および、VDM++仕様の生成を行う。以降、本論文では、Markdownで記述した画面遷移システムの仕様を、Markdowm仕様と呼ぶ。

\begin{table}[tp]
\centering
\caption{Markdownの主な記法}
\label{tab:Markdown_Example}
\begin{tabular}{|l|l|}
\hline
\textbf{表現の種類} & \textbf{記法の例} \\
\hline
見出し & \texttt{\#} 見出し\\ \hline
強調 & \texttt{**}強調\texttt{**} \\ \hline
箇条書き & \begin{tabular}{@{}l@{}} \texttt{-} 箇条書き1 \\ \texttt{-} 箇条書き2 \end{tabular} \\ \hline
引用 & \texttt{>} 引用 \\ \hline
リンク & [リンク名](リンクURL) \\ \hline
コード & \texttt{\`}コード\texttt{\`} \\ \hline
コードブロック & \begin{tabular}{@{}l@{}} \texttt{\`}\texttt{\`}\texttt{\`} \\ コードブロック\\\texttt{\`}\texttt{\`}\texttt{\`} \end{tabular} \\
\hline
\end{tabular}
\end{table}



\section{JSON}\label{sec:JSON}
JSON(JavaScript Object Notation)\cite{JSON}は、軽量のテキストベースのデータ交換フォーマットである。
JSONは、データをキーと値のペアで表現し、人間にも機械にも理解しやすい構造を持っている。
JSONはWebサービスやアプリケーションの設定ファイルとして広く利用されている。
JSON形式のファイルは .json という拡張子を持つ。
JSONは構造化データを簡潔に表現でき、言語や環境を問わずパース(解析)、および、シリアライズ(生成)が容易である

本研究では、Markdown仕様を解析した後、
CTMのノード配置の内部表現としてJSONを用いる。

\section{MVVM}\label{sec:MVVM}
MVVM(Model-View-ViewModel)\cite{MVVM}は、Model、View、ViewModelの 3 つのコンポーネントからなるソフトウェアアーキテクチャの一種である。
この3つのコンポーネントそれぞれが異なる役割を果たす。
3つのコンポーネントそれぞれの役割を、以下に示す。

\begin{itemize}
    \item Model:アプリケーションのデータをカプセル化する非ビジュアルクラスであり、データや業務ロジックを保持、および、提供する役割
    \item View:画面に表示するものの構造、レイアウト、外観を定義する役割
    \item ViewModel:ModelとViewとのやりとりを調整する役割
\end{itemize}

CommunityToolkit.Mvvm\cite{MVVMtool}は、高速モジュール式 MVVM ライブラリである。本ライブラリは、アプリを構築するための最初の実装を提供する、
標準型、自己完結型、軽量型のコレクションにアクセスするために使用する。

本研究の拡張では、MVVMに基づいた実装を行う。また、CommunityToolkit.MvvmのRelayCommandを使用する。
各メソッドにRelayCommand属性を付与することにより、拡張後の\tool 内でそのメソッドが非同期実行のコマンドとして扱えるため、重い処理を行っていてもUIを止めずに処理をバックグラウンドで実行できる。

\section{.NET}\label{sec:NET}
.NET は、Microsoft によって開発されたマルチプラットフォーム対応のアプリケーション開発基盤である\cite{NET}。
C\#を主なプログラミング言語として用い、デスクトップアプリケーション、Web アプリケーション、モバイルアプリケーションなど、さまざまな形態のソフトウェアを統一的な環境で開発できる。
.NET では、実行環境とともに、基本クラスライブラリと呼ばれる標準ライブラリ群が提供されている。

本研究で用いる.NETの標準ライブラリを、以下に示す。
\begin{itemize}
    \item System.Text.RegularExpressions:

    正規表現による文字列解析機能を提供する.NET 標準ライブラリである。本研究では、Markdown仕様を解析する際に用いる。
    \item System.Collections.Generic:

    ジェネリックコレクションを提供する.NET 標準ライブラリである。本研究では、Markdown仕様の解析結果やCTMの要素の集合を管理するために List\textless T \textgreater を用いている。
    また、画面名やボタン名の重複生成を防止するため、HashSet\textless T \textgreater を用いた一意性の管理を行う。
    \item System.IO:

    ファイルおよびディレクトリ操作を行うための .NET 標準ライブラリである。本研究では、Markdown 仕様ファイルの読み込み、JSON 形式の中間データの保存、および、生成した VDM\texttt{++} 仕様ファイルの書き込みに用いる。
\end{itemize}


\section{.NET MAUI}\label{sec:NET_MAUI}
拡張後の\tool では、.NET MAUI(.NET Multi-platform App UI)\cite{NET_MAUI}を用いて主なGUI部分を実装する。
.NET MAUI は、C\#およびXAMLを使用してクロスプラットフォームのアプリケーションを開発するためのフレームワークである。

本研究の拡張で使用する.NET MAUIの標準ライブラリを、以下に示す。

\begin{itemize}
    \item Microsoft.Maui.Controls:

    GUI を構成するための基本的な UI コンポーネントを提供する名前空間である。
    本研究では、「1画面」を表すページクラスであるContentPageを、各画面の基底クラスとして使用し、ボタン、ラベル、入力欄などのUI要素を配置することで、
    仕様編集画面や操作画面を構成する。
    また、Shellクラスを用いて、スタートページと仕様編集ページ間の画面遷移を管理する。
    さらに、UIコンテナである\texttt{ContentView}、描画用UIコンポーネントである\texttt{GraphicsView}、スクロール用コンテナである\texttt{ScrollView}を用いてCTMの表示を行う。

    \item Microsoft.Maui.Controls.Xaml:

    XAML による UI 定義を可能にする名前空間である。
    本研究では、画面レイアウトを XAML ファイルにより宣言的に記述し、画面構造と処理ロジックの分離を実現する。

    \item Microsoft.Maui.Graphics:

    クロスプラットフォームな 2D 描画 API を提供する名前空間である。
    本研究では、カスタム描画を行うための表示用UIコンポーネントであるGraphicsViewと、GraphicsView 上に何をどのように描画するかを定義するためのインタフェースであるIDrawableを用いて、
    CTMのノードや矢印を描画する処理を実装する。

    \item Microsoft.Maui.ApplicationModel:

    アプリケーションの実行環境やデバイス機能へのアクセスを提供する名前空間である。
    本研究では、非同期処理後に UI を更新する際に、
    .NET MAUI におけるユーザインタフェース操作用の API である MainThread を用い、
    UI スレッド上での処理制御を行う。


    \item Microsoft.Maui.Storage:

    アプリケーションで使用するファイルや設定情報を扱うための機構を提供する名前空間である。
    本研究では、ユーザが選択したプロジェクトフォルダやファイルパスを管理する処理に使用する。

    \item Microsoft.Maui.Input:

    タッチ、クリック、ドラッグなどの入力イベントを扱うための名前空間である。
    本研究では、クリックやドラッグ操作を検出し、GUI要素の選択や移動といったユーザ操作を取得するために用いる。

    \item Microsoft.Maui.Layouts:

    UI 要素の配置を制御するレイアウトコンテナを提供する名前空間である。
    本研究では、UI要素を行と列の格子状に配置するためのレイアウトコンテナであるGridや、UI要素を縦方向または横方向に順番に並べるためのレイアウトコンテナであるStackLayoutを用いて、レイアウトを構成する。
\end{itemize}

\section{CommunityToolkit.Maui}\label{sec:CommunityToolkit.Maui}

CommunityToolkit.Maui は、.NET MAUI アプリケーション開発を支援するために提供されている
オープンソースの拡張ライブラリである\cite{CommunityToolkit.Maui}。
本ライブラリは、.NET MAUI 標準機能を補完する形で、
UI コンポーネント、ビヘイビア、エフェクト、ダイアログ表示機構などを提供する。

本研究では、CommunityToolkit.Maui が提供する \texttt{Popup} 機構を利用している。
\texttt{Popup} 機構は、画面全体を遷移させることなく、一時的な入力画面や確認画面を表示するための UI コンポーネントである。

\section{Windows.Storage.Pickers}\label{sec:Windows.Storage.Pickers}

Windows.Storage.Pickersは、Windows プラットフォームにおいて、
ユーザにフォルダを選択させるための標準的なPickerAPIを提供する名前空間である\cite{FolderPicker}。
本ライブラリは、フォルダ選択ダイアログを通じて、
ユーザ操作によるパス取得を安全かつ統一的に行うための機構を提供する。

本研究で拡張する\tool では、ユーザがMarkdown仕様を管理するプロジェクトフォルダを選択する際に用いる。

\section{WinUI}\label{sec:WinUI}

WinUI(Windows UI Library)\cite{WinUI}は、
Windows 向けユーザインタフェースライブラリである。
WinUIは、Windowsアプリケーションにおける
UIコンポーネントの描画、および、マウスやキーボードなどの入力イベント処理を担うネイティブUIフレームワークである。

本研究で拡張する\tool では、ポインタイベントを実現するため、
WinUIが提供するポインタイベントAPIを利用している。


\section{\tool}\label{sec:2vdm-spec-generator}
\tool(2vdm-spec-generator)\cite{2vdm-spec-generator}は、
画面遷移システムの自然言語仕様から\VDM 仕様を作成するための
仕様記述作成手法を内部に持ち、
ユーザが記述したMarkdown仕様に対して、
当該手法に基づく解析、および、\VDM への変換処理を行うツールである。
本節では、\tool が採用している
画面遷移システムを対象とした\VDM 仕様記述作成手法について説明し、
その後、\tool の機能について述べる。

\subsection{画面遷移システムを対象とした\VDM 仕様記述作成手法}\label{sec:way-of-convert}

本手法では、Markdown仕様を一定の文法で記述することを前提とし、
その文法に基づいて、記述したMarkdown仕様から\VDM 仕様記述への変換を行う。
本手法が対象とするMarkdown仕様の構成を、以下に示す。

\begin{itemize}
\item システムに存在する画面の名前をすべて記載した、画面一覧仕様が存在すること。
\item 各画面に1対1で対応した、画面仕様が存在すること。
\end{itemize}

本手法では、\VDM による画面遷移システムの仕様は、システムにおける表示部分を表現する表示部クラス、
現在の画面の状態を管理する画面管理クラス、および、各画面の仕様に基づく画面定義を表現する画面クラスから成る。
なお、本手法において表示部クラスは、対象の仕様の内容にかかわらず一定のものとして定義する。

本手法で\VDM へ変換を行う画面一覧仕様の記述例を図\ref{fig:specRuleA}に、画面仕様の記述例(画面A)を図\ref{fig:specRuleB}に、それぞれ示す。
また、定義する表示部クラスの例をリスト\ref{lst:hyougibu}に、\VDM に変換後の図\ref{fig:specRuleA}に対応する画面管理クラスの例をリスト\ref{lst:VDMPP_ExampleA}に、
図\ref{fig:specRuleB}に対応する画面クラスの例をリスト\ref{lst:VDMPP_ExampleB}に、それぞれ示す。
\begin{figure}
    \centering
    \includegraphics[width=0.4\linewidth]{./images/specRuleA.png}
    \caption{画面一覧仕様の記述例}
    \label{fig:specRuleA}
\end{figure}

\begin{figure}
    \centering
    \includegraphics[width=0.8\linewidth]{./images/specRuleB.png}
    \caption{画面仕様の記述例}
    \label{fig:specRuleB}
\end{figure}

\begin{figure}[tp]
    \begin{lstlisting}[caption={表示部クラス}, label={lst:hyougibu}, language={VDM_PP}]
    class 「表示部」

    types
      「入力値」 = token;
      「表示」 = [「入力値」];

    instance variables
      入力値: 「入力値」;
      表示: 「表示」;

    operations
      public
        入力操作: 「入力値」 ==> ()
        入力操作(入力値) ==
          is not yet specified;
      public
        削除操作: () ==> ()
        削除操作() ==
          is not yet specified;

    end 「表示部」
    \end{lstlisting}
\end{figure}

\begin{figure}[tp]
    \begin{lstlisting}[caption={図\ref{fig:specRuleA}に対応する画面管理クラス}, label={lst:VDMPP_ExampleA}, language={VDM_PP}]
class (*@「@*)画面管理(*@」@*)

types
  (*@「@*)画面状態(*@」@*)= (*@「@*)画面A(*@」@*) | (*@「@*)画面B(*@」@*) | (*@「@*)画面C(*@」@*) | (*@「@*)画面D(*@」@*);

instance variables
  static 現在画面:(*@「@*)画面状態(*@」@*):= new(*@「@*)画面A(*@」@*)()

end(*@「@*)画面管理(*@」@*)
\end{lstlisting}
\end{figure}


\begin{figure}[tp]
\begin{lstlisting}[caption={図\ref{fig:specRuleB}に対応する画面クラス}, label={lst:VDMPP_ExampleB}, language={VDM_PP}]
class (*@「@*)画面A(*@」@*)
types
(*@「@*)ボタン状態(*@」@*)=(*@\vdmtag{非押下}@*) | (*@\vdmtag{1}@*) | (*@\vdmtag{2}@*) | (*@\vdmtag{3}@*) | (*@\vdmtag{確定}@*);
values
 タイムアウト時間 := 30 ;
instance variables
 押下ボタン:(*@「@*)ボタン状態(*@」@*):= (*@\vdmtag{非押下}@*);

operations
 private
  押下時操作:(*@「@*)ボタン状態(*@」@*)==> ()
  押下時操作 ((*@\texttt{押下ボタン}@*)) ==
    cases 押下ボタン:
        (*@\vdmtag{1}@*) -> 表示部(*@.@*)入力操作(1),
        (*@\vdmtag{2}@*) -> 表示部(*@.@*)入力操作(2),
        (*@\vdmtag{3}@*) -> 表示部(*@.@*)入力操作(3),
        (*@\vdmtag{確定}@*) -> if 表示部に1 が入力されている ()
                  then (*@「@*)画面管理(*@」@*)(*@\texttt{\`}@*)現在画面 := new(*@「@*)画面K(*@」@*)()
                  if 表示部に1 が入力されていない ()
                  then (*@「@*)画面管理(*@」@*)(*@\texttt{\`}@*)現在画面 := new(*@「@*)画面F(*@」@*)()
        others -> skip
      end
    pre 押下ボタン <> (*@\vdmtag{非押下}@*)
    post 押下ボタン = (*@\vdmtag{非押下}@*)
private
    タイムアウト時画面遷移: () ==> ()
    タイムアウト時画面遷移 () ==
    (*@「@*)画面管理(*@」@*)(*@\texttt{\`}@*)現在画面 := new(*@「@*)スタート画面(*@」@*)();
private
    表示部に1 が入力されている: () ==> bool
    表示部に1 が入力されている() == is not yet specified;
private
    表示部に1 が入力されていない: () ==> bool
    表示部に1 が入力されていない() == is not yet specified;

end (*@「@*)画面A(*@」@*)
\end{lstlisting}
\end{figure}

本手法では、画面遷移システムの\VDM 仕様を作成するために、以下の2つのルールを定義する。

\begin{itemize}
    \item 自然言語記述ルール(\ref{sec:Specrule}節で後述)

    Markdown仕様の自然言語記述ルール。以降本論文では、Markdown仕様の記述ルールと呼ぶ。
    
    \item \VDM 変換ルール(\ref{sec:ConvRule}節で後述)

    Markdown仕様の記述ルールに基づいたMarkdown仕様から、\VDM 仕様への変換ルール。
    以下の2つの変換ルールがある。
    \begin{itemize}
        \item 変換ルールA\\
        画面一覧仕様から画面管理クラスへの変換ルール。
        \item 変換ルールB\\
        画面仕様から画面クラスへの変換ルール。
    \end{itemize}
\end{itemize}

本手法の画面遷移システムの\VDM 仕様記述作成手順を、以下に示す。
\begin{enumerate}
  \item Markdown仕様の記述ルールに基づいて、Markdown仕様を作成する。
  \item 変換ルールAを用いて、1.で作成した画面一覧仕様から、画面管理クラスの\VDM 仕様記述を作成する。
  \item 表示部クラスの\VDM 仕様記述を作成する。
  \item 変換ルールBを用いて、1.で作成した各画面仕様から、画面クラスの\VDM 仕様記述をそれぞれ作成する。
\end{enumerate}

\subsection{Markdown仕様の記述ルール}\label{sec:Specrule}

本手法でのMarkdown仕様の記述ルールについて説明する。
画面一覧仕様におけるMarkdown仕様の記述ルールを、以下に示す。図\ref{fig:specRuleA}に示した画面一覧仕様は、以下のルールに則って記述している。
\begin{itemize}
\item Markdownで記述していること。
\item "画面一覧" をタイトルとしたレベル1見出しに続けて、システムに存在する画面の名前をアイテムとしたリストを記述していること。
このリスト記述範囲を、画面一覧フィールドとする。
\end{itemize}

画面仕様におけるMarkdown仕様の記述ルールを、以下に示す。図\ref{fig:specRuleB}に示した画面仕様は、以下のルールに則って記述している。

\begin{itemize}
    \item Markdownで記述していること。
    \item 画面の名前をタイトルとしたレベル2見出しを記述していること。
    このタイトル記述範囲を、画面名記述フィールドとする。
    \item 対象画面にタイムアウト時間が存在する場合、
    画面名記述フィールドに続けてタイムアウト時間をアイテムとしたリストを記述していること。
    このリスト記述範囲を、タイムアウト記述フィールドとする。
    \item "有効ボタン一覧" をタイトルとしたレベル3見出しに続けて、
    対象画面に存在するボタンの名前をアイテムとしたリストを記述していること。
    このリスト記述範囲を、有効ボタン記述フィールドとする。
    \item "イベント一覧" をタイトルとしたレベル3見出しに続けて、
    対象画面に存在するイベントをアイテムとしたリストを記述していること。
    このリスト記述範囲を、イベント記述フィールドとする。
    \item イベント記述フィールドにおいて分岐を伴うイベントが存在する場合、
    分岐条件と遷移先イベントをアイテムとしたサブリストを記述していること。
    このリスト記述範囲を、分岐イベント記述フィールドとする。
\end{itemize}

図\ref{fig:specRuleB}に示すように、各画面仕様には、その画面の名前を画面名記述フィールドに、
タイムアウト時間をタイムアウト記述フィールドに、有効ボタンを有効ボタン記述フィールドにリスト形式で記述し、想定するイベントをイベント記述フィールドにリスト形式で記述する必要がある。
加えて、イベント記述フィールドにおいて、ボタン押下時に、表示部に入力された値で遷移
先の画面が異なる場合は、分岐イベント記述フィールドに記述を追加する必要がある。

\subsection{\VDM 変換ルール}\label{sec:ConvRule}
本節では、Markdown仕様の記述ルールに基づいたMarkdown仕様から\VDM 仕様記述への変換ルールA、および、
変換ルールBについて説明する。

\subsubsection{変換ルールA}

Markdown仕様から\VDM 仕様記述へ変換を行う際、画面一覧仕様から画面管理クラスへの変換ルール(変換ルールA)に則って、記述を行う。
この変換ルールAを、表\ref{tab:gamenichiran}に示す。
表\ref{tab:gamenichiran}における【】で囲まれた部分は、対象とする画面一覧仕様の記述に依存する部分であることを表す。

\begin{table*}[tp]
    \centering
    \caption{画面一覧仕様から画面管理クラスへの変換ルール(変換ルールA)}
    \label{tab:gamenichiran}
    \begin{tabular}{|c|c|c|}
        \hline
        変換ルール & 画面一覧仕様表記 & 画面管理クラスの\VDM 仕様記述\\
        \hline \hline
    A-1 &
    \begin{tabular}{l}
    \# 画面一覧\\
    - 【画面A】\\
    - 【画面B】\\
    - 【画面C】\\
    ...
    \end{tabular} & 
    \begin{tabular}{l}
        class 「画面管理」\\
        types\\
        \hspace{4pt}「画面状態」 = 【画面A】\textbar【画面B】\textbar【画面C】...;\\
        instance variables\\
        \hspace{4pt}static 現在画面 : 「画面状態」 := new 「【画面A】」();\\
        end 「画面管理」
    \end{tabular} \\ \hline
    
    \end{tabular}
\end{table*}

\subsubsection{変換ルールB}

Markdown仕様からVDM\texttt{++}仕様記述へ変換を行う際、画面仕様から画面クラスへの変換ルール(変換ルールB)に則って、記述を行う。
この変換ルールBを、表\ref{tab:kakugamen}に示す。表\ref{tab:kakugamen}における【】で囲まれた部分は、対象とする画面仕様の記述に依存する部分であることを表している。

\begin{table*}[tp]
    \centering
    \scriptsize
    \renewcommand{\arraystretch}{0.8}
    \caption{画面仕様から画面クラスへの変換ルール(変換ルールB)}
    \label{tab:kakugamen}
    \begin{tabularx}{168mm}{|p{1mm}|p{59mm}|X|}
        \hline
    \multicolumn{1}{|c|}{変換ルール} & \multicolumn{1}{c|}{画面仕様表記} & \multicolumn{1}{c|}{画面クラスのVDM++仕様記述}\\
        \hline \hline

    \multicolumn{1}{|c|}{B-1} &
    \begin{tabular}{l}
    \#\#【画面の名前】
    \end{tabular} & 
    \begin{tabular}{l}
        class 【画面の名前】 is subclass of 「画面管理」\\
        ...\\
        end 【画面の名前】\\
    \end{tabular} \\ \hline

    \multicolumn{1}{|c|}{B-2} &
    \begin{tabular}{l}
    \#\#\# 有効ボタン一覧\\
    - 【ボタン1】\\
    - 【ボタン2】\\
    ...
    \end{tabular} &
    \begin{tabular}{l}
        types\\
        \hspace{4pt}「ボタン状態」 = \texttt{\textless{}}非押下\texttt{\textgreater{}} \hspace{1pt}\textbar\hspace{1pt} \texttt{\textless{}}\hspace{-4pt}【ボタン1】\hspace{-4pt}\texttt{\textgreater{}} \textbar\hspace{1pt} \texttt{\textless{}}\hspace{-4pt}【ボタン2】\hspace{-4pt}\texttt{\textgreater{}}...;\\
        instance variables\\
        \hspace{4pt} 押下ボタン: 「ボタン状態」 =: \texttt{\textless{}}非押下\texttt{\textgreater{}};\\
    \end{tabular} \\ \hline

    \multicolumn{1}{|c|}{B-3} &
    \begin{tabular}{l}
    \#\#\# イベント一覧\\
    - 【ボタン1】押下 → 表示部に【追加値】を追加\\
    - 【ボタン2】押下 → 表示部から値を削除\\
    - 【ボタン3】押下 →\\
    \hspace{8pt} - 【操作時分岐条件1】 → 【画面A】へ\\
    \hspace{8pt} - 【操作時分岐条件2】 → 【画面B】へ\\
    - 【ボタン4】押下 → 【画面C】へ\\
    ...\\
    ※ タイムアウトを除く
    \end{tabular} &
    \begin{tabular}{l}
    instance variables\\
    \hspace{4pt} 表示部 =: new 「表示部」();\\
    operations\\
    \hspace{4pt}private\\
    \hspace{8pt} 押下時操作: 「ボタン状態」 \texttt{==>} ()\\
    \hspace{8pt} 押下時操作(押下ボタン) ==\\
    \hspace{12pt} cases 押下ボタン:\\
    \hspace{16pt} \texttt{\textless{}}\hspace{-4pt}【ボタン1】\hspace{-4pt}\texttt{\textgreater{}} -\texttt{\textgreater{}} 表示部\hspace{1pt}.入力操作(\hspace{-4pt}【追加値】\hspace{-4pt}),\\
    \hspace{16pt} \texttt{\textless{}}\hspace{-4pt}【ボタン2】\hspace{-4pt}\texttt{\textgreater{}} -\texttt{\textgreater{}} 表示部\hspace{1pt}.削除操作(),\\
    \hspace{16pt} \texttt{\textless{}}\hspace{-4pt}【ボタン3】\hspace{-4pt}\texttt{\textgreater{}} -\texttt{\textgreater{}} if 【操作時分岐条件1】\hspace{-4pt}() then \\
    \hspace{76pt} 「画面管理」\texttt{\`}現在画面 := new「【画面A】」\hspace{-4pt}()\\
    \hspace{76pt} if 【操作時分岐条件2】\hspace{-4pt}() then \\
    \hspace{76pt} 「画面管理」\texttt{\`}現在画面 :=new「【画面B】」\hspace{-4pt}(),\\
    \hspace{16pt} \texttt{\textless{}}\hspace{-4pt}【ボタン4】\hspace{-4pt}\texttt{\textgreater{}} -\texttt{\textgreater{}} 「画面管理」\texttt{\`}現在画面 := new「【画面C】」\hspace{-4pt}(),\\
    \hspace{16pt} ...,\\
    \hspace{16pt} others -\texttt{\textgreater{}} skip\\
    \hspace{12pt} end\\
    \hspace{8pt} pre 押下ボタン \texttt{\textless{}}\texttt{\textgreater{}} \texttt{\textless{}}非押下\texttt{\textgreater{}}\\
    \hspace{8pt} post 押下ボタン = \texttt{\textless{}}非押下\texttt{\textgreater{}};\\
    \end{tabular} \\ \hline

    \multicolumn{1}{|c|}{B-4} &
    \begin{tabular}{l}
    \#\#\# イベント一覧\\
    ...\\
    - 【ボタン2】押下 →\\
    \hspace{8pt} - 【操作時分岐条件1】 → 【画面A】へ\\
    \hspace{8pt} - 【操作時分岐条件2】 → 【画面B】へ\\
    ...\\
    \end{tabular} &
    \begin{tabular}{l}
    operations\\
    \hspace{4pt}private\\
    \hspace{8pt} 【操作時分岐条件1】: () \texttt{==>} bool\\
    \hspace{8pt} 【操作時分岐条件1】() ==\\
    \hspace{12pt} is not yet specified;\\
    \hspace{4pt}private\\
    \hspace{8pt} 【操作時分岐条件2】: () \texttt{==>} bool\\
    \hspace{8pt} 【操作時分岐条件2】() ==\\
    \hspace{12pt} is not yet specified;\\
    \end{tabular} \\ \hline

    \multicolumn{1}{|c|}{B-5} &
    \begin{tabular}{l}
    - 【タイムアウト時間】秒でタイムアウト\\
    ...\\
    \#\#\# イベント一覧\\
    ...\\
    - タイムアウト → 【画面Dへ】\\
    \end{tabular} &
    \begin{tabular}{l}
    values\\
    \hspace{4pt} タイムアウト時間 := 【タイムアウト時間】;\\
    operations\\
    \hspace{4pt} private\\
    \hspace{8pt} タイムアウト時画面遷移: () \texttt{==>} ()\\
    \hspace{8pt} タイムアウト時画面遷移() ==\\
    \hspace{12pt} 「画面管理」\texttt{\`}現在画面 :=new「【画面D】」\hspace{-4pt}();
    \end{tabular} \\ \hline
    
    \end{tabularx}
\end{table*}

\subsection{機能}\label{sec:Function}
既存の\tool の外観を、図\ref{fig:old2VSG} に示す。
既存の\tool は、Markdown仕様を編集するファイル内容編集エリア、VDM++仕様を表示する変換後\VDM 記述表示エリアを持ち、Markdown仕様の作成を支援するため、
以下の機能を提供する。
\begin{figure}[tp]
  \centering
  \includegraphics[width= 1.0\linewidth]{./images/old2vsg.png}
  \caption{既存の\tool の外観}
  \label{fig:old2VSG}

\end{figure}

\begin{itemize}
    \item Markdownで記述した画面管理仕様と画面仕様で構成するMarkdown仕様の編集機能
    \item 変換ルールAおよび変換ルールBに基づく\VDM 仕様への変換機能
    \item 生成した\VDM 仕様の表示および保存機能
\end{itemize}

以降に、各機能について説明する。
\subsubsection{Markdown仕様の編集機能}
本機能は、
Markdown仕様の記述ルールに基づいたMarkdown仕様を
Markdownで記述、および、編集する機能である。
\tool は、
画面一覧仕様および各画面に対応する画面仕様を
Markdown仕様として管理し、
ユーザに対してファイル内容編集エリアを提供することで、
Markdown仕様の作成および修正を可能にする。

ユーザは、
ファイル内容編集エリアを用いて、
画面一覧仕様および画面仕様を直接記述し、
Markdown仕様の記述ルールに従ったMarkdown仕様を作成する。
また、
Markdown仕様ファイルの新規作成、
既存ファイルの編集、
および保存といった基本的なファイル操作を
ツール上から実行できるため、
Markdown仕様の作成作業を一元的に管理できる。

\subsubsection{変換ルールAおよび変換ルールBに基づく\VDM 仕様への変換機能}
本機能は、
Markdown仕様に対して、
変換ルールAおよび変換ルールBを適用し、
\VDM 仕様を生成する機能である。

\tool は、
画面一覧仕様に対して変換ルールAを適用し、
現在画面の状態を管理する画面管理クラスを生成する。
また、
各画面に対応する画面仕様に対して変換ルールBを適用し、
表示部クラスを利用した画面固有の振る舞いと
画面遷移を記述する画面クラスを生成する。

この処理により、
Markdown仕様から、
\VDM 仕様への変換を
2つの変換ルールに基づいて一貫して実行できる。
その結果、
\VDM 仕様作成における記述漏れや解釈の不整合を抑制できる。

\subsubsection{生成した\VDM 仕様の表示および保存機能}
本機能は、
変換ルールAおよび変換ルールBに基づく\VDM 仕様への変換機能によって生成した\VDM 仕様を変換後\VDM 記述表示エリアに表示し、
ファイルとして保存する機能である。

\tool は、
ファイル内容編集エリアとは独立した変換後\VDM 記述表示エリアを設け、
変換後の\VDM 仕様を即座に表示する。
これにより、
ユーザは、
Markdown仕様の編集と変換後\VDM 仕様の確認を、
同一画面内で繰り返し行える。
