\chapter{研究の準備}\label{cha:Preparation}
本章では、本研究を進めるにあたって必要となる基礎知識について説明する。
\section{画面遷移システム}\label{sec:ScreenTransitionSystem}

画面遷移システムとは、画面遷移図を工程成果物として持つことが可能なシステムの総称である。\cite{screen-system}。
本研究における画面システムの定義は以下の通りである。
\begin{itemize}
    \item システムに複数の画面が存在する。
    \item 1つの画面ごとに、1つ以上の内部状態が存在する。
    \item 1つの画面ごとに、以下のいずれか1つに変化する。
    \begin{itemize}
        \item 他の画面状態
        \item 条件と変化先の内部状態
    \end{itemize}
    \end{itemize}

\section{VDM\texttt{++}}\label{sec:VDM++}

形式手法の1つにVDM(Vienna Development Method)がある。\cite{VDM}。
\section{Markdown}\label{sec:Markdown}
\subsection{Markdig}\label{sec:Markdig}
\section{JSON}\label{sec:JSON}
\section{MVVM}\label{sec:MVVM}
\section{.NET}\label{sec:NET}
\subsection{.NET MAUI}\label{sec:NET_MAUI}
本ツールでは、.NET Multi-platform App UI(以下、.NET MAUI と表記する)を用いてGUI部分を実装している。
.NET MAUI は、C\#およびXAMLを使用してクロスプラットフォームのアプリケーションを開発するためのフレームワークである。
本節では、実装に用いている.NET MAUI の主要なライブラリについて説明する。
\subsection{Microsoft.Maui.Graphics}\label{sec:Microsoft.Maui.Graphics}
\section{CommunityToolkit.Maui}\label{sec:CommunityToolkit.Maui}
\section{CommunityToolkit.Maui.Mvvm}\label{sec:CommunityToolkit.Maui.Mvvm}
\section{System.Text.RegularExpressions}\label{sec:System.Text.RegularExpressions}
\section{System.IO}\label{sec:System.IO}
\section{Overture}\label{sec:Overture}
\section{\tool}\label{sec:2vdm-spec-generator}
\toolFullName(以下、\tool と表記する)\cite{2vdm-spec-generator}は、
画面遷移システムの仕様記述をMarkdown形式で作成し、
その記述に基づいてVDM\texttt{++}仕様を自動生成するツールである。
本節では、\tool の機能、仕様記述ルール、VDM\texttt{++}変換ルール、および構造について説明する。
\subsection{機能}\label{sec:Function}
\subsection{Markdown形式仕様記述ルール}\label{sec:Specrule}
\subsection{VDM\texttt{++}変換ルール}\label{sec:ConvRule}
\subsection{構造}\label{sec:Structure}