\chapter{研究の準備}\label{cha:Preparation}

本章では、LaTeX での論文執筆に必要な基本的な書き方について説明する。
LaTeX の詳細な文法については、文献\cite{kimura-latex}に詳しい解説がある。

\section{セクションとサブセクション}

\verb|\section{}| でセクション、\verb|\subsection{}| でサブセクションを作成できる。

\subsection{サブセクションの例}

このようにサブセクションを作成できる。
さらに細かく分けたい場合は \verb|\subsubsection{}| も使用できる。

\section{箇条書き}

\subsection{番号なし箇条書き}

\verb|itemize| 環境を使用する:

\begin{itemize}
  \item 項目1
  \item 項目2
  \item 項目3
\end{itemize}

\subsection{番号付き箇条書き}

\verb|enumerate| 環境を使用する:

\begin{enumerate}
  \item 最初の項目 \label{enum:first}
  \item 2番目の項目
    \begin{enumerate}
      \item サブ項目A \label{enum:sub-a}
      \item サブ項目B \label{enum:sub-b}
    \end{enumerate}
  \item 3番目の項目 \label{enum:third}
\end{enumerate}

リスト項目にもラベルを付けて参照できる。
例えば、\verb|\ref{enum:first}| で\ref{enum:first}を参照でき、
サブ項目は\verb|\ref{enum:sub-a}| で\ref{enum:sub-a}のように参照できる。
\verb|\cref| を使えば、\cref{enum:first}や\cref{enum:sub-a}のように参照できる。

\section{強調表示}

以下のようなフォントスタイルが使用できる:

\begin{itemize}
  \item \textbf{太字}: \verb|\textbf{太字}|
  \item \textit{Italic}: \verb|\textit{Italic}|
  \item \underline{下線}: \verb|\underline{下線}|
\end{itemize}

一部のフォントスタイルは、日本語に対応していないもの (\verb|\textit{斜体}| など) があるので注意。

\section{verbatim 環境}

コマンドやコードをそのまま表示したい場合は、\verb|verbatim| 環境またはインライン \verb|\verb| コマンドを使用する。

\begin{verbatim}
これは verbatim 環境の例です。
LaTeX のコマンドも \command{そのまま} 表示されます。
\end{verbatim}

インラインで表示する場合は、\verb|\verb|コマンド| のように書く。

\section{ラベルと参照}

\subsection{ラベルの付け方}

章、節、図、表などにラベルを付けることで、後から参照できる:

\begin{verbatim}
\section{セクション名}\label{sec:label_name}
\end{verbatim}

\subsection{参照の仕方}

ラベルを付けた箇所を参照するには、\verb|\ref{}| コマンドを使用する。
例えば、この章は第\ref{cha:Preparation}章であり、第\ref{cha:Introduction}章で本テンプレートの概要を説明した。
