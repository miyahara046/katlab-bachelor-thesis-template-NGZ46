\chapter{試作するツールの機能}\label{cha:Function}

本章では、図、表、数式の挿入方法について説明する。

\section{図の挿入}

図を挿入するには、\verb|figure| 環境と \verb|\includegraphics| コマンドを使用する。

\subsection{基本的な図の挿入}

図\ref{fig:example}に例を示す。

\begin{figure}[tp]
  \centering
  \includegraphics[width=0.5\linewidth]{./images/syokuji_computer.png}
  \caption{図の挿入例}
  \label{fig:example}
\end{figure}

上記の図は以下のコードで挿入できる:

\begin{verbatim}
\begin{figure}[tp]
  \centering
  \includegraphics[width=0.5\linewidth]{./images/syokuji_computer.png}
  \caption{図の挿入例}
  \label{fig:example}
\end{figure}
\end{verbatim}

\subsection{画像ファイルの配置}

画像ファイルは \verb|images/| ディレクトリに配置する。
サポートされている画像形式:PNG、JPG、PDF など。

\section{表の挿入}

表を挿入するには、\verb|table| 環境と \verb|tabular| 環境を使用する。

\subsection{基本的な表の挿入}

表\ref{tb:example}に例を示す。

\begin{table}[tp]
  \caption{表の挿入例}
  \label{tb:example}
  \centering
  \begin{tabular}{|l|c|r|}
    \hline
    左寄せ & 中央寄せ & 右寄せ \\
    \hline
    データ1 & データ2 & データ3 \\
    データ4 & データ5 & データ6 \\
    \hline
  \end{tabular}
\end{table}

上記の表は以下のコードで挿入できる:

\begin{verbatim}
\begin{table}[tp]
  \caption{表の挿入例}
  \label{tb:example}
  \centering
  \begin{tabular}{|l|c|r|}
    \hline
    左寄せ & 中央寄せ & 右寄せ \\
    \hline
    データ1 & データ2 & データ3 \\
    データ4 & データ5 & データ6 \\
    \hline
  \end{tabular}
\end{table}
\end{verbatim}

\section{数式の挿入}

\subsection{インライン数式}

文中に数式を挿入する場合は、\verb|$...$| または \verb|\(...\)| を使用する。
例:$E = mc^2$ は有名な公式である。

\subsection{ディスプレイ数式}

独立した行に数式を表示する場合は、\verb|equation| 環境を使用する。

\begin{equation}\label{eq:example}
  \int_0^\infty e^{-x^2} dx = \frac{\sqrt{\pi}}{2}
\end{equation}

式(\ref{eq:example})のように、数式にも番号とラベルを付けて参照できる。

\subsection{複数行の数式}

複数行の数式を表示する場合は、\verb|align| 環境を使用する:

\begin{align}
  a + b &= c \\
  x + y &= z
\end{align}

\section{図表の参照}
\label{sec:ref}

図や表を参照するには、\verb|\ref{}| コマンドを使用する:

\begin{itemize}
  \item 図\ref{fig:example}のように図を参照
  \item 表\ref{tb:example}のように表を参照
  \item 式(\ref{eq:example})のように数式を参照
\end{itemize}

より便利な自動参照コマンドも利用できる。
自動参照コマンドは、「図」「表」「式」などの名前を自動で判定して表示する。

\subsection{crefコマンド(推奨)}

\verb|\cref{}| コマンドを使うと、図表の種類を自動判定して日本語で表示する:

\begin{itemize}
  \item 図:\cref{fig:example}のように図を参照
  \item 表:\cref{tb:example}のように表を参照
  \item 式:\cref{eq:example}のように数式を参照
  \item 章:\cref{cha:Introduction}のように章を参照
  \item 節:\cref{sec:ref}のように節を参照
\end{itemize}

\subsection{autorefコマンド}

\verb|\autoref{}| コマンドも同様に使用できる:

\begin{itemize}
  \item 図:\autoref{fig:example}のように図を参照
  \item 表:\autoref{tb:example}のように表を参照
  \item 式:\autoref{eq:example}のように数式を参照
  \item 章:\autoref{cha:Introduction}のように章を参照
\end{itemize}

\textbf{注意:}章や節の参照は、\verb|\cref| を使うと「3章」「3.1節」のように表示され、
\verb|\autoref| では番号のみが表示される。より柔軟な表示が可能な \verb|\cref| の使用を推奨する。
