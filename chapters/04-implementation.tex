\chapter{拡張した2VSGの実装}\label{cha:Implementation}
本章では、拡張した\tool の実装について説明する。以下に実装の詳細について説明する。

\section{スタートページの実装}
本節では、本研究の拡張として新たに追加したスタートページの実装について説明する。
スタートページは、ツール起動時に最初に表示する画面であり、
Markdown形式による仕様記述を行うページ、
および、GUI操作によってVDM\texttt{++}仕様を編集するページの
二つの編集方式へ遷移するための入口として機能する。

スタートページには、
ツール名を示すタイトル表示領域と、
「Markdown仕様記述ページ」、
「GUI操作によるVDM\texttt{++}仕様編集ページ」
へ遷移する二つのボタンを配置している。
これにより、ユーザはツール起動直後に、
自身が採用したい仕様編集方式を明示的に選択できる。

各遷移ボタンには、
画面遷移コマンドを割り当てている。
これらのコマンドは、
.NET MAUIのインタフェースを用いて実装しており、
ボタン押下時に \texttt{Shell.Current.GoToAsync} を呼び出すことで、
対応するページへの画面遷移を実現している。
このとき、各ページは、
それぞれShellのルーティング機構を用いて識別し、、
アプリケーション全体で一貫した遷移管理を行う。

また、それぞれのページには、
「スタートページに戻る」ためのボタンを実装している。
このボタン操作を検知した場合も、
同様にShellを用いた画面遷移要求を発行し、
スタートページへ戻る処理を行う。
これにより、ユーザは編集方式を切り替えたい場合でも、
ツールを再起動することなく、
任意のタイミングでスタートページへ戻ることが可能である。

以上のように、
スタートページはツール全体の起点として、
編集方式の選択および画面遷移の統一的管理を担う。

本処理の流れを以下に示す。
\begin{enumerate}
  \item ツール起動時にスタートページを初期画面として表示する。
  \item ユーザが「Markdown」ボタンを押下した場合、
  Markdown仕様記述ページへ遷移する。
  \item ユーザが「NoCode」ボタンを押下した場合、
  GUI操作によるVDM\texttt{++}仕様編集ページへ遷移する。
  \item 各編集ページにおいて「スタートページに戻る」ボタンが押下された場合、
  スタートページへの遷移要求を受け付け、
  スタートページへ画面を遷移する。
\end{enumerate}


\section{GUI操作によるVDM\texttt{++}仕様編集ページの構成}
以降の節では、GUI操作によるVDM\texttt{++}仕様編集ページについて説明する。
本研究の拡張で追加した部分もGUI操作によるVDM\texttt{++}仕様編集ページ既存の\tool 同様にMVVMアーキテクチャ(\ref{sec:MVVM}節を参照)を採用している。MVVMアーキテクチャ実装のために本研究では
CommunityToolkit.Mvvm(\ref{sec:CommunityToolkit.Mvvm}節を参照)を使用している。拡張部分のシステム構成を、図\ref{fig:system-architecture}に示す。
拡張後の\tool は、以下の6つの主要な処理部で構成する。
\begin{itemize}
      \item View 層
      \begin{itemize}
            \item 描画部
      \end{itemize}
      \item ViewModel 層
      \begin{itemize}
            \item ユーザ操作監視部
      \end{itemize}
      \item Model 層
      \begin{itemize}
            \item プロジェクト管理部
            \item 解析部
            \item GUI要素生成部
            \item 変換部
      \end{itemize}
\end{itemize}

MVVM に基づき、各処理部は責務を明確にし、連携して動作する。
View 層は、GUIの表示、および、ユーザ入力の受付を担っており、描画部はCTMを可視化する。
ViewModel 層は、View 層と Model 層間のデータバインディングを担当しており、
ユーザ監視部は、ユーザ入力を、それぞれの操作について判断し、
Model 層の各処理部へ処理を指示する。
Model 層は、プロジェクト管理部、解析部、GUI要素生成部、変換部から構成し、
仕様データの永続化、解析、GUI要素データの生成、および、マークダウン形式からVDM++形式への変換を担う。
この構成により、Model 層は View 層を直接参照せず、
ViewModel 層を介して状態のみを受け渡す。

本ツールでは、CTM(\ref{sec:CTM}節を参照)領域に表示する各CTM要素が保持するデータをGUIElement(\ref{sec:ParsingComponent}節を参照)とする。\verb|GUIElement|の中で
以下に示すデータを保持するGUIElementをGUI要素と呼ぶ。
\begin{itemize}
      \item CTM上での座標(X、Y)
      \item CTM上でのサイズ(Width、Height)
      \item 選択状態フラグ(IsSelected)
      \item 可動性フラグ(IsMovable)
\end{itemize}

以降、各処理部の実装について、それぞれ説明する。

\begin{figure}[tp]
  \centering
  \includegraphics[width=0.8\linewidth]{./images/system-architecture.png}
  \caption{拡張した\tool のシステム構成}
  \label{fig:system-architecture}
\end{figure}


\section{プロジェクト管理部}\label{sec:ProjectManagementComponent}

本処理部は、以下の5つの主要な処理から構成する。
\begin{itemize}
  \item フォルダ選択処理
  \item フォルダおよびファイル探索処理
  \item ファイル読込処理
  \item ファイル更新処理
  \item 画面切り替え処理
  \item 新規ファイル作成処理
\end{itemize}

本処理部の入力は、ユーザ操作監視部(\ref{sec:UserOperationMonitoring}節を参照)によるユーザ操作イベント、および、プロジェクトフォルダである。
出力は、解析部、および、変換部へ渡すMarkdown文字列、GUI要素生成部へ渡すGUI配置情報(JSONファイル)、プロジェクトフォルダへ出力する各ファイル(Markdownファイル、VDM++ファイル、JSONファイル)である。

プロジェクト管理部は、ユーザが選択するプロジェクト(フォルダ配下のMarkdownファイル)の状態を管理し、
解析部(\ref{sec:ParsingComponent}節を参照)、および、変換部(\ref{sec:ConversionComponent}節を参照)へ渡す入力データ(Markdown文字列)、
およびGUI要素生成部(\ref{sec:GUIElementGenerationComponent}節を参照)へ渡すGUI要素の配置情報(JSONファイル)を準備するとともに、
編集結果をファイルへ出力する処理部である。
本処理部でのファイルシステム入出力には 、System.IO (\ref{sec:NET}節を参照)を用いる。

本処理部のシステム構成を、図\ref{fig:proj-system}に示す。

以降、各処理について説明する。

\begin{figure}[tp]
  \centering
  \includegraphics[width=1.0\linewidth]{./images/proj-system.png}
  \caption{プロジェクト管理部のシステム構成図}
  \label{fig:proj-system}
\end{figure}
\subsection{フォルダ選択処理}\label{sec:PM_SelectFolder}

フォルダ選択処理は、ユーザが編集対象のプロジェクトフォルダを選択した際に、
選択フォルダの絶対パスを取得して状態変数 \verb| SelectedFolderPath| に保持し、
フォルダおよびファイル探索処理(\ref{sec:PM_FolderFileSearch}節を参照)を起動して
ツリー表示用データ \verb|ObservableCollection<FolderItem> FolderItems| を構築する処理である。

本処理では、
System.IO
(\ref{sec:NET}節を参照)を用いて、
フォルダおよびファイルの列挙処理を行う。

本処理の流れを、以下に示す。

\begin{enumerate}
  \item フォルダ選択要求の受理

  ユーザ操作監視部からのフォルダ選択要求、および、プロジェクトフォルダのパス\verb|folder.Path|を入力として、
  本処理を実行する。

  \item 選択フォルダパスを内部状態として保持

  \verb|folder != null| の場合に限り、
  2.で取得したフォルダパス\verb|folder.Path|を
  \verb|SelectedFolderPath|に代入して、
  選択フォルダパスとして\verb|SelectedFolderPath| をプロジェクト管理部が保持する。

  \item フォルダおよびファイル探索処理の呼出し

  フォルダおよびファイル探索処理へ探索要求を行い、
  \verb|SelectedFolderPath| 配下のフォルダおよび
  Markdown ファイルを走査して
  ツリー表示用データ構造である
  \verb|FolderItems| を構築する
  (\ref{sec:PM_FolderFileSearch}節を参照)。
  なお、\verb|FolderItems| はユーザ操作監視部へ渡し、
  ユーザ操作監視部が保持する。

  \item 選択状態の初期化

  フォルダ選択直後の状態として、
  現在選択中のアイテムを
  選択フォルダ自身に対応付ける。
  これにより、フォルダ直下にファイルが存在しない場合でも、
  編集対象のプロジェクトフォルダを一意に保持できる

\end{enumerate}

\subsection{フォルダおよびファイル探索処理}\label{sec:PM_FolderFileSearch}

フォルダおよびファイル探索処理は、探索要求を受け取り、
選択した、プロジェクトフォルダ配下のフォルダ、および、Markdown ファイルを探索し、
プロジェクトフォルダ内に存在する、フォルダ、および、ファイルを表現するデータ構造である\verb|FolderItem|を構築する。
その後、\verb|FolderItem|のコレクションである、
\verb|FolderItems| を構築する処理である。
\verb|FolderItem| は、表\ref{tb:FolderItem}に示すデータ構造を持つ。

本処理では、
System.IO
(\ref{sec:NET}節を参照)を用いて、
フォルダおよびファイルの列挙を行う。

本処理の流れを、以下に示す。

\begin{enumerate}
  \item 探索処理の初期化

  選択フォルダパス\verb|SelectedFolderPath| が空でないことを確認した後、
  \verb|FolderItems| を初期化する。

  \item ルートフォルダ直下の要素の探索

  \verb|SelectedFolderPath| 配下のフォルダおよびファイルを探索する。
  探索時には、
  フォルダと Markdown ファイル(拡張子 \verb|.md|)のみを探索対象とし、
  それ以外のファイルは対象外とする。

  フォルダの要素の探索を、以下に示す

  \begin{enumerate}[label=\roman*.]
  \item フォルダ要素の登録

  発見した各フォルダについて、
  表示用ノードとして \verb|FolderItem| を生成する。生成の際、プロジェクトフォルダ直下(\verb|FolderItem.Level == 0|)の時、\verb|FolderItem.IsVisible|を\verb|True|にし、
  プロジェクトフォルダ直下でない時(\verb|FolderItem.Level != 0|)の時\verb|FolderItem.IsVisible|を\verb|false|にする。
  \verb|FolderItem|を生成後、生成した\verb|FolderItem|を\verb|FolderItems| に追加する。

  \item Markdown ファイルの要素の登録

  発見した Markdown ファイルについて、
  ファイル表示用の \verb|FolderItem| を生成する。3.同様に、生成の際、プロジェクトフォルダ直下(\verb|FolderItem.Level == 0|)の時、\verb|FolderItem.IsVisible|を\verb|True|にし、
  プロジェクトフォルダ直下でない時(\verb|FolderItem.Level != 0|)の時、\verb|FolderItem.IsVisible|を\verb|false|にする。
  \verb|FolderItem|を生成後、生成した\verb|FolderItem|を\verb|FolderItems| に追加する。

  \item 画面索引辞書(\verb|screenIndex|)の設定

      発見した各 Markdown ファイルについて、画面クラスのクラス名(以降、画面クラス名と呼ぶ)から当該ファイルを特定できるよう、
      画面索引辞書(\verb|screenIndex|)へ対応関係を登録する。
      登録にあたり、Markdown 文字列の先頭行を取得し、
      「\verb|# 画面一覧|」である場合は画面管理クラスとして扱うため索引登録を行わない。
      一方で、先頭行が「\verb|##|」で始まる場合は画面クラスとして扱い、
      先頭行から抽出した画面クラス名をキー、Markdown ファイルの絶対パスを値として登録する。
      \verb|screenIndex| はプロジェクト管理部が保持し、
      画面切り替え処理(\ref{sec:PM_ScreenTransitionProcess}節を参照)で利用する。

  \item 再帰的探索の実行

  探索中のフォルダ内にサブフォルダが存在する場合には、
  i.~iii.の手順を再帰的に適用し、
  階層構造を \verb|FolderItem.Level| により表現した
  フラットなリスト構造として
  \verb|FolderItems| に追加する。

  \end{enumerate}

  \item \verb|FolderItems|の出力

  探索結果として構築した \verb|FolderItems| をユーザ操作監視部へ出力する。
\end{enumerate}


\begin{table}[tp]
\centering
\caption{FolderItemデータ構造}
\label{tb:FolderItem}
\begin{tabular}{|l|l|p{9cm}|}
\hline
\textbf{属性名} & \textbf{型} & \textbf{説明} \\
\hline
Name & string & 表示名(フォルダ名またはファイル名) \\
\hline
FullPath & string & 絶対パス(フォルダまたはファイル) \\
\hline
Level & int & ルートからの階層レベル(インデント表示に利用) \\
\hline
IsExpanded & bool & 展開状態\\
\hline
IsVisible & bool & ツリー表示における可視状態\\
\hline
IsFolder & bool & フォルダかを判定するフラグ \\
\hline
IsFile & bool & ファイルかを判定するフラグ \\
\hline
\end{tabular}
\end{table}


\subsection{ファイル読込処理}\label{sec:PM_FileLoadProcess}

ファイル読込処理は、
ユーザがツリー表示上で選択した Markdown ファイルを対象として、
その内容を文字列として読み込み、操作ボタン領域の表示する操作ボタンの判定を行う表示パターンフラグの切替を行い、
解析部、および、変換部で利用する内部データを初期化する処理である。

表示パターンフラグは以下の3つのbool値の変数である。
\begin{itemize}
      \item \verb|IsScreenListAddButtonVisible|
      \item \verb|IsClassAllButtonVisible|
      \item \verb|IsClassAddButtonVisible|
\end{itemize}

フォルダ選択処理を行う前までは表示パターンフラグのすべての変数に\verb|false|を入力している。

本処理では、
System.IO
(\ref{sec:NET}節を参照)を用いて、
ファイル内容の読込を行う。

本処理の流れを、以下に示す。

\begin{enumerate}
  \item ファイル選択要求の受理

  ユーザ操作監視部からのファイル選択要求を受け取り、
  選択対象が Markdown ファイルであることを確認する。

  \item 選択ファイルパスの確定

  ユーザが選択したファイルの絶対パスを
  プロジェクト管理部が\verb|SelectedFilePath| として保持する。

  \item Markdown ファイルの内容の読込

  \verb|SelectedFilePath| が示す Markdown ファイルを読み込み、
  その内容の文字列を抽出する。この抽出した文字列を以降ではMarkdown 文字列と呼ぶ。

  \item Markdown 文字列の内部状態への保持

  3.で抽出した Markdown 文字列を、
  プロジェクト管理部が\verb|markdownText| として保持する。

  \item 上部操作ボタン表示フラグの決定および切替

  \verb|markdownText| の先頭から空行を除いた最初の行を取得し、
  その内容に基づいてツール上部の操作ボタン表示パターンフラグを切り替える。
  
  先頭行が、画面一覧を示す見出し(\verb|# 画面一覧|)である場合は、
  画面一覧編集用の操作を表示するため、\verb|IsScreenListAddButtonVisible|を\verb|true|にし、\verb|IsClassAllButtonVisible| と\verb|IsClassAddButtonVisible| を\verb|false|に切り替える  
  
  先頭行が、画面定義を示す見出し(\verb|##|)で始まる場合は、
  画面詳細編集用の操作を表示するため、\verb|IsClassAllButtonVisible|を\verb|true|にし、\verb|IsScreenListAddButtonVisible| と\verb|IsClassAddButtonVisible| を\verb|false|に切り替える
  
  上記以外の場合は、初期状態としてクラス追加用の操作を表示するため、\verb|IsClassAddButtonVisible|を\verb|true|にし、\verb|IsScreenListAddButtonVisible| と\verb|IsClassAllButtonVisible| を\verb|false|に切り替える
  本切替により、ユーザは現在編集中の仕様に応じた操作のみを選択可能となる。

  \item 解析部の呼出し

  4.で保持した\verb|markdownText|を入力として、
  解析部(\ref{sec:ParsingComponent}節を参照)を呼び出す。
  解析部は Markdown 文字列を解析し、
  GUIElementのリストである\verb|elements|を生成しプロジェクト管理部が受け取る。

  \item GUI 要素配置情報の存在確認

  選択対象 Markdown ファイルと同名の
  GUI 要素配置情報ファイル
  (拡張子 \verb|.positions.json|)が存在するかを確認する。

  \item GUI要素生成部の呼出し

      GUI 要素配置情報ファイルが存在する場合は、
      そのファイルを読み込む。
      読み込んだGUI要素配置情報ファイルと解析部から受け取った\verb|elements|を入力として、
      GUI要素生成部(\ref{sec:GUIElementGenerationComponent}節を参照)を呼び出す。
      GUI要素生成部は、
      \verb|elements| と \verb|.positions.json| の情報をもとに、
      GUI要素の配置情報を設定した上で、\verb|elements|をユーザ操作監視部へ渡す。
\end{enumerate}

\subsection{ファイル更新処理}\label{sec:FileUpdateProcess}
ファイル更新処理は、
ユーザによる編集結果をユーザ操作監視部から受け取り、
Markdown 仕様、VDM\texttt{++} 仕様、
および GUI 要素配置情報(JSONファイル)として
プロジェクトファイルへ反映する処理である。
本処理では、
System.IO(\ref{sec:NET}節を参照)、
および System.Text.Json(\ref{sec:JSON}節を参照)
を用いて、
各種ファイルの保存を行う。

本処理の流れを、以下に示す。

\begin{enumerate}
  \item ファイル更新要求の受理

  ユーザ操作監視部からのファイル更新要求を受け取り、
  本処理を実行する。

  \item Markdown 仕様の生成

  ユーザ操作監視部から\verb|elements| を入力として、
  変換部の
  GUI 操作から Markdown 仕様への変換処理
  (\ref{sec:GUItoMarkdownConversionProcess}節を参照)を呼び出す。
  変換処理により、
  更新後の Markdown 文字列
  \verb|updatedMarkdown| を生成する。

  \item Markdown ファイルの保存

  2.で生成した \verb|updatedMarkdown| を、
  \verb|SelectedFilePath| が示す Markdown ファイル(\verb|.md|)へ
  書き込み、保存する。

  保存する際、プロジェクト管理部内で保持する、\verb|markdownText|の画面クラス名と、\verb|updatedMarkdown|の画面クラス名を比較し、クラス名がが変わっている場合は\verb|screenIndex|(\ref{sec:PM_ScreenTransitionProcess}節を参照)の画面クラス名を変更する。

  \item VDM\texttt{++} 仕様の生成

  3.で保存した Markdown 文字列 \verb|updatedMarkdown| を入力として、
  変換部の
  Markdown 仕様から VDM\texttt{++} 仕様への変換処理
  (\ref{sec:ConversionComponent}節を参照)を呼び出す。
  これにより、
  VDM\texttt{++} 文字列 \verb|VdmContent| を生成する。

  \item VDM\texttt{++} ファイルの保存

  4.で生成した \verb|VdmContent| を、
  選択中 Markdown ファイルと同名の
  VDM\texttt{++} ファイル(\verb|.vdmpp|)へ書き込み、保存する。
  ファイルが存在しない場合は新規作成する。
  また、\verb|VdmContent|はユーザ操作監視部が内部で保持するため、ユーザ操作監視部へ出力する

  \item GUI 要素配置情報の抽出

  \verb|elements| が持つ各 GUI 要素について、
  要素名 \verb|GUIElement.Name| と
  座標 \verb|(GUIElement.X, GUIElement.Y)| を抽出する

  \item JSON形式のGUI要素配置情報の生成

  6.で抽出した(要素名, 座標)の組をもとに、
  JSON 形式のGUI要素配置情報
  \verb|positionsJson| を生成する。

  \item 配置情報ファイルの保存

  7.で生成した \verb|positionsJson| を、
  プロジェクト管理部内で保持する、\verb|SelectedFilePath| の Markdown ファイルと同名の
  \verb|.positions.json| ファイルへ書き込み、保存する。
  同名のファイルが存在しない場合は新規作成する。

  \item \verb|markdownText|の更新

  プロジェクト管理部内で保持する、\verb|markdownText|に、2.で生成した \verb|updatedMarkdown|を保存する。
\end{enumerate}

JSONファイルの出力例を、コード\ref{lst:JSON_example}に示す。このJSONファイルには、GUI要素の名称とCTM領域での座標情報を含み、自然言語仕様記述ルール(\ref{sec:Specrule}節を参照)に含まない位置情報のみを保持する。

この処理により、
GUI表示、Markdown仕様、VDM++仕様、および配置情報が
常に同一の仕様内容を表す状態を維持する。

\begin{figure}[tp]
\begin{lstlisting}[caption={JSONファイル出力例}, label={lst:JSON_example}, language={}]
[
  {
    "Name": "1",
    "X": 20,
    "Y": 111
  },
  {
    "Name": "2",
    "X": 20,
    "Y": 211
  },
  {
    "Name": "3",
    "X": 20,
    "Y": 311
  }
]
\end{lstlisting}
\end{figure}

\subsection{画面切り替え処理}\label{sec:PM_ScreenTransitionProcess}

画面切り替え処理は、ユーザ操作監視部から画面切り替え要求(\ref{sec:UOM_DoubleClick}節を参照)、および、切り替え先の画面クラス名を入力として受け取り、
切り替え先Markdownファイルのパスを特定し、
ファイル読込処理(\ref{sec:PM_FileLoadProcess}節を参照)へ
ファイル読込要求、および、Markdownファイルパスを出力することで編集対象となる画面クラスを切り替える処理である。

本処理の流れを、以下に示す。

\begin{enumerate}
  \item ユーザ操作監視部から画面要素のダブルクリック操作により、
        画面切り替え要求および切り替え先画面クラス名を受け取る。

  \item プロジェクト管理部が保持している画面索引辞書(\verb|screenIndex|)を参照し、
        切り替え先画面クラス名\verb|GUIElement.Name|に対応する Markdown ファイルパスを特定する。

  \item 対応する Markdown ファイルが存在する場合、
        対象ファイルを次の編集対象として設定する。

        対応する Markdown ファイルが存在しない場合はエラーをユーザ操作監視部に出力し、処理を中断する。

  \item 編集対象の変更に伴い、
        ファイル読込処理へファイル読込要求、および、\verb|SelectedFilePath|へ対象Markdownファイルのパスを出力する。
\end{enumerate}

\subsection{新規ファイル作成処理}\label{sec:PM_FileCreate}

新規ファイル作成処理は、ユーザ操作監視部からファイル作成要求(\ref{sec:UOM_EditCommand}節を参照)、および、ファイル名を入力として受け取り、
新規Markdownファイルを作成、および、\verb|SelectedFilePath|へ新規Markdownファイルのパスを格納し、
ファイル読み込み処理へファイル読込要求、フォルダおよびファイル探索処理へ探索要求、プロジェクトフォルダへ新規Markdownファイルを出力する処理である。

本処理の流れを以下に示す。
\begin{enumerate}
      \item ユーザ操作監視部からファイル作成要求、および、ファイル名を入力として受け取る。
      \item プロジェクト管理部内で保持する、\verb|SelectedFolderPath|をもとに新規Markdownファイルのパスを生成する。
      \item プロジェクト管理部内で保持する、\verb|SelectedFilePath|へ新規Markdownファイルのパスを格納する。
      \item 2.で作成したパスへ空のマークダウンファイルを作成する。
      \item フォルダおよびファイル探索処理へ探索要求を出力する。
      \item ファイル読み込み処理へファイル読込要求を出力する。
\end{enumerate}

\section{解析部}\label{sec:ParsingComponent}

解析部は、記述ルールに従ったMarkdown仕様(コード\ref{lst:markdown_example}を参照)の文字列を解析し、CTM(\ref{sec:CTM}節を参照)表示、および、操作の基礎となる構造データを生成する処理部である。
本処理部では、表\ref{tab:ctm_elements}に示した各CTM要素を構成するデータを抽出することを目的とする。

本処理部では、.NETの標準ライブラリであるSystem.Text.RegularExpressions(\ref{sec:NET}節を参照)を用いて Markdown文字列の各行を解析する。

本処理部の入力はプロジェクト管理部から受けとるMarkdown文字列であり、出力はプロジェクト管理部に渡すGUIElementのリストである\verb|elements|である。
ここで、GUIElement は、各CTM 要素を構成する構造データである。GUIElementのデータ構造を、表\ref{tb:GUIElement}に示す。
また、条件分岐イベント要素はGUIElementに加え、分岐条件とイベントを格納するリスト\verb|EventBranch|を用いてデータを保持する。\verb|EventBranch|のデータ構造を、表\ref{tb:EventBranch}に示す。

具体例として、コード\ref{lst:markdown_example}に示したMarkdown仕様を活用する。
コード\ref{lst:markdown_example}のMarkdown仕様に対して本処理部が生成する
GUIElement一覧を表\ref{tab:gui_element_example}に、\verb|EventBranch|を表\ref{tab:event_branch_example}にそれぞれ示す。
本ツールでは、自然言語仕様記述ルール(\ref{sec:Specrule}節を参照)に基づき、Markdown仕様中で「\verb|###|」で始まる行を見出し行と呼ぶ。

以降、本処理部の解析処理について説明する。
\subsection{解析処理}\label{sec:ParsingProcess}

本処理は、Markdown仕様から、CTM要素である、画面要素、タイムアウト要素、ボタン要素、イベント要素、および条件分岐イベント要素を抽出し、
抽出したCTM要素をGUIElementとして構造化し、GUIElementのリストである\verb|elements|を出力する処理である。

本処理は、ユーザがプロジェクト内のMarkdown仕様ファイルを選択または更新した際に、
プロジェクト管理部が呼び出す。
本処理はファイル単位で実行し、
各ファイルの内容に基づいて \verb|elements| を生成する。

本処理では、独自に定めた3つの正規表現パターンを用いてMarkdown文字列の各行を解析し、各CTM要素を抽出する。
各CTM要素の抽出に使用する独自に定めた3つの正規表現パターンを、表\ref{tab:parsing_rules}に示す。
本処理部では正規表現パターンの1つである\verb|BulletPattern|によって抽出した文字列を箇条書き本文と呼ぶ。

本処理の流れを、図\ref{fig:parsing-flow}と以下に示す。

\begin{table}[tp]
\centering
\caption{GUIElementデータ構造}
\label{tb:GUIElement}
\begin{tabular}{|l|l|p{7cm}|}
\hline
\textbf{属性名} & \textbf{型} & \textbf{説明}\\ 

\hline
Type & GuiElementType & CTM要素の種類(Screen, Button, Event, Timeout) \\

\hline
Name & string & CTM要素の名称 \\

\hline
Target & string & 遷移先となるCTM要素名 \\

\hline
X & float & CTM上での X 座標 \\

\hline
Y & float & CTM上での Y 座標 \\

\hline
Width & float & CTM要素の横幅 \\
  
\hline
Height & float & CTM要素の高さ \\

\hline
IsSelected & bool & CTM要素がユーザにより選択されているかどうか \\

\hline
IsMovable & bool & ドラッグによる移動が可能かどうか \\

\hline
Branches & List\textless EventBranch \textgreater & 条件分岐を表す分岐リスト \\


\hline
IsBranch & bool & Branchesを保持しているかどうか \\

\hline
\end{tabular}
\end{table}

\begin{table}[tp]
\centering
\caption{EventBranchデータ構造}
\label{tb:EventBranch}
\begin{tabular}{|l|l|p{10cm}|}
\hline
\textbf{属性名} & \textbf{型} & \textbf{説明}\\ 
\hline
EventBranch.Condition & string & 分岐条件 \\

\hline
EventBranch.Target & string & 分岐条件成立時のイベント \\
\hline
\end{tabular}
\end{table}

\begin{table}[tp]
\centering
\caption{コード3.1のMarkdown仕様記述例から生成するGUIElement一覧}
\label{tab:gui_element_example}
\begin{tabular}{|l|l|l|l|}
\hline
\textbf{Type} & \textbf{Name} & \textbf{Target} & \textbf{備考} \\
\hline
Screen & 画面1 & -- & 画面クラス \\ 
\hline
Timeout & 80 秒 & -- & タイムアウト定義 \\ 
\hline
Button & ボタン1 & 表示部に1 を追加 & 有効ボタン \\ 
\hline
Button & ボタン2 & 表示部に2 を追加 & 有効ボタン \\ 
\hline
Button & ボタン3 & 表示部に3 を追加 & 有効ボタン  \\ 
\hline
Button & ボタン4 & 表示部に4 を追加 & 有効ボタン \\ 
\hline
Button & ボタン5 & 表示部に5 を追加 & 有効ボタン \\ 
\hline
Button & ボタン6 & 表示部に6 を追加 & 有効ボタン \\ 
\hline
Button & 確定 & -- & 有効ボタン \\ 
\hline
Event & 画面A & 80 秒 & タイムアウトイベント \\ 
\hline
Event & 表示部に1 を追加 & 表示部に1 を追加 & 単一イベント \\ 
\hline
Event & 表示部に2 を追加 & 表示部に2 を追加 & 単一イベント \\ 
\hline
Event & 表示部に3 を追加 & 表示部に3 を追加 & 単一イベント \\ 
\hline
Event & 表示部に4 を追加 & 表示部に4 を追加 & 単一イベント \\ 
\hline
Event & 表示部に5 を追加 & 表示部に5 を追加 & 単一イベント \\ 
\hline
Event & 表示部に6 を追加 & 表示部に6 を追加 & 単一イベント \\ 
\hline
Event & 確定押下 & -- & 条件分岐を含むイベント \\ 
\hline
\end{tabular}
\end{table}

\begin{table}[tp]
\centering
\caption{コード3.1のMarkdown仕様記述例から生成するEventBranch}
\label{tab:event_branch_example}
\begin{tabular}{|l|l|}
\hline
\textbf{Condition} & \textbf{Target} \\ 
\hline
表示部に1 が入力されている & 画面K \\ 
\hline
表示部に1 が入力されていない & 画面F \\ 
\hline
\end{tabular}
\end{table}


\begin{table}[tp]
\centering
\caption{解析部で用いる正規表現パターン}
\label{tab:parsing_rules}
\begin{tabular}{|l|p{7cm}|p{6cm}|}
\hline
\textbf{規則名} & \textbf{定義(概要)} & \textbf{用途} \\
\hline
BulletPattern
& \texttt{\detokenize{@^\s*(?:-|\*|・|・)\s+(?<Text>.+?)\s*$}}
& ボタン名、画面要素名、イベント行、タイムアウト行の
箇条書き本文の抽出判定 \\
\hline
EventPattern
& \texttt{\detokenize{@^(?<Name>.*?)\s*→\s*(?<Target>.*)$}}
& タイムアウト行および条件分岐行の
「→」を基準とした左右要素分解 \\
\hline
OperationPattern
& \texttt{\detokenize{@^(?<Operation>.*?)(?<Trigger>押下)\s*→\s*(?<Target>.*)$}}
& イベント一覧における
トリガー、イベントの抽出 \\
\hline
\end{tabular}
\end{table}



\begin{figure}[tp]
  \centering
  \includegraphics[width=0.7\linewidth]{./images/kaiseki_flow.png}
  \caption{解析処理フロー}
  \label{fig:parsing-flow}

\end{figure}

\begin{enumerate}
      \item 各行の分割 

      プロジェクト管理部から入力として受け取ったMarkdown文字列を改行で分割し、
      各行を1つの要素とした配列を生成する。
      具体的には、改行で分割した各行から \verb|\r| を除去し、
      \verb|List<string> lines| として保持する。
      以降の処理は、この \verb|lines| を走査対象として用いる。


      \item 先頭行判定
      
      先頭行(\verb|lines[0]|)を判定材料として、
      自然言語仕様記述ルール(\ref{sec:Specrule}節を参照)に則り、入力のMarkdown文字列が、「画面管理クラス」の仕様か、「画面クラス」の仕様かを決定する。
      
      具体的には、
      \verb|lines[0] == "# 画面一覧"| を満たす場合は、
      画面管理クラスとして扱い、A.「画面管理」の場合を行う。
      一方で、 \verb|lines[0]| が(\verb|"## "|) で始まる場合、
      画面クラスとして扱い、B.「画面クラス」の場合を行う。
      
      具体例では、先頭行 \verb|lines[0]| が
      「\verb|## 画面1|」 であるため、例で示すMarkdown仕様は画面クラスとして扱う。

      \item GUI要素の抽出

      先頭行の判定結果より以下の2つの場合分けを行いGUI要素の抽出を行う
      \begin{enumerate}[label=\Alph*.]
            \item 「画面一覧」の場合
            \begin{enumerate}[label=\roman*.]
                  \item 画面抽出
                  
                  画面管理クラスを解析する場合、
                  2行目(\verb|lines[1]|)以降を走査し、\verb|BulletPattern|によって抽出した箇条書き本文を画面要素名として抽出する。

                  抽出した画面につき、 \verb|GUIElement.Type| に \verb|Screen|、
                  \verb|GUIElement.Name| に画面要素名を設定したGUIElementをそれぞれ生成し、\verb|elements| に追加する。
                  拡張後の\tool では、各CTM要素の種別を\verb|enum|クラスとし、判断する、各CTM要素を表す\verb|enum|を以下に示す。
                  \begin{itemize}
                        \item \verb|Screen|:画面要素
                        \item \verb|Button|:ボタン要素
                        \item \verb|Event|:イベント要素
                        \item \verb|Timeout|:タイムアウト要素
                  \end{itemize}
            \end{enumerate}

            \item 「画面クラス」の場合
            \begin{enumerate}[label=\roman*.]
                  \item 画面クラス名抽出

                  画面クラス名を解析する場合、
                  先頭行(\verb|lines[0]|)の文字列を画面クラス名として抽出する。
                  ここでは、\verb|GUIElement.Type| に \verb|Screen|、
                  \verb|GUIElement.Name| に画面クラス名を設定したGUIElementを生成し、\verb|elements| に追加する。

                  具体例では、先頭行の「\verb|## 画面1|」から画面クラス名「画面1」を抽出し、
                  \verb|GUIElement.Type|に\verb|Screen|、\verb|GUIElement.Name|に\verb|画面1|を設定したGUIElementを生成し、\verb|elements| に追加する。

                  \item タイムアウト抽出

                  自然言語仕様記述ルール(\ref{sec:Specrule}節を参照)に基づき、タイムアウト時間は2行目に記述するため、
                  2行目(\verb|lines[1]|)をタイムアウト定義行として扱い、
                  \verb|BulletPattern|によって抽出した箇条書き本文からタイムアウトに関する情報を抽出する。
                  2行目が箇条書き行でない場合、タイムアウト定義が存在しないものとしてスキップする。

                  ここでは、箇条書き本文中から文字「\verb|で|」 を探索し、
                  「\verb|で|」の前の文字列をタイムアウト名として抽出する。
                        抽出したタイムアウト名をもとに、
                  \verb|GUIElement.Type|に \verb|Timeout|、\verb|GUIElement.Name|にタイムアウト名を設定したGUIElementを生成し、\verb|elements| に追加する。
                  
                  具体例では、2行目の「\verb|- 80 秒でタイムアウト|」をタイムアウト定義行として解析する。
                  箇条書き本文から「\verb|で|」より前の文字列(\verb|80 秒|)を抽出する。
                  抽出したタイムアウト名をもとに、
                  CTM要素の種別として\verb|GUIElement.Type|に\verb|Timeout|、CTM要素の名称として\verb|GUIElement.Name| に 80 秒を設定したGUIElementを生成し、 \verb|elements| に追加する。

                  \item ボタン抽出

                  \begin{enumerate}[label=\alph*.]
                        \item 有効ボタン記述フィールドを表す見出し行を探索

                        B.2.の次行以降を走査し、行頭が(\verb|#|)から始まるかどうかを確認する。
                        (\verb|#|)を見つけた場合、該当行の(\verb|#|)を取り除いた文字列を見出し名として抽出する。
                        抽出した見出し名が「有効ボタン一覧」と一致した場合、該当行の次行を有効ボタン記述フィールドの開始行として扱う。

                        具体例では、タイムアウトの次行である3行目\verb|lines[2]|から走査を開始し、4行目\verb|lines[3]|の見出し行「\verb|### 有効ボタン一覧|」を発見する。

                        \item 有効ボタン記述フィールドの範囲を決定

                        見出し行の次行以降を走査し、有効ボタン記述フィールドの終了行を探索する。
                        終了行の判定は、以下のいずれかを満たす行に到達した場合とする。
                        \begin{itemize}
                              \item 見出し行(「\verb|###|」で始まる行)に到達する。
                              \item ファイル末尾に到達する。
                        \end{itemize}

                        具体例では、有効ボタン記述フィールドの終了行は13行目\verb|lines[12]|の見出し行「\verb|### イベント一覧|」であるため、
                        12行目\verb|lines[11]|までを有効ボタン記述フィールドとする。

                        \item ボタン要素を抽出

                        有効ボタン記述フィールド内の各行について、正規表現パターンの1つである\verb|BulletPattern| により箇条書きかどうかを判定し、
                        \verb|BulletPattern|によって抽出した箇条書き本文をボタン名として抽出する。
                        箇条書きでない行はボタン要素として扱わず、スキップする。
                        
                        具体例では、5行目\verb|lines[4]|から12行目\verb|lines[11]|までの各行を走査し、
                        抽出した箇条書き本文をそれぞれボタン名として抽出する。5行目の「\verb|ボタン1|」から、
                        11行目の「\verb|確定|」までをそれぞれ抽出し、12行目は空行であるためスキップする。

                        \item ボタン要素をGUIElementとして生成

                        抽出したボタン名について、\verb|HashSet<string>|(\ref{sec:NET}節を参照)を用いて重複判定を行う。
                        重複していない場合、\verb|GUIElement.Type|に \verb|Button|、
                        \verb|GUIElement.Name|にボタン名を設定したGUIElementを生成し、\verb|elements| に追加する。
                        重複していた場合、該当行のボタン要素の生成をスキップする。
                        
                        具体例では、抽出した各ボタン名について重複判定を行い、
                        全て重複していないため、各ボタン名をもとに
                        \verb|GUIElement.Type|に\verb|Button|、\verb|GUIElement.Name|に\verb|ボタン1|から\verb|確定|をそれぞれ設定したGUIElementを生成し、\verb|elements| に追加する
                  \end{enumerate}

                  \item イベント抽出
                  \begin{enumerate}[label=\alph*.]
                        \item イベント記述フィールドを表す見出し行を探索

                        B.iii.同様に見出し行を探索し、
                        抽出した見出し名が「イベント一覧」と一致した場合、
                        該当行の次行をイベント記述フィールドの開始行として扱う。

                        具体例では、有効ボタン記述フィールドの次行である13行目\verb|lines[12]|から走査を開始し、
                        13行目\verb|lines[12]|の見出し行「\verb|### イベント一覧|」を発見する。

                        \item イベント記述フィールドの範囲を決定

                        見出し行の次行以降を走査し、有効ボタン記述フィールドの終了行を探索する。
                        終了行の判定は、以下のいずれかを満たす行に到達した場合とする。
                        \begin{itemize}
                              \item 見出し行(「\verb|###|」で始まる行)に到達する。
                              \item ファイル末尾に到達する。
                        \end{itemize}。

                        具体例では、イベント記述フィールドの終了行はファイル末尾であるため、最後の行である23行目\verb|lines[22]|までをイベント記述フィールドとする。

                        \item イベント行文字列を抽出

                        イベント記述フィールド内の各行について、
                        \verb|BulletPattern| により箇条書き行かどうかを判定し、
                        箇条書き本文をイベント文字列として抽出する。
                        箇条書き行でない場合はイベント定義ではないためスキップする。

                        具体例では、14行目\verb|lines[13]|から24行目\verb|lines[23]|までの各行を走査し、
                        箇条書き本文をそれぞれイベント文字列として抽出する。
                        14行目の「\verb|タイムアウト → 画面A へ|」から、
                        23行目の「\verb|表示部に1 が入力されていない → 画面F へ|」までをそれぞれ抽出する。


                        \item イベント行の書式を判定

                        抽出したイベント文字列に対し、
                        まず、正規表現パターンの1つである\verb|OperationPattern| を適用し、
                        「\texttt{\detokenize{@^(?<Operation>.*?)(?<Trigger>押下)\s*→\s*(?<Target>.*)$}}」に該当するかを判定する。
                        該当する場合は、
                        (\verb|押下 →|)の直前までの文字列を対象ボタン名\verb|buttonkey|、
                        (\verb|→|)以降の文字列をイベント内容\verb|eventContent|としてそれぞれ抽出する。

                        ここで、\verb|OperationPattern|に該当するが、
                        抽出したイベント内容\verb|eventContent|が空である場合は、
                        当該イベントは条件分岐イベントとして扱う。
                        条件分岐イベントを検出した際は、直後の行から順に走査し、条件分岐行を抽出し、\verb|elements|へ条件分岐イベントとして追加する。
                        抽出した条件分岐行に対する詳細な処理規則および生成するデータ構造については、
                        \ref{sec:ParsingBranch}節で詳しく説明する。

                        また、\verb|OperationPattern|に該当しない場合は、正規表現パターンの1つである\verb|EventPattern| を適用し、
                        「左辺 \verb|→| 右辺」の形式として扱い、
                        (\verb|→|)を区切り文字として、左側文字列\verb|leftText|と、右側文字列\verb|rightText|を抽出する。

                        具体例では、14行目の「\verb|タイムアウト → 画面A へ|」が\verb|OperationPattern|に該当せず
                        「\verb|EventPattern|」に該当するため、
                        左側文字列\verb|leftText|として「\verb|タイムアウト|」を、
                        右側文字列\verb|rightText|として「\verb|画面A へ|」を抽出する。
                        15行目の「\verb|ボタン1 押下 → 表示部に1 を追加|」が
                        \verb|OperationPattern|に該当するため、
                        対象ボタン名\verb|buttonkey|として「\verb|ボタン1|」を、
                        イベント内容\verb|eventContent|として「\verb|表示部に1 を追加|」を抽出する。

                        \item イベント行とボタン要素の対応付け

                        \verb|OperationPattern|に該当するイベントについて、
                        抽出した対象ボタン名\verb|buttonkey|をキーとして、
                        \verb|elements|から既に生成済みのボタン要素を探索する。
                        探索時には、
                        \verb|elements|中のボタン要素(\verb|GUIElement.Type == Button|)が成立する要素それぞれの名称(\verb|GUIElement.Name|)と\verb|buttonkey|で比較を行い、
                        対応するボタン要素を特定する。
                        対応するボタン要素が存在する場合、
                        対応するボタン要素のTarget属性\verb|GUIElement.Target|に
                        イベント内容\verb|eventContent|を設定する。
                        対応するボタン要素が存在しない場合、
                        当該イベントは対応ボタンの無いイベントとして扱い、スキップする。
                        
                        具体例では、15行目の「\verb|ボタン1 押下 → 表示部に1 を追加|」に対し、
                        抽出した対象ボタン名\verb|buttonkey|である「\verb|ボタン1|」をキーとして
                        \verb|elements|から対応するボタン要素を探索し、特定する。
                        対応するボタン要素「\verb|ボタン1|」はelementsに存在しするため、
                        対応するボタン要素のTarget属性\verb|GUIElement.Target|にイベント内容\verb|eventContent|である「\verb|表示部に1 を追加|」を設定する。
                        21行目の「\verb|確定 押下 → |」に対しては、
                        抽出した対象ボタン名\verb|buttonkey|である「\verb|確定|」をキーとして
                        \verb|elements|から対応するボタン要素を探索し、特定する。
                        対応するボタン要素「\verb|確定|」はelementsに存在するが、
                        イベント内容\verb|eventContent|が空であるため、
                        当該イベントは条件分岐イベントとして扱い、ボタン要素との対応付けをスキップする。

                        \item 単一イベント要素を生成

                        条件分岐イベントに該当しない場合、
                        解析部は単一イベントとして要素生成を行う。
                        \verb|OperationPattern|形式のイベントでは、
                        \verb|GUIElement.Type == Event| の\verb|GUIElement|を生成する。
                        このとき、
                        \verb|eventContent|を \verb|GUIElement.Name| 、および、
                        \verb|GUIElement.Target|それぞれに設定し、
                        \verb|elements| に追加する。
                        
                        一方で、
                        \verb|EventPattern|形式のイベントでは、
                        \verb|GUIElement.Type| に\verb|Event|、
                        \verb|GUIElement.Name| に\verb|leftText|、
                        \verb|GUIElement.Target|に\verb|rightText|、
                        を設定した\verb|GUIElement|を生成する。
                        また、左辺文字列(\verb|leftText|)が「\verb|leftText == タイムアウト|」である場合、
                        既に生成済みのタイムアウト要素(\verb|GUIElement.Type == Timeout|)を
                        \verb|elements| から探索する。
                        タイムアウト要素が存在する場合、
                        \verb|GUIElement.Type|に \verb|Event| 、\verb|GUIElement.Name|に\verb|rightText|、
                        \verb|GUIElement.Target| に該当するタイムアウト要素の\verb|GUIElement.Name|を設定した\verb|GUIElement|を生成する。

                        具体例では、14行目の「\verb|タイムアウト → 画面A へ|」に対し、
                        抽出した左側文字列(\verb|leftText|)が「\verb|タイムアウト|」であるため、
                        既に生成済みのタイムアウト要素を探索する。
                        \verb|elements|中にタイムアウト要素\verb|80 秒|が存在するため、
                        \verb|GUIElement.Type| に\verb|Event|、\verb|GUIElement.Name|に\verb|画面A へ|、
                        \verb|GUIElement.Target|に\verb|80 秒|を設定したGUIElementを生成し、 \verb|elements| に追加する。
                        
                        15行目の「\verb|ボタン1 押下 → 表示部に1 を追加|」に対し、
                        \verb|GUIElement.Type| に\verb|Event|、\verb|GUIElement.Name|に\verb|表示部に1 を追加|、
                        \verb|GUIElement.Target|に\verb|表示部に1 を追加|を設定したGUIElementを生成し、 \verb|elements| に追加する。


                  \end{enumerate}

            \end{enumerate}
      \end{enumerate}
      \item elements出力

      全要素の抽出完了後、
      解析部は\verb|elements|を出力する.
\end{enumerate}

\subsection{条件分岐の解析処理}\label{sec:ParsingBranch}

本節では、イベント一覧セクション内に記述している条件分岐イベントについて、
分岐条件および分岐先を抽出し、
GUIElement の \verb|Branches|(表\ref{tb:EventBranch})へ格納する処理を説明する。
条件分岐イベントは、単一のトリガ(ボタン押下等)に対して複数の遷移候補を列挙する記述であり、
通常の単一イベントとは異なる解析規則を適用する必要があるため、
独立した処理として扱う。

\subsubsection{対象とするMarkdown記述形式}

解析部は、条件分岐イベントを以下の形式で判定する。
親イベント行とは、条件分岐イベントのトリガを表す行であり、
分岐行とは、親イベント行に対応する各分岐条件および分岐先を表す行である。

\begin{itemize}
      \item 親イベント行:\verb|- {Operation}{押下} → 空|(例:\verb|- 確定 押下 → 空|)
      \item 分岐行:\verb|- {Condition} → {Target}|(例:\verb|  - 表示部に1が入力されている → 画面Kへ|)
\end{itemize}

親イベント行は \verb|OperationPattern|
により検出し、、
分岐行は \verb|EventPattern|
により \verb|→| を基準に左右へ分解する。

\subsubsection{条件分岐イベント解析処理の処理の流れ}\label{sec:ParsingBranchFlow}

本処理は、イベント一覧セクションの走査中に検出した
「親イベント行(\verb|OperationPattern|に一致し、かつ\verb|→|右辺が空)」を起点として、
後続の分岐行を収集し、\verb|Branches|を保持する親イベント要素を生成する処理である。
本処理の流れを以下に示す。

\begin{enumerate}
      \item 親イベント行候補の判定

      イベント一覧セクション内で抽出したイベント文字列に対し、
      \verb|OperationPattern| を適用し、
      操作名 \verb|buttonkey| と\verb|→|以降文字列 \verb|eventContent| を抽出する。
      ここで \verb|eventContent| が空である場合、
      当該行を条件分岐イベントの親イベント行として扱い、
      分岐収集処理へ遷移する。
      \verb|eventContent| が空でない場合は条件分岐として扱わず、
      本処理は実行しない。

      具体例では、親イベント行「\verb|確定押下 → |」は \verb|OperationPattern| に一致し、
      \verb|buttonkey|として\verb|確定|、\verb|eventContent|として\verb|""| を得るため、
      条件分岐イベントとして扱う。

      \item 親イベントに対応するボタン要素の特定

      親イベント行から抽出した\verb|buttonkey| をキーとして、
      既に生成済みのボタン要素(\verb|GUIElementType.Button|)を \verb|elements| から探索する。
      探索時には、前後空白の除去および大小文字を無視した比較を行う。
      対応するボタン要素が存在する場合、
      以降の親イベント要素生成時に参照するため、
      \verb|correspondingButton| として保持する。

      具体例では、\verb|buttonkey == 確定| をキーとして
      有効ボタン一覧で生成済みのボタン要素「\verb|確定|」を探索し、
      \verb|correspondingButton|に\verb|確定|を保持する。

      \item 分岐収集の初期化

      親イベント行の直後の行番号を開始位置として設定する。
      実装では、親イベント行の行番号を \verb|k| としたとき、
      \verb|startK = k + 1| とする。
      併せて、分岐情報を保持する \verb|List<EventBranch> branches| を初期化する。

      具体例では、「\verb|確定 押下 → |」の直後行から分岐行収集を開始し、
      \verb|branches| を空のリストとして初期化する。

      \item 分岐行の収集範囲の走査

      \verb|startK| から順に行を走査し、
      各行が分岐行として扱えるかどうかを判定しながら収集する。
      分岐行として扱う候補行は、箇条書き本文を抽出できる行である。
      具体的には、\verb|BulletPattern|により
      箇条書き本文 \verb|nestedContent| を抽出できた場合に限り、
      当該行を分岐候補として扱う。

      具体例では、親イベント行の直後の
      「\verb|  - 表示部に1が入力されている → 画面Kへ|」および
      「\verb|  - 表示部に1が入力されていない → 画面Fへ|」
      の2行が箇条書きとして抽出可能であるため、分岐候補として扱う。

      \item 終了条件の判定

      分岐行の誤取り込みを防ぐため、
      走査中に以下のいずれかを満たした場合、
      分岐行の収集を終了する。

      \begin{itemize}
            \item 空行に到達した場合。
            \item 見出し行(\verb|###|で始まる行)に到達した場合。
            \item 箇条書き本文抽出に失敗した場合。
            \item 箇条書き本文 \verb|nestedContent| が
                  \verb|OperationPattern| に一致する場合、
                  または \verb|nestedContent| が「タイムアウト」で始まる場合。
      \end{itemize}

      具体例では、2つの分岐行の次に空行、および、見出し行が現れた時点で
      分岐行収集を終了する。

      \item 分岐行の本文抽出と左右要素の分解

      分岐候補行について、箇条書き本文 \verb|nestedContent| を取得した後、
      \verb|EventPattern| を適用し、
      \verb|→| を境に左側文字列 \verb|leftText| と右側文字列 \verb|rightText| を抽出する。
      \verb|EventPattern| に一致した場合は、
      \verb|leftText| を分岐条件(\verb|Condition|)、
      \verb|rightText| を分岐先(\verb|Target|)として扱う。
      一致しない場合は、
      当該本文を条件のみの記述として扱い、
      \verb|Condition|に\verb|nestedContent|を設定し、
      \verb|Target| は空として扱う。

      具体例では、
      「\verb|表示部に1が入力されている → 画面Kへ|」を
      \verb|Condition == 表示部に1が入力されている|、
      \verb|Target == 画面Kへ| に分解する。

      \item EventBranchの生成とbranchesへの追加

      6.で得た(\verb|Condition|, \verb|Target|)をもとに
      \verb|EventBranch| を生成し、\verb|branches| に追加する。
      追加は行の出現順に行い、
      仕様記述における分岐の順序を保持する。

      具体例では、2つの分岐について
      \verb|(表示部に1が入力されている、画面Kへ)|、
      \verb|(表示部に1が入力されていない、画面Fへ)|
      の順に \verb|branches| へ追加する。

      \item 外側走査位置の更新

      分岐行の収集は複数行にわたるため、
      収集終了時点の走査位置 \verb|startK| を用いて
      外側のイベント走査インデックス \verb|k| を更新する。
      これにより、収集済みの分岐行を
      再度イベント定義として処理することを防止する。

      \item 親イベント要素の生成

      \verb|branches| を1件以上収集できた場合、
      親イベント要素(\verb|GUIElement.Type == Event|)を生成し、
      \verb|GUIElement.Branches |に\verb| branches| を設定して \verb|elements| に追加する。
      親イベント要素の \verb|GUIElement.Name| は操作トリガを表すために
      \verb|"{correspondingButton}押下"| の形式とする。
      さらに、
      そのボタン要素\verb|GUIElement.Type == Button|のTarget属性に「\verb|"{correspondingButton}押下"|」を設定する。

      具体例では、
      親イベント要素を
      \verb|GUIElement.Type |に\verb| Event|、
      \verb|GUIElement.Name |に\verb| 確定押下|、
      \verb|GUIElement.Target |に\verb| 空|として
      \verb|GUIElement|に設定する。

      また、対応するボタン要素「\verb|確定|」のTarget属性\verb|GUIElement.Target|に
      「\verb|確定押下|」を設定する。

      \item 不完全な分岐記述の扱い

      分岐行を1件も収集できなかった場合は、
      条件分岐として成立しない記述であるため、
      親イベント要素を生成せず、
      当該親イベント行はスキップする。

      具体例では、分岐行が1件も無い場合には
      「確定押下」の親イベント要素を生成しない。
\end{enumerate}

\section{GUI要素生成部}\label{sec:GUIElementGenerationComponent}

GUI要素生成部は、解析部(\ref{sec:ParsingComponent}節を参照)で抽出した
\verb|elements| をプロジェクト管理部(\ref{sec:ProjectManagementComponent}節を参照)から入力として受け取る。
同時に、対象とするMarkdownファイルに対応する
GUI配置情報(JSON)をプロジェクト管理部から受け取る。
GUI要素生成部は、この\verb|elements| 、および、GUI配置情報をもとに、
画面上で操作可能なGUI要素のリストを生成する処理部である。

本論文では解析部およびGUI要素生成部の双方で同一の \verb|GUIElement| クラスを用いる。
ただし、解析部が出力する \verb|GUIElement| は以下のデータを保持する。
\begin{itemize}
      \item \verb|GUIElement.Type|:CTM要素の種別
      \item \verb|GUIElement.Name|:CTM要素の名称
      \item \verb|GUIElement.Target|:CTM要素のターゲット情報
      \item \verb|GUIElement.Branches|:条件分岐イベントにおける分岐情報リスト
\end{itemize}

一方、GUI要素生成部が出力する\verb|GUIElement|は、以下のデータを追加で保持する。
\begin{itemize}
      \item \verb|X|:CTM要素のX座標
      \item \verb|Y|:CTM要素のY座標
      \item \verb|Width|:CTM要素の幅
      \item \verb|Height|:CTM要素の高さ
      \item \verb|IsMovable|:CTM要素の移動可能状態
      \item \verb|IsSelected|:CTM要素の選択状態
\end{itemize}

本論文では、GUI要素生成部が出力する\verb|GUIElement|をGUI要素と呼ぶ。
また、GUI要素のリストをGUI要素リストと呼ぶ。

GUI要素生成部の入力は、解析部の出力である \verb|elements| と、
必要に応じてプロジェクト管理部から受け取る配置情報(JSON)である。
出力は、ユーザ操作監視部(\ref{sec:UserOperationMonitoring}節を参照)に渡すGUI要素リスト\verb|elements|である。

GUI要素生成部で用いる通常ノードの描画仕様を表\ref{tab:gui_node_spec}に、
条件分岐イベントにおける可視ノードの描画仕様を表\ref{tab:gui_node_branch}に示す。

GUI要素の座標は、GUI配置情報(JSON)が存在する場合は
\verb|GUIElement.Name| をキーとして復元し、
配置情報が存在しない場合、または該当要素の座標が存在しない場合は、
初期配置規則に基づいて決定する。
初期配置座標は、同一種別内のインデックス $i$ に基づき
式\ref{eq:initial_position_x}および式\ref{eq:initial_position}で決定する。
ここで $i$ は、同一種別の要素に対して解析順に付与する 0 始まりの連番である。
また $d$ は要素間の縦方向間隔を表す定数である。
$X_0$、$Y_0$ は初期配置の基準座標を表す定数であり、
CTM領域の左上を原点とする座標系において画面端との余白を確保するために定義する。
実際には、$X_0 = 40px$、$Y_0 = 40px$、$d = 80px$ として設定する。
\begin{equation} \label{eq:initial_position_x}
X_i = X_0\quad
\end{equation}
\begin{equation}\label{eq:initial_position}
Y_i = Y_0 + i \cdot d
\end{equation}

本処理の流れを以下に示す。

\begin{enumerate}
      \item 入力の受理と出力リストの初期化

      解析部の出力である \verb|elements| を入力として受け取り、
      GUI要素リスト(出力用のリスト)を初期化する。
      併せて、種別ごとの配置インデックス $i$ を 0 で初期化する。

      \item \verb|elements| の走査と要素種別の判定

      \verb|elements| を先頭から順に走査し、
      各要素について \verb|GUIElement.Type| を参照して種別を判定する。

      \begin{itemize}
      \item 画面要素の生成

      \verb|GUIElement.Type| が \verb|Screen| の場合、
      以下の手順でGUI要素を生成する。
      \begin{enumerate}[label=\roman*.]
            \item 当該種別のインデックス $i$ を取得する。
            \item 表\ref{tab:gui_node_spec}に基づき、画面要素の形状、および、既定サイズを設定する。
            \item 座標を決定する。
                  JSONに一致する座標が存在する場合はそれを設定し、
                  存在しない場合は式\ref{eq:initial_position_x}および式\ref{eq:initial_position}により初期配置座標を算出して設定する。
            \item 画面要素はユーザ操作により移動可能とし、
                  \verb|IsMovable = true| を設定する。
            \item 生成したGUI要素をGUI要素リストへ追加し、
                  インデックス $i$ を更新する。
      \end{enumerate}

      \item ボタン要素の生成

      \verb|GUIElement.Type| が \verb|Button| の場合、
      以下の手順でGUI要素を生成する。
      \begin{enumerate}[label=\roman*.]
            \item 当該種別のインデックス $i$ を取得する。
            \item 表\ref{tab:gui_node_spec}に基づき、ボタン要素の形状・既定サイズを設定する。
            \item 座標を決定する。
            JSONに一致する座標が存在する場合はそれを設定し、
            存在しない場合は初期配置規則に基づき縦方向に重ならない位置へ配置する。
            \item ボタン要素はユーザ操作により移動可能とし、
            \verb|IsMovable = true| を設定する。
            \item 生成したGUI要素をGUI要素リストへ追加し、
            インデックス $i$ を更新する。
      \end{enumerate}

      \item タイムアウト要素の生成

      \verb|GUIElement.Type| が \verb|Timeout| の場合、
      以下の手順でGUI要素を生成する。
      \begin{enumerate}[label=\roman*.]
            \item 表\ref{tab:gui_node_spec}に基づき、タイムアウト要素の形状・既定サイズを設定する。
            \item 座標を決定する。
            JSONに一致する座標が存在する場合はそれを設定し、
            存在しない場合は画面左上付近の固定位置となるよう既定オフセットを適用して設定する。
            \item タイムアウト要素は画面構造の基準として扱うため移動不可とし、
            \verb|IsMovable = false| を設定する。
            \item 生成したGUI要素をGUI要素リストへ追加する。
      \end{enumerate}

      \item イベント要素の生成

      \verb|GUIElement.Type| が \verb|Event| の場合、
      以下の手順でGUI要素を生成する。
      \begin{enumerate}[label=\roman*.]
            \item 当該イベント要素が条件分岐を含むかどうかを判定する。
            具体的には \verb|Branches| が1つ以上存在する場合を条件分岐イベントとする。
            \item 条件分岐を含まない場合は、
            表\ref{tab:gui_node_spec}に基づき矩形形状のイベント要素を生成する。
            座標は JSON を優先し、
            JSONが存在しない場合は対応するボタン要素またはタイムアウト要素の右側に来るように初期配置する。
            \item 条件分岐を含む場合は、
            親イベント要素(イベント本体ノード)を表示対象外として扱い、
            条件分岐イベントにおける Condition ノードおよび Target ノードの
            具体的な描画位置、形状の頂点位置、
            および分岐線の描画に関する幾何的計算は、
            GUI要素生成部では行わない。
            これらの計算は、描画部(\ref{sec:DrawingComponent}節を参照)において、
            基準となるイベント要素および分岐順序に基づいて動的に決定する。

      \end{enumerate}

      \end{itemize}
      \item 出力

      全要素の生成が完了した後、
      生成したGUI要素リストをユーザ操作監視部へ出力する。
\end{enumerate}



\begin{table}[tp]
\centering
\caption{通常ノードの描画仕様}
\label{tab:gui_node_spec}
\begin{tabular}{|l|l|c|c|p{4.5cm}|}
\hline
\textbf{要素} & \textbf{形状} & \textbf{幅(px)} & \textbf{高(px)} & \textbf{配置・備考} \\
 \hline
Screen
& 角丸矩形
& 160 & 45
& CTM領域の左列(x = 40px)に縦配置(間隔80Px)、移動可 \\
 \hline

Button
& 楕円
& 80 & 45
& CTM領域の左列(x = 40px)に縦配置(間隔80Px)、移動可 \\
 \hline

Event
& 矩形
& 160 & 45
& CTM領域の右列(x = 240px)に縦配置(間隔80Px)、条件分岐時は本体非表示、移動可 \\
 \hline

Timeout
& 楕円
& 112 & 45
& 左上(x = 40px, y = 8px)固定配置、移動不可 \\
 \hline
\end{tabular}
\end{table}



\begin{table}[tp]
\centering
\caption{条件分岐イベントにおける可視ノードの描画仕様}
\label{tab:gui_node_branch}
\begin{tabular}{|l|l|c|c|p{4cm}|}
\hline
\textbf{要素} & \textbf{形状} & \textbf{幅(px)} & \textbf{高(px)} & \textbf{備考} \\
 \hline
Condition
& ダイヤモンド
& 176 & 50
& 分岐条件を表す可視ノード \\
 \hline

Target
& 矩形
& 152 & 36
& 分岐先を示す中間ノード \\
 \hline
\end{tabular}
\end{table}


\section{変換部}\label{sec:ConversionComponent}
変換部は、編集内容をMarkdown仕様、および、VDM++仕様へ変換する処理部である。
本処理部では、Markdown仕様からVDM++仕様の変換に加え、ユーザによるGUI操作結果をMarkdown仕様へ反映する処理を行う。
Markdown仕様からVDM++仕様への変換処理については、既存の\tool のMarkdown変換部(\ref{sec:Structure}節を参照)を利用する。

GUI操作からMarkdown仕様への変換処理は、GUI操作による編集結果を差分として反映するのではなく、
GUI要素リストの現在状態からMarkdown仕様の全体を再生成する方式を採用している。
この設計により、
編集履歴や操作順序に依存しない仕様を生成できる。

以降、本節では、本研究で追加したGUI操作からMarkdown仕様への変換処理、
および、出力順序整理処理について説明する。

\subsection{GUI操作からMarkdown仕様への変換処理}\label{sec:GUItoMarkdownConversionProcess}
本処理は、GUI要素生成部(\ref{sec:GUIElementGenerationComponent}節を参照)で生成したGUI要素リスト\verb|elements|、および、プロジェクト管理部より受け取る編集前のMarkdown文字列を入力として、
編集後Markdown文字列へ変換する処理である。
まず、GUI要素リスト内の各GUI要素を、種類および関連関係に基づいて分類する。
次に、各GUI要素を
Markdown仕様記述として出力する順序を決定する。

出力規則は\tool の自然言語仕様記述ルール(\ref{sec:Specrule}節を参照)に準拠しており、
以下の要素を順に出力する。
\begin{itemize}
      \item 画面
      \item タイムアウト
      \item 有効ボタン一覧
      \item イベント一覧
\end{itemize}

変換処理の結果として生成するMarkdown文字列は、
プロジェクト管理部に引き渡し、
ファイル更新処理(\ref{sec:FileUpdateProcess}節を参照)により保存する。
これにより、
常にCTM上の編集操作と仕様記述との間に
対応関係を保つことが可能である。

本処理の流れを、以下に示す。
\begin{enumerate}
      \item 入力として、GUI要素リスト \verb|elements| および
        プロジェクト管理部から取得したMarkdown文字列「\verb|markdownText|」を受け取る。

      \item 取得したGUI要素リスト、および、Markdown文字列を
        出力順序決定のために出力順序整理処理に入力する(\ref{sec:ConversionOrderProcess}節を参照)。
        
      \item 出力順序整理処理が出力したMarkdown文字列各行を1つの要素とした配列である
      \verb|lines| に格納したMarkdown文字列各行を連結し、
        最終的なMarkdown文字列
        \verb|markdown| を生成する。
      \item 生成した \verb|markdown| を
        プロジェクト管理部に渡し、ファイル更新処理(\ref{sec:FileUpdateProcess}節を参照)により保存する。

\end{enumerate}


\subsection{出力順序整理処理}\label{sec:ConversionOrderProcess}
本処理では、GUI要素リスト \verb|elements| 、およびMarkdown文字列\verb|markdownText|を入力として、
Markdown文字列のうち「有効ボタン一覧」および「イベント一覧」の並び順を
CTM上の配置順序に一致するよう再構成する。
本処理は、Markdown仕様ファイルの内容を変更することなく、
CTM上の配置順序とMarkdown仕様上の記述順序を一致させることのみを目的とする。

なお、本処理では、
GUI 要素リスト\verb|elements|に存在しないイベント記述であっても、
Markdown 仕様中に記述が存在する場合は、
仕様データの欠落を防ぐ目的から自動的に削除しない設計としている。

本処理の流れを、以下に示す。
\begin{enumerate}
      \item GUI操作からMarkdown仕様への変換処理(\ref{sec:GUItoMarkdownConversionProcess}節を参照)から入力として、Markdown文字列「\verb|markdownText|」と
        GUI要素リストを受け取る。
      \item Markdown文字列を読み込み、Markdown文字列各行を1つの要素とした配列(\verb|lines|)として保持する。
      \item 行配列 \verb|lines| 内から、
        ボタン一覧見出し \verb|"### 有効ボタン一覧"| の行位置を 完全一致で探索し、行番号を\verb|buttonIdx| として、
        イベント一覧見出し \verb|"### イベント一覧"| の行位置を 完全一致で探索し、行番号を\verb|eventIdx| として保持する。
      \item ボタン一覧の並び順を表す配列として、
        \verb|buttonList| を生成する。
        \verb|buttonList| は、以下の処理で生成する。
        \begin{enumerate}[label=\roman*.]
            \item GUI要素リストから \verb|GUIElement.Type==Button| のGUI要素を抽出する。
             \item 抽出したGUI要素を、GUI要素が保持する\verb|Y| 座標の昇順(Y座標の値が小さい順)で整列する。
            \item 各GUI要素の \verb|GUIElement.Name| をⅱ.で整列した順に取り出し、\verb|buttonList|に格納する。
        \end{enumerate}
      \item \verb|buttonList| が空でない場合、
        \verb|buttonIdx| の有無に応じて以下の処理を行い、
        Markdown文字列のボタン一覧セクション(見出しおよび箇条書き)を更新する。
        \begin{itemize}
            \item \verb|buttonIdx| が存在する場合:
                既存のボタン一覧セクションを
                \verb|buttonList| に基づく箇条書きへ置換する。
            \item \verb|buttonIdx| が存在しない場合:
                  新規にボタン一覧セクションを生成する。ボタン一覧セクションを生成する位置は、
                  \verb|eventIdx| の有無に応じて以下のように決定する。
                  \begin{itemize}
                        \item \verb|eventIdx| が存在する場合:イベント一覧見出しの直前へ挿入
                        \item \verb|eventIdx| が存在しない場合:行配列末尾へ追加
                  \end{itemize}
        \end{itemize}
        ただし、\verb|buttonList| が空の場合は、空見出し生成を避けるため、
        ボタン一覧セクションの置換、および、新規作成を行わない。
      \item イベント一覧の並び順を表す配列として、
        \verb|blocksOrder| を生成する。
        \verb|blocksOrder| は、以下の処理で生成する。
        \begin{enumerate}[label=\roman*.]
            \item GUI要素リストから \verb|GUIElement.Type==Event| かつ \verb|GUIElement.Name| が空でないGUI要素を抽出する。
            \item 抽出したGUI要素を、GUI要素が保持する\verb|Y| 座標の昇順で整列する。
            \item 各GUI要素の \verb|GUIElement.Name| をⅱ.で整列した順に取り出し、\verb|blocksOrder| に格納する。
            \end{enumerate}
      \item Markdown文字列上のイベント一覧セクションを「ブロック単位」で再構成する。
        本処理では、イベント一覧セクション内の記述を、以下の処理でブロック化する。
        \begin{enumerate}[label=\roman*.]
            \item \verb|"- "| で始まる行をブロック先頭とする。
             \item 直後に続く行のうち、
                先頭が2スペース以上(\verb|"  "|)またはタブ(\verb|"\t"|)で始まる行を同一ブロックに含める。
        \end{enumerate}
        これにより、条件分岐を含むイベント(ネストされた箇条書き)も
        1つのブロックとして保持され、並べ替え時に分割しない。
      \item ブロックの並べ替えは、\verb|blocksOrder| の順序に従って行う。
        編集前のMarkdown文字列を走査し、\verb|blocksOrder| と部分一致しなかったブロックは、
        仕様欠落を防ぐため元の順序を保ったまま末尾へ追加する。
      \item 整理後の \verb|lines| に対し、空行を正規化する処理を適用し、
        連続空行や不要な空行を除去する。空行を正規化する処理を以下に示す。
        \begin{enumerate}[label=\roman*.]
            \item 行配列を先頭から順に走査し、
                  各行が空行(空文字列、または空白文字のみからなる行)であるかを判定する。

            \item 直前の行も空行である場合、
                  該当行は連続空行とみなし、
                  該当行を排除する。

            \item 行配列の先頭が空行である場合、
                  見出し行や本文の直前に不要な空行が存在する状態となるため、
                  先頭の空行を除去する。

            \item 行配列の末尾が空行である場合、
                  文書末尾に不要な空行が存在する状態となるため、
                  末尾の空行を除去する。

            \item 上記処理の結果、
                  空行は最大で1行のみが連続して出現する状態となり、
                  Markdown仕様として可読性の高い行配列を得る。
      \end{enumerate}
  \item 正規化後の \verb|lines|を出力する。
\end{enumerate}


\section{描画部}\label{sec:DrawingComponent}
描画部は、ユーザ操作監視部(\ref{sec:UserOperationMonitoring}節を参照)
から受け取ったGUI要素リスト\verb|elements|をもとに、ツール上部に設置するボタンとCTM領域を描画、および、再描画する処理部である。
本処理部の目的は、ユーザが仕様の構造を直感的に把握できる視覚的表現を提供することである。
本描画部は、.NET MAUIの標準ライブラリであるMicrosoft.Maui.Graphics(\ref{sec:NET_MAUI}節を参照)を用いて実装している。

描画部は、
各CTM要素を外接矩形として扱い、
要素種別に応じた形状、寸法、および、描画スタイルを適用する。
本研究における外接矩形とは、CTM要素を囲む最小の矩形領域を指し、
CTM要素の位置およびサイズを定義するために用いる。
各要素種別に応じた寸法および形状は、
GUI要素生成部で定義した規則に基づいて決定しており、
通常ノード、および、条件分岐ノードについては、
それぞれ表\ref{tab:gui_node_spec}、
表\ref{tab:gui_node_branch}に示す描画仕様を用いる。

\subsection{初期描画処理}\label{sec:first-draw}
初期描画処理は、ユーザ操作監視部から受けとった初期描画要求、および、GUI要素リスト\verb|elements|をもとに、
初期状態のCTMを生成する処理である。

本処理の流れを、以下に示す。
\begin{enumerate}
    \item CTM領域を初期化し、背景を消去する。
    \item GUI要素リスト\verb|elements|を走査し、各GUI要素の種別\verb|GUIElement.Type|を判定する。
    \item 要GUI素種別に応じて、矩形、楕円、菱形の描画形状および描画スタイルを決定する。
    \item 各GUI要素が保持する座標情報をもとに、CTM領域上へCTM要素、および、ラベル文字列を描画する。
    \item 遷移関係を表す要素については、GUI要素生成部で設定した接続情報を参照し、
    接続元および接続先ノードの外接矩形から矢印の開始点および終了点を算出した上で、
    遷移を示す矢印を描画する。
    \item 条件分岐を持つ要素\verb|GUIElement.IsBranch == true|については、通常のイベントノード本体を描画せず、条件分岐構造を展開して描画する。具体的には、以下の手順で描画を行う。
    \begin{enumerate}[label=\roman*.]
      \item 条件表示領域を「条件文を表示する矩形」として描画し、
      その右側に分岐点を示す菱形を描画する。

      \item \verb|Branches|の要素数 $n$ を取得し、縦方向に $n$ 個並べて配置する。
      このとき、\verb|Branches|は同一幅、同一高さの外接矩形として扱い、
      条件表示領域のY座標を基準とし、一定の垂直間隔でオフセットを加えることで配置する。
      具体的には、基準Y座標を $Y_{\mathrm{base}}$、
      \verb|Branches|の垂直間隔を $\Delta Y$(分岐先領域の高さを含む定数)とすると、
      $i$ 番目の\verb|Branches|のY座標 $Y_i$ は式\ref{eq:branch_y}により算出する。

      \begin{equation}\label{eq:branch_y}
            Y_i = Y_{\mathrm{base}} + i \cdot \Delta Y
      \end{equation}

      \item 分岐条件から分岐先へは、
      外接矩形に基づき矢印の開始点および終了点を算出して接続線を描画する。
      矢印の開始点は分岐点(菱形)の右端中央、終了点は分岐先領域の左端中央とし、
      矢印を付与して分岐方向を示す。
      分岐先が画面遷移を表す場合は、分岐先名称が画面要素として解釈できるかを判定し、
      解釈できない場合は矢印の描画スタイルを変更して参照不整合を視覚的に示す。

      \item 条件分岐は編集対象の単位となるため、
      描画と同時にヒットテスト用の領域情報を生成して保持する。
      保持する情報は、条件表示領域の外接矩形、分岐点(菱形)の外接矩形、
      および各分岐先領域の外接矩形であり、
      これらを親要素および分岐番号 $i$ に対応付けて記録する。
\end{enumerate}


    \item 描画したすべてのノードおよび分岐要素について、
    外接矩形情報を記録し、ヒットテスト処理
    (\ref{sec:UOM_HitTest}節を参照)で利用可能な状態として保持する。

    \item VDM\texttt{++}描画要求を出力する。
\end{enumerate}

\subsubsection{矢印の開始点および終了点の算出}
矢印の開始点および終了点は、接続元および接続先ノードの外接矩形情報をもとに算出する。
具体的には、接続元ノードの右端中央を開始点とし、接続先ノードの左端中央を終了点とする。
ここで、接続元ノードの左上座標を$(X_s, Y_s)$、幅を$W_s$、高さを$H_s$、
接続先ノードの左上座標を$(X_t, Y_t)$、幅を$W_t$、高さを$H_t$とすると、
開始点$(x_s, y_s)$および終了点$(x_t, y_t)$は、式\ref{eq:arrow_points_start}、\ref{eq:arrow_points_end}により算出する。

\begin{equation}\label{eq:arrow_points_start}
      (x_s, y_s) = (X_s + W_s, Y_s + \frac{H_s}{2})
\end{equation}

\begin{equation}\label{eq:arrow_points_end}
      (x_t, y_t) = (X_t, Y_t + \frac{H_t}{2})
\end{equation}

\subsection{再描画処理}\label{sec:Re-Draw}
再描画処理は、ユーザ操作や仕様変更に伴い、CTM表示を更新する処理である。
ユーザ操作監視部からの再描画要求を受け取り、再描画したツール上部のボタン、およびCTM領域を更新する。
この処理により、仕様の変更が即座に視覚的に反映され、ユーザが最新の状態を把握できるようにする。

本処理の流れを、以下に示す。

\begin{enumerate}
    \item ユーザ操作監視部からの再描画要求を受信する。
    \item 更新対象となるGUI要素について、座標情報および選択状態などの内部状態を反映する。
    \item 描画領域を初期化し、既存の描画内容を消去する。
    \item 初期描画処理と同一の手順により、全GUI要素および遷移関係を再描画する。
    \item ユーザ操作監視部により選択中と判定した要素については、枠線の色および太さを変更し、視覚的に選択状態であることを強調表示する。
    \item VDM\texttt{++}描画要求を出力する。
\end{enumerate}

\subsection{操作ボタン生成および表示切替処理}
\label{sec:DrawingComponent_OperationButtons}

本処理は、ツール上部に表示する操作ボタン群を生成、および、切り替える処理である。
ツール上部の操作ボタンは固定的なUI要素ではなく、
ユーザの操作状況に応じて動的に構成する。本処理により、
ユーザが現在実行可能な操作を直感的に把握できるようにする。

本処理の入力は、
ユーザ操作監視部が管理する現在の表示パターンフラグである。
描画部は、表示パターンフラグに基づいて
対応する操作ボタンを生成、および、描画する。
操作ボタンの表示パターンは\ref{sec:GUI-operation-control-function}で説明する。

本処理の流れを以下に示す。

\begin{enumerate}
  \item ユーザ操作監視部が管理する現在の表示パターンフラグを取得する。
  \item 取得した表示パターンフラグをもとに、表示するボタンを決定する。表示パターンフラグと表示パターンの対応関係を表\ref{tab:button_appear}に示す。
  \item 決定した操作ボタンをツール上に生成する。
  \item 生成した操作ボタン群をツール上部に配置し、描画する。
  \item 選択中GUI要素の変更や表示パターンの遷移が発生した場合には、
        操作集合を再取得し、
        本処理を再実行することで表示内容を更新する。
\end{enumerate}

\begin{table}[tp]
  \centering
  \caption{表示パターンフラグと表示方法の対応関係}
  \label{tab:display_pattern_flags}
  \begin{tabular}{|c|c|c|c|}
    \hline
    \rule{0pt}{4.5ex}\textbf{表示パターン} &
    \textbf{\shortstack{IsScreenList\\AddButtonVisible}} &
    \textbf{\shortstack{IsClassAll\\ButtonVisible}} &
    \textbf{\shortstack{IsClassAdd\\ButtonVisible}} \\
    \hline
    表示パターンA & false & false & false \\
    \hline
    表示パターンB & true & false & false \\
    \hline
    表示パターンC & false & true & false \\
    \hline
    表示パターンD & false & false & true \\
    \hline
  \end{tabular}
\end{table}


\subsection{フォルダツリー描画機能}\label{sec:FolderTree}
フォルダツリー描画機能は、フォルダツリー表示領域に
プロジェクト管理部で構築した
\verb|FolderItems| を
ツリー形式で描画する処理である。
描画部は、\verb|FolderItems| の各要素\verb|FolderItem|が保持する
階層情報\verb|Level|、展開状態\verb|IsExpanded|、可視状態\verb|IsVisible|、
および種別情報\verb|IsFolder|/\verb|IsFile|に基づき、表示内容を構成する。
本処理では、.NET MAUIのデータバインディング機構を用いて
可視状態と表示内容の更新を行う。

本処理の流れを、以下に示す。

\begin{enumerate}
  \item フォルダ描画要求の受理
  ユーザ操作監視部からのフォルダ描画要求を受信し、
  現在の\verb|FolderItems|を取得する。

  \item 表示要素の生成
  \verb|FolderItems| を上から順に走査し、各\verb|FolderItem|に対して表示用のUI要素を生成する。
  ここで、\verb|Level|をインデント量に変換し、
  ツリーの階層を水平方向のオフセットとして表現する。
 
  \item アイコンおよびラベルの決定

  \verb|IsFolder|が\verb|true|の要素にはフォルダアイコンを付与し、
  \verb|IsFile|が\verb|true|の要素には何も付与しない。
  表示名は\verb|Name|を用い、ラベル文字列として描画する。
  フォルダの場合は、\verb|IsExpanded|の状態に応じて
  展開、および、折りたたみを切り替える。

  \item 可視状態の反映

  各要素の可視性は\verb|IsVisible|を反映し、
  \verb|false|の場合は表示領域から非表示とする。
  これにより、フォルダの展開、および、折りたたみ操作に伴う表示更新を再描画要求のみで反映できる。

  \item 描画内容の更新

  生成したUI要素をフォルダツリー表示領域に配置し、
  \verb|FolderItems|の更新に合わせて表示を再構成する。
\end{enumerate}

\section{VDM\texttt{++}描画処理}
VDM\texttt{++}描画処理は、VDM\texttt{++}表示領域に
プロジェクト管理部で生成した
\verb|vdmContent| を
描画する処理である。
描画部は、ユーザ操作監視部が保持する\verb|vdmContent|を参照し編集対象とするMarkdown仕様に対応したVDM\texttt{++}仕様を描画する。

本処理の流れを以下に示す。

\begin{enumerate}
  \item VDM\texttt{++}描画要求の受信

  VDM\texttt{++}描画要求を受信し、
  ユーザ操作監視部の\verb|vdmContent| を取得する。

  \item 表示領域の初期化

  VDM\texttt{++}表示領域の内容をクリアし、
  以降の文字列描画のための状態を初期化する。

  \item VDM\texttt{++}文字列の描画

  \verb|vdmContent| を表示領域に反映する。
  改行やインデントを保持したまま表示できるよう、
  文字列をそのままテキストとして設定する。

\end{enumerate}

\section{ユーザ操作監視部}\label{sec:UserOperationMonitoring}

ユーザ操作監視部は、GUI上で発生するユーザ操作イベントを監視し、
操作対象のGUI要素を特定した上で、プロジェクト管理部(\ref{sec:ProjectManagementComponent}節を参照)、および、
描画部(\ref{sec:DrawingComponent}節を参照)へ処理要求を発行する統括処理部である。
本処理部は、 MVVM における View 層と ViewModel 層間の橋渡しとして振る舞い、
描画部と連携してヒットテストを行い、
プロジェクト管理部へ更新要求、描画部へ再描画要求をする。

本処理部の入力は、 ユーザ操作イベントであるクリック、および、ドラッグ、
描画部が保持するGUI要素外接矩形情報、
GUI要素リストである。
出力は、操作対象 GUI 要素リスト、編集コマンド(追加、削除、編集)要求、ファイル更新要求、および、ユーザの操作イベントである。

本処理部で用いるユーザ操作イベントのポインタ座標は、.NET MAUI (\ref{sec:NET_MAUI}節を参照)の座標系に基づいており、
CTM領域の左上を原点とする座標系である。

本処理部の主要な処理は、以下の7つである。
\begin{itemize}
      \item ヒットテスト処理
      \item 左クリック操作監視処理
      \item 右クリック操作監視処理
      \item 左ダブルクリック操作監視処理
      \item ドラッグ移動操作監視処理
      \item 操作ボタンのクリック操作監視処理
      \item 編集コマンド実行処理
\end{itemize}

以降、各ユーザ操作に対する処理を説明する。

\subsection{ヒットテスト処理}\label{sec:UOM_HitTest}
ヒットテスト処理は、ユーザの操作座標がどのGUI要素に該当するかを判定し、
操作対象GUI要素を確定する処理である。

本処理は、描画部が保持する外接矩形情報を用いて、
操作座標が外接矩形内に存在するかを判定する。
ユーザの操作座標 $(x, y)$ が矩形内に存在するかどうかは、
式\ref{eq:hit}により判定する。

\begin{equation}\label{eq:hit}
X \leq x \leq X + W \;\land\; Y \leq y \leq Y + H
\end{equation}

本処理の流れを、以下に示す。
\begin{enumerate}
  \item 操作座標 $(x, y)$ を受け取る。
  \item 描画部から外接矩形リスト \verb|hitRects| を取得する。外接矩形リスト\verb|hitRects|は、
        表\ref{tab:hit_test_rect}に示す属性を持つ。   
  \item \verb|hitRects| を先頭から走査し、式\ref{eq:hit}により包含判定を行う。
  \item 複数のGUI要素がヒットする場合は、描画順序(前面優先)に基づき最前面GUI要素を採用する。
      描画順序は、外接矩形情報の \verb|ZIndex| 属性を参照し、値が大きいほど前面に描画されていると判定する。
  \item 採用したGUI要素を \verb|SelectedElement| として確定する。
  \item ヒットしたGUI要素が存在しない場合、\verb|SelectedElement| を \verb|null| に設定する。
\end{enumerate}

\begin{table}[tp]
\centering
\caption{外接矩形情報}
\label{tab:hit_test_rect}
\begin{tabular}{|c|c|p{8cm}|}
\hline
\textbf{属性} & \textbf{型} & \textbf{説明} \\
\hline
Element & GuiElement &
当該外接矩形に対応する GUI 要素。
ヒット判定が成立した場合、選択するGUI要素。 \\
\hline
X & float &
外接矩形の左上 X 座標。 \\
\hline
Y & float &
外接矩形の左上 Y 座標。\\
\hline
W & float &
外接矩形の幅。\\
\hline
H & float &
外接矩形の高さ。\\
\hline
ZIndex & int &
描画順序を表す値。
複数の外接矩形がヒットした場合に、前面要素を判定するために用いる。 \\
\hline
\end{tabular}
\end{table}

\subsection{左クリック操作監視処理}\label{sec:UOM_Click}
左クリック操作監視処理は、左クリックによる選択操作を検出し、
対象GUI要素の選択状態(\verb|IsSelected|)を更新する処理である。

入力は左クリック座標、出力は選択状態更新および再描画要求である。

本処理の流れを、以下に示す。
\begin{enumerate}
  \item 左クリック座標 $(x, y)$ を受け取る。座標と同時に
        左ボタンによるクリックイベントの通知を受け取る。
  \item ヒットテスト処理(\ref{sec:UOM_HitTest}節を参照)を呼び出し、\verb|SelectedElement| を確定する。
  \item 既存の選択GUI要素が存在する場合、当該GUI要素の \verb|GUIElement.IsSelected| を \verb|false| に更新する。
  \item \verb|SelectedElement| の \verb|GUIElement.IsSelected| を \verb|true| に更新する。
  \item 描画部へ再描画要求を発行し、強調表示状態を更新する。強調表示状態は、
        選択中GUI要素の枠線色をオレンジ色へ変更、および、太さを太くすることで表現する。
\end{enumerate}

\subsection{右クリック操作監視処理}\label{sec:UOM_Context}
右クリック操作監視処理は、右クリックによる編集要求を検出し、
対象GUI要素の種類(\verb|GUIElement.Type==Screen、Button、Event、Timeout|)および条件分岐(\verb|GUIElement.IsBranch == true|)に基づき、
表示する操作候補(コンテキストメニュー)を決定し、コンテキストメニュー内コマンドの選択結果を編集コマンドとして発行する処理である。

本処理の入力は、右クリック座標と現在の GUI要素リスト、出力は編集コマンド(追加、削除、編集)である。
本処理の流れを以下に示す。
\begin{enumerate}
  \item 右クリック座標 $(x, y)$ を受け取る。座標と同時に
        右ボタンによるクリックイベントの通知を受け取る。
  \item ヒットテスト処理(\ref{sec:UOM_HitTest}節を参照)により\verb|SelectedElement|を確定する。
  \item 対象GUI要素の \verb|GUIElement.Type| に応じて候補操作集合を決定する。
        例えば Button 要素には「イベント追加」を含め、Screen 要素には「画面編集」「タイムアウト編集」を含める。
  \item 条件分岐イベントの場合は、分岐領域(branch)部分に対する右クリックを区別し、
        \verb|branchIndex|(分岐番号)を編集要求に含める。
        これにより「分岐の追加、分岐の編集、分岐の削除」を対象分岐に紐付けて実行できる。
  \item ユーザが選択した候補操作を編集コマンドとして編集コマンド実行処理(\ref{sec:UOM_EditCommand}節を参照)へ通知する。
\end{enumerate}

\subsection{左ダブルクリック操作監視処理}\label{sec:UOM_DoubleClick}
左ダブルクリック操作監視処理は、画面要素に対する左クリックを監視し、
短時間に同一要素に対し2回クリックを検出した場合に左ダブルクリックとして確定し、。
プロジェクト管理部の画面切り替え処理(\ref{sec:PM_ScreenTransitionProcess}節を参照)へ
画面切り替え要求、および、対象とする画面要素の名称\verb|GUIElement.Name| を出力する。
本処理の目的は、CTM 図上の 画面要素を起点として、対応する画面クラスへ遷移することである。

本処理では、直前の左クリック情報として「直前クリック時刻」と「直前クリック対象要素」を保持する。
直前クリック時刻は \verb|_lastLeftClickTime|、直前クリック対象要素は \verb|_lastLeftClickElement| として管理し、
ダブルクリック判定の閾値は \verb|DoubleClickThresholdMs|(400 ms)とする。
なお、時刻取得には \verb|DateTime.UtcNow| を用いて差分を計算する。

本処理の流れを以下に示す。
\begin{enumerate}
  \item 左クリック座標 $(x, y)$ を受け取る。座標と同時に
        左ボタンによるクリックイベントの通知を受け取る。
  \item ヒットテスト処理により、座標 $(x, y)$ に該当する GUI 要素 \verb|SelectedElement| を特定する。
        要素が存在しない場合は、以降の処理を行わず終了する。
  \item \verb|SelectedElement| が画面\verb|GUIElement.Type == Screen| 要素である場合のみ、ダブルクリック判定対象とする。
        画面要素以外をクリックした場合は、ダブルクリック判定用の情報である
        左クリック時刻 \verb|_lastLeftClickTime| 、および、左クリック対象要素 \verb|_lastLeftClickElement| をクリアし、単発クリックとして扱う。
  \item 画面要素であった場合、現在時刻 \verb|now| を取得し、
        直前クリック対象が同一要素であること、および、
        時刻差分が閾値以内であることを同時に満たすか判定する。
        同一要素性の判定には、要素参照が同一であることを用いる。
  \item 上記条件を満たす場合、左ダブルクリックとして確定し、
        プロジェクト管理部の画面切り替え処理(\ref{sec:PM_ScreenTransitionProcess}節を参照)へ
        画面切り替え要求、および、対象とする画面要素の名称\verb|GUIElement.Name| を出力する。
        なお、確定後は誤検出防止のため、\verb|_lastLeftClickTime| と \verb|_lastLeftClickElement| を初期化する。
  \item 条件を満たさない場合は単発クリックとして扱い、
        \verb|_lastLeftClickElement \leftarrow clickedElement|、
        \verb|_lastLeftClickTime \leftarrow now| として保持情報を更新し、次回クリックに備える。
\end{enumerate}

\subsection{ドラッグ操作監視処理}\label{sec:UOM_Drag}
ドラッグ操作監視処理は、GUI要素の配置変更操作を検出し、
対象GUI要素の座標 \verb|X, Y| を更新する処理である。
入力は押下開始座標、ポインタ移動量、解放座標であり、出力は更新後座標および(必要に応じた)ファイル更新要求である。

本処理の流れを、以下に示す。
\begin{enumerate}
  \item 左クリック開始時にヒットテストを行い、ドラッグ対象 \verb|DraggingElement| を確定し、押下開始座標を取得する。
  \item 対象GUI要素の \verb|GUIElement.IsMovable| を参照し、移動不可(\verb|IsMovable == false|)である場合はドラッグを開始しない。
  \item ドラッグ開始時点のGUI要素座標と、押下開始座標との差分を \verb|offsetX, offsetY| として保持する。
  \item ポインタ移動イベントごとに、現在座標から \verb|offset| を差し引き、
        新しいGUI要素座標 $(X', Y')$ を算出して \verb|DraggingElement.X, DraggingElement.Y| を更新する。
  \item 座標更新ごとに描画部へ再描画要求を行い、移動結果を即時反映する。
  \item 解放時に開放座標を確定し、CTM要素の配置を変更した状態でスナップが入り、配置情報の保存が必要な場合は
        プロジェクト管理部のファイル更新処理(\ref{sec:FileUpdateProcess}節を参照)を要求する。
\end{enumerate}

\subsection{操作ボタンのクリック操作監視処理}\label{sec:UOM_TopBarClick}

上部操作ボタンのクリック操作監視処理は、
ツール上部に配置した操作ボタンのクリックを検知し、
対応する編集コマンドを確定して編集コマンド実行処理(\ref{sec:UOM_EditCommand}節を参照)へ通知する処理である。
本処理により、CTM領域上の右クリックメニューと同等の編集操作を、
ツール上部の操作ボタンからも実行可能とする。

本処理の入力は、上部操作ボタンのクリックイベント、および、該当ボタンに対応付けたコマンド種別である。
出力は、確定した編集コマンド種別、および、必要に応じて操作対象要素\verb|SelectedElement|を含む編集要求である。
なお、上部操作ボタンはUI部品として配置しているため、
CTM領域に対する座標入力を用いるヒットテスト処理(\ref{sec:UOM_HitTest}節を参照)は本処理では用いず、
.NET MAUI (\ref{sec:NET_MAUI}節を参照)が提供する標準のUI イベント機構であるButton のクリックイベント、および 、Command バインディングを用いて実装している。

本処理の流れを以下に示す。
\begin{enumerate}
  \item 上部操作ボタンのクリックイベントを受け取る。
        クリックイベントには、押下されたボタンに対応する編集コマンド種別を含める。
  \item 現在の選択要素 \verb|SelectedElement| を参照し、
        コマンド種別が操作対象要素を必要とするか判定する。
  \item 操作対象要素を必要とするコマンドであり、
        \verb|SelectedElement| が \verb|null| である場合は、
        エラー表示(警告)を行い、以降の処理を中断する。
  \item 操作対象要素を必要とするコマンドである場合は、
        \verb|SelectedElement| を編集要求に含める。
        操作対象要素を必要としないコマンドの場合は、
        \verb|SelectedElement| を参照せずに編集要求を構成する。
  \item 構成した編集要求を編集コマンド実行処理(\ref{sec:UOM_EditCommand}節を参照)へ通知する。
\end{enumerate}

\section{フォルダツリー操作監視処理}\label{sec:UOM_FolderTree}
フォルダツリー操作監視処理は、フォルダツリー表示領域に対する
クリック操作を監視し、フォルダの展開、および、折りたたみ、
編集対象ファイルの切替を行う処理である。
本処理はユーザ操作監視部に属し、
ツリー表示の選択状態と
プロジェクト管理部のファイル読込処理を連携させる。

本処理の流れを、以下に示す。

\begin{enumerate}
  \item ユーザによるフォルダツリー表示領域内の項目の選択イベントを受け取る。

  \item 選択した項目がフォルダかファイルかを\verb|FolderItem|の
  \verb|IsFolder|、および、\verb|IsFile|により判定する。

  \item フォルダを選択した場合は、
  \verb|IsExpanded| を反転し、
  配下の\verb|FolderItems|について
  \verb|IsVisible| を再設定する。
  これにより、展開時は子要素を表示し、
  折りたたみ時は子要素を非表示とする。
  更新後、描画部へフォルダ描画要求を通知する。

  \item ファイルを選択した場合は、\verb|FolderItem|の
  \verb|FullPath| を現在の選択ファイルとして保持し、
  プロジェクト管理部へファイル読込要求
  (\ref{sec:PM_FileLoadProcess}節を参照)、および、ファイル更新要求(\ref{sec:PM_FileUpdateProcess}節を参照)を通知する。
  読込結果として\verb|elements|、表示パターンフラグ、\verb|vdmContent|を更新し、
  描画部へ初期描画要求
  (\ref{sec:first-draw}節を参照)を通知する。

\end{enumerate}


\subsection{編集コマンド実行処理}\label{sec:UOM_EditCommand}

本処理は、右クリックメニューや画面上の操作ボタン(追加、編集、削除、保存)により確定した編集コマンドを実行し、
必要な入力値(名称、遷移先、条件文等)をダイアログで取得した上で、
GUI要素リストを更新する処理である。

本処理は、編集コマンド種別、選択中要素 \verb|SelectedElement|、GUI要素リストを入力として受け取り、
要素の追加、更新、削除を行った上で、更新後のGUI要素リスト、描画部への再描画要求、
プロジェクト管理部へのファイル更新要求、UI上の表示ボタン切替状態を出力する。

本処理の流れを以下に示す。

\begin{enumerate}
  \item 編集コマンド種別と、操作対象\verb|SelectedElement|を受け取る。

  \item コマンド種別に応じて、必要な入力項目を決定する。
  例えば、ボタン追加ではボタン名、イベント追加では「対象ボタン」「イベント内容の選択」「イベント」を要求する。
  条件分岐追加では「条件文」と「イベント」を要求する。

  \item 入力が必要な場合、ダイアログを表示し、ユーザから文字列を取得する。
  \begin{itemize}
    \item キャンセルした場合は、処理を中断し、GUI要素リスト\verb|elements|を変更しない。
    \item 入力値が空、または重複する場合は、エラー表示(警告)を行い、再入力または中断とする。
  \end{itemize}

  \item コマンド種別に応じてGUI要素リストを更新する。
  更新パターンを、以下に示す。
  \begin{itemize}
      \item \textbf{画面追加}:
      \begin{enumerate}
            \item 画面追加用のダイアログを表示し、画面要素名を取得する。
            \item 既存画面要素名と重複する場合は追加しない
            \item 新しい画面要素(\verb|Type=Screen, Name=画面要素名|)をGUI要素リスト\verb|elements|に追加する。
            \item プロジェクト管理部へ更新要求を通知する(\ref{sec:FileUpdateProcess}節を参照)。
            \item 描画部へ再描画要求(\ref{sec:Re-Draw}節を参照)を通知する。
      \end{enumerate}
      \item \textbf{ボタン追加}:
      \begin{enumerate}
            \item ボタン追加用のダイアログを表示し、ボタン名を取得する。
            \item 既存ボタン名と重複する場合は追加しない
            \item 新しいボタン要素(\verb|Type=Button, Name=ボタン名|)をGUI要素リスト\verb|elements|に追加する。
            \item プロジェクト管理部へ更新要求を通知する(\ref{sec:FileUpdateProcess}節を参照)。
            \item 描画部へ再描画要求(\ref{sec:Re-Draw}節を参照)を通知する。
      \end{enumerate}
      \item \textbf{イベント追加}:
      \begin{enumerate}
            \item 対象ボタン選択用のダイアログを表示し、対象ボタン名を取得する。
            \item 対象のボタン要素に対して、すでにイベントが存在する場合は追加しない。
            \item 条件分岐イベント選択用のダイアログを表示し、単一イベントまたは条件分岐イベントを選択する。
            \begin{itemize}
                  \item 単一イベント選択時:
                  \begin{enumerate}
                        \item 単一イベント追加用のダイアログを表示し、イベント内容を取得する。
                        \item 新しいイベント要素(\verb|Type=Event|)を生成してGUI要素リスト\verb|elements|に追加する。
                        このとき、イベント内容を \verb|Name| と \verb|Target| に保持し、
                        描画部が矢印接続を構成できるようにする。
                        \item プロジェクト管理部へ更新要求を通知する(\ref{sec:FileUpdateProcess}節を参照)。
                  \end{enumerate}
                  \item 条件分岐イベント選択時:
                  \begin{enumerate}
                        \item 条件分岐イベント追加用のダイアログを表示し、
                        分岐条件文と遷移先イベントを取得する。
                        \item 新しい条件分岐イベント要素(\verb|Type=Event, Branches=null|)を生成してGUI要素リスト\verb|elements|に追加する。
                        このとき、分岐条件文と遷移先を \verb|Branches| に保持し、
                        描画部が条件分岐表示を構成できるようにする。
                        \item プロジェクト管理部へ更新要求を通知する(\ref{sec:FileUpdateProcess}節を参照)。
                  \end{enumerate}
            \end{itemize}
            \item 描画部へ再描画要求(\ref{sec:Re-Draw}節を参照)を通知する。
      \end{enumerate}
      \item \textbf{タイムアウト追加}:
      \begin{enumerate}
            \item タイムアウト追加用のダイアログを表示し、タイムアウト内容を取得する。
            \item 新しいタイムアウト要素(\verb|Type=Timeout|)を生成してGUI要素リスト\verb|elements|に追加する。
            \item タイムアウトは1画面につき1つとする規則により、既存Timeoutが存在する場合は更新(上書き)する。
            \item プロジェクト管理部へ更新要求を通知する(\ref{sec:FileUpdateProcess}節を参照)。
            \item 描画部へ再描画要求(\ref{sec:Re-Draw}節を参照)を通知する。
      \end{enumerate}

      \item \textbf{削除}:
      \begin{enumerate}
            \item 対象GUI要素の名称を確認用ダイアログで表示し、削除確認を行う。
            \item ユーザが削除をキャンセルした場合は、処理を中断し、GUI要素リスト\verb|elements|を変更しない。
            \item ユーザが削除を確認した場合は、GUI要素リスト\verb|elements|から対象要素を除去する。
            \item ボタン要素を削除する場合、当該ボタン対象とするイベント要素も同時に削除する。
            \item プロジェクト管理部が保持するMarkdown文字列を参照し、対象要素に対する記述をMarkdown文字列から除去する。
            \item プロジェクト管理部へ更新要求を通知する(\ref{sec:FileUpdateProcess}節を参照)。
            \item 描画部へ再描画要求(\ref{sec:Re-Draw}節を参照)を通知する。
      \end{enumerate}

      \item \textbf{編集}:
      \begin{enumerate}
            \item 対象GUI要素の種別を判定する。
            \item 対象GUI要素の種別に応じて、現在の名称、イベント、
            条件文をダイアログで表示し、編集用入力欄を提供する。
            \item ユーザがキャンセルした場合は、処理を中断し、GUI要素リスト\verb|elements|を変更しない。
            \item ユーザが編集を完了した場合、入力値をGUI要素リスト\verb|elements|内の対象GUI要素に反映する。
            \item 反映後の名称が既存要素と重複する場合は処理を実行しない。
            \item プロジェクト管理部へ更新要求を通知する(\ref{sec:FileUpdateProcess}節を参照)。
            \item 描画部へ再描画要求(\ref{sec:Re-Draw}節を参照)を通知する。
      \end{enumerate}

      \item \textbf{フォルダ選択}:
      \begin{enumerate}
            \item Windows.Storage.Pickers(\ref{sec:Windows.Storage.Pickers}節を参照)を用いてインターフェースを表示し、
            プロジェクトフォルダのパス\verb|folder.path|を取得する。
            \item プロジェクト管理部へフォルダ選択要求を通知し(\ref{sec:PM_SelectFolder})、\verb|folder.path|を渡す。
            \item プロジェクト管理部から\verb|FolderItems|を受け取り、ユーザ操作部内で保持している\verb|FolderItems|を更新する。
            \item 描画部へフォルダ描画要求を通知する。
      \end{enumerate}

      \item \textbf{新規ファイル作成}:
      \begin{enumerate}
            \item ファイル名入力用ダイアログを表示しファイル名を取得する。
            \item 取得したファイル名が\verb|FolderItems|内に存在しないか探索する。
            \item \verb|FolderItems|内に同一のファイル名が存在する場合エラーを表示し、処理を中断する。
            \item \verb|FolderItems|内に同一のファイル名が存在しない場合はプロジェクト管理部へ、ファイル作成要求を通知する(\ref{sec:PM_FileCreate}節を参照)。
            \item プロジェクト管理部から\verb|FolderItems|を受け取り、ユーザ操作部内で保持している\verb|FolderItems|を更新する。
            \item 描画部へフォルダ描画要求を通知する。
      \end{enumerate}
  \end{itemize}

\end{enumerate}
