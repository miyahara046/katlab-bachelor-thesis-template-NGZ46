\chapter{拡張したツールの実装}\label{cha:Implementation}

本章では、LaTeX でソースコードを挿入する方法について説明する。
本テンプレートの実装方法については、文献\cite{kimura-automation}で詳しく述べられている。

\section{listings パッケージ}

本テンプレートでは、\verb|listings| パッケージを使用してソースコードを挿入できる。
このパッケージは、paper.tex で既に読み込まれているため、すぐに使用できる。

\section{基本的なコードの挿入}

\subsection{Python のコード例}

コード\ref{lst:python}に Python のコード例を示す。

\begin{figure}[tp]
\begin{lstlisting}[caption={Python のコード例}, label={lst:python}, language=Python]
def hello_world():
    print("Hello, World!")

if __name__ == "__main__":
    hello_world()
\end{lstlisting}
\end{figure}

\subsection{Java のコード例}

コード\ref{lst:java}に Java のコード例を示す。

\begin{figure}[tp]
\begin{lstlisting}[caption={Java のコード例}, label={lst:java}, language=Java]
public class HelloWorld {
    public static void main(String[] args) {
        System.out.println("Hello, World!");
    }
}
\end{lstlisting}
\end{figure}

\subsection{JavaScript のコード例}

コード\ref{lst:javascript}に JavaScript のコード例を示す。

\begin{figure}[tp]
\begin{lstlisting}[caption={JavaScript のコード例}, label={lst:javascript}, language=java]
function helloWorld() {
    console.log("Hello, World!");
}

helloWorld();
\end{lstlisting}
\end{figure}

\section{サポートされている言語}

listings パッケージは、以下の言語をサポートしている:

\begin{itemize}
  \item Python
  \item Java
  \item JavaScript
  \item C / C++
  \item Ruby
  \item PHP
  \item その他多数
\end{itemize}

言語の指定は、\verb|language=言語名| で行う。

\section{コードの参照}

コードを参照する場合は、図や表と同様に \verb|\ref{}| コマンドを使用する。
例えば、コード\ref{lst:python}、コード\ref{lst:java}、コード\ref{lst:javascript}のように参照できる。
