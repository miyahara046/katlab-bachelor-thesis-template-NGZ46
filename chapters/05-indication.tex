\chapter{適用例}\label{cha:Indication}
本章では、本研究で拡張後の\tool が正しく動作することを、以下の機能ごとに確認する。
\begin{itemize}
    \item ページ遷移機能
    \item 描画機能
    \item GUI編集による仕様編集機能
\end{itemize}

以降、各機能について確認する。
\section{ページ遷移機能の確認}\label{sec:switch-page}
本節では、「スタートページ」、「Markdown仕様記述ページ」、および、「GUI操作による\VDM 仕様編集ページ」の各ページで正しくページ遷移できることを確認する。

以下の手順で拡張後の\tool を実行し、確認を行う。
\begin{enumerate}
\item 「スタートページ」の「Markdown」ボタンをクリックする
\item 「Markdown仕様記述ページ」の「スタートページに戻る」ボタンをクリックする
\item 「スタートページ」の「Nocode」ボタンをクリックする
\item 「GUI操作による\VDM 仕様編集ページ」の「スタートページに戻る」ボタンをクリックする
\end{enumerate}

\begin{figure}[tp]
  \centering
  \includegraphics[width=1.0\linewidth]{./images/T-01.png}
  \caption{手順を追記した「スタートページ」の外観}
  \label{fig:T-01}
\end{figure}
\begin{figure}[tp]
  \centering
  \includegraphics[width=1.0\linewidth]{./images/T-02.png}
  \caption{手順を追記した「Markdown仕様記述ページ」の外観}
  \label{fig:T-02}
\end{figure}
\begin{figure}[tp]
  \centering
  \includegraphics[width=1.0\linewidth]{./images/T-03.png}
  \caption{手順を追記した「GUI操作による\VDM 仕様編集ページ」の外観}
  \label{fig:T-03}
\end{figure}
「スタートページ」の外観、「Markdown仕様記述ページ」の外観、「GUI操作による\VDM 仕様編集ページ」の外観に、ページ遷移ができるかどうかを確認する手順において、ボタンをクリックした箇所を図示したものを、
それぞれ図\ref{fig:T-01}、図\ref{fig:T-02}、図\ref{fig:T-03}に示す。
手順1を実行すると、図\ref{fig:T-01}の「スタートページ」から図\ref{fig:T-02}の「Markdown仕様記述ページ」へ遷移した。
手順2を実行すると、図\ref{fig:T-02}の「Markdown仕様記述ページ」から図\ref{fig:T-01}の「スタートページ」へ遷移した。
手順3を実行すると、図\ref{fig:T-01}の「スタートページ」から図\ref{fig:T-03}の「GUI操作による\VDM 仕様編集ページ」へ遷移した。
手順4を実行すると、図\ref{fig:T-03}の「GUI操作による\VDM 仕様編集ページ」から図\ref{fig:T-01}の「スタートページ」へ遷移した。

以上より、「スタートページ」、「Markdown仕様記述ページ」、および、「GUI操作による\VDM 仕様編集ページ」で、正しくページ遷移できることを確認した。

\section{描画機能の確認}
本節では、描画機能が正しく動作することを確認する。
なお、本節で適用対象とする画面遷移システムの仕様は、
株式会社フルタイムシステム\cite{fts}が開発し、実際に運用している画面遷移システムの画面の仕様の一部を参考に作成した。
作成した仕様を、図\ref{fig:FTSScreen}に示す。また、この仕様をもとに作成した画面一覧仕様をリスト\ref{lst:T-ScreenList}に、BOX数入力画面の画面仕様をリスト\ref{lst:T-Screen}に、それぞれ示す。
さらに、本節で使用するプロジェクトフォルダの構成を、図\ref{fig:test-folder-structure}に示す。

\begin{figure}[tp]
  \centering
  \includegraphics[width=1.0\linewidth]{./images/TestA.png}
  \caption{作成した画面(BOX数入力画面)の仕様}
  \label{fig:FTSScreen}
\end{figure}

\begin{figure}[tp]
\begin{lstlisting}[caption={図\ref{fig:FTSScreen}の仕様をもとに作成した画面一覧仕様(画面一覧.md)}, label={lst:T-ScreenList}]
# 画面一覧

- BOX 数入力画面
- 入力確認画面
- BOX 選択画面
- スタート画面
- ボックス設定画面
\end{lstlisting}
\end{figure}

\begin{figure}[tp]
\begin{lstlisting}[caption={図\ref{fig:FTSScreen}の仕様をもとに作成した画面仕様(BOX数入力画面.md)}, label={lst:T-Screen}]
## BOX 数入力画面
- 30 秒でタイムアウト

### 有効ボタン一覧
- 0
- 1
- 2
- 3
- 4
- 5
- 6
- 7
- 8
- 9
- 訂正
- 中止
- 確認

### イベント一覧
- タイムアウト → スタート画面へ
- 0 押下 → 表示部に0 を追加
- 1 押下 → 表示部に1 を追加
- 2 押下 → 表示部に2 を追加
- 3 押下 → 表示部に3 を追加
- 4 押下 → 表示部に4 を追加
- 5 押下 → 表示部に5 を追加
- 6 押下 → 表示部に6 を追加
- 7 押下 → 表示部に7 を追加
- 8 押下 → 表示部に8 を追加
- 9 押下 → 表示部に9 を追加
- 訂正押下 → 表示部の文字削除
- 中止押下 → ボックス設定画面へ
- 確認押下 →
  - 表示部に0 を入力している → 入力確認画面へ
  - 表示部に0 以外を入力している → BOX 選択画面へ

\end{lstlisting}
\end{figure}

\begin{figure}[tb]
\centering
\small % 全体を少し小さく(不要なら消してよい)
\begin{forest}
for tree={
font=\ttfamily,
grow'=0,
child anchor=west,
parent anchor=south,
anchor=west,
calign=first,
s sep=3pt, 
l sep=6pt,
edge path={
\noexpand\path [draw, \forestoption{edge}]
(!u.south west) +(7.5pt,0) |- (.child anchor)\forestoption{edge label};
},
before typesetting nodes={
if n=1
{insert before={[,phantom]}}
{}
},
fit=band,
before computing xy={l=35pt},
}
[FTSフォルダ
  [Screenフォルダ
    [BOX数入力画面.md]
    [BOX選択画面.md]
    [スタート画面.md]
    [ボックス設定画面.md]
    [入力確認画面.md]
  ]
  [表示部クラス.vdmpp]
  [画面一覧.md]
]
\end{forest}
\caption{テスト用プロジェクトフォルダ構成(FTSフォルダ)}
\label{fig:test-folder-structure}
\end{figure}

作成した画面一覧仕様、画面仕様、および、プロジェクトフォルダを用いて、描画機能が正しく動作することを確認する。
確認を行う機能は、以下の4つである。
\begin{itemize}
  \item CTM描画機能
  \item フォルダツリー描画機能
  \item \VDM 仕様描画機能
  \item 操作ボタン描画機能
\end{itemize}
以降、各機能について確認する。

\subsection{CTM描画機能}
CTM描画機能を、以下の2つについて確認する。
\begin{itemize}
  \item 画面一覧仕様、および、画面仕様をCTMとして表示できる
  \item 遷移先がないイベントを赤く強調表示できる
\end{itemize}

まず、リスト\ref{lst:T-ScreenList}に示した画面一覧仕様を\tool に入力する。
リスト\ref{lst:T-ScreenList}に示した画面一覧仕様のCTMを表示したCTM領域を、図\ref{fig:T-CTM-ScreenList}に示す。

\begin{figure}[tp]
  \centering
  \includegraphics[width=1.0\linewidth]{./images/T-CTM-ScreenList.png}
  \caption{リスト\ref{lst:T-ScreenList}を入力したCTM領域}
  \label{fig:T-CTM-ScreenList}
\end{figure}

次に、リスト\ref{lst:T-Screen}に示した画面仕様を\tool に入力する。
リスト\ref{lst:T-Screen}に示した画面仕様のCTMを表示したCTM領域を、図\ref{fig:T-CTM-Screen}に示す。

\begin{figure}[tp]
  \centering
  \includegraphics[width=1.0\linewidth]{./images/T-CTM-Screen.png}
  \caption{リスト\ref{lst:T-Screen}を入力したCTM領域}
  \label{fig:T-CTM-Screen}
\end{figure}

最後に、リスト\ref{lst:T-Screen}に示した画面仕様の32行目の
「\- 中止押下 \texttt{→} ボックス設定画面へ」を「\- 中止押下 \texttt{→} BOX設定画面へ」に変更して故意に誤りを加えた画面仕様を、\tool に読み込ませる。
この意図的に誤りを加えた画面仕様のCTMを表示したCTM領域を、図\ref{fig:T-CTM-Screeneror}に示す。

\begin{figure}[tp]
  \centering
  \includegraphics[width=0.8\linewidth]{./images/T-CTM-Screen-eror.png}
  \caption{リスト\ref{lst:T-Screen}に意図的に誤りを加えた画面仕様を入力したCTM領域}
  \label{fig:T-CTM-Screeneror}
\end{figure}

図\ref{fig:T-CTM-ScreenList}、図\ref{fig:T-CTM-Screen}、および、図\ref{fig:T-CTM-Screeneror}より、
画面一覧仕様、画面仕様、および、意図的に誤りを加えた画面仕様と対応の取れたCTM要素をすべて表示できているため、CTM領域の表示がすべて正しいことを確認できる。

以上より、CTM描画機能が正しく動作することを確認した。

\subsection{フォルダツリー描画機能}
フォルダツリー描画機能を、以下の4つについて確認する。
\begin{itemize}
  \item プロジェクトフォルダを表示できる
  \item フォルダを展開できる
  \item フォルダとMarkdown仕様ファイル(拡張子が.md)のみを表示できる
  \item ファイルを選択することで、編集対象のファイルを変更できる
\end{itemize}

まず、フォルダ選択で、図\ref{fig:test-folder-structure}に示したプロジェクトフォルダである「FTSフォルダ」を選択する。
「FTSフォルダ」を選択した直後のフォルダツリー表示領域を、図\ref{fig:T-Tree}に示す。
次に、「FTSフォルダ」内にある「Screenフォルダ」を左クリックする。左クリックした直後のフォルダツリー表示領域を、図\ref{fig:T-Tree-2}に示す。
\begin{figure}[tp]
  \centering
  \includegraphics[width=0.5\linewidth]{./images/T-first-tree.png}
  \caption{「FTSフォルダ」を選択した直後のフォルダツリー表示領域}
  \label{fig:T-Tree}
  \end{figure}
  \begin{figure}[tp]
  \centering
  \includegraphics[width=0.5\linewidth]{./images/T-second-tree.png}
  \caption{「Screenフォルダ」を左クリックした直後のフォルダツリー表示領域}
  \label{fig:T-Tree-2}
  \end{figure}
図\ref{fig:test-folder-structure}、図\ref{fig:T-Tree}、および、図\ref{fig:T-Tree-2}より、フォルダツリー表示領域に、プロジェクトフォルダを表示できていること、フォルダを展開できること、
および、フォルダとMarkdown仕様ファイル(拡張子が.md)のみを表示できていることを確認できる。

また、編集対象のファイルを変更できることを確認するために、フォルダツリー表示領域内の「BOX数入力画面.md」を左クリックする。
左クリックする前の「Screenフォルダ」を表すCTM領域を、図\ref{fig:T-CTM-Folder}に示す。フォルダツリー表示領域内の「BOX数入力画面.md」を左クリックした直後のCTM領域は、図\ref{fig:T-CTM-Screen}に示した表示に切り替わった。
このことから、フォルダツリー表示領域内のファイルを選択することで編集対象のファイルを変更できることを確認できる。
\begin{figure}[tp]
  \centering
  \includegraphics[width=0.5\linewidth]{./images/T-CTM-Folder.png}
  \caption{左クリック前の「Screenフォルダ」を表すCTM領域}
  \label{fig:T-CTM-Folder}
\end{figure}

以上より、フォルダツリー描画機能が正しく動作することを確認した。

\subsection{\VDM 仕様描画機能}

\VDM 仕様描画機能では「編集対象として選択したMarkdown仕様に対応する\VDM 仕様を描画できる」ことについて確認する。

まず、図\ref{fig:T-Tree}に示したフォルダツリー表示領域内の「画面一覧.md」を左クリックする。「画面一覧.md」を左クリックした直後の\VDM 仕様表示領域を、図\ref{fig:T-VDM}に示す。
次に、図\ref{fig:T-Tree-2}に示したフォルダツリー表示領域内の「BOX数入力画面.md」を左クリックする。「BOX数入力画面.md」を左クリックした直後の\VDM 仕様表示領域を、図\ref{fig:T-VDM-2}に示す。
\begin{figure}[tp]
  \centering
  \includegraphics[width=0.8\linewidth]{./images/T-VDM.png}
  \caption{「画面一覧」を左クリックした直後の\VDM 仕様表示領域}
  \label{fig:T-VDM}
\end{figure}
\begin{figure}[tp]
  \centering
  \includegraphics[width=0.8\linewidth]{./images/T-VDM-2.png}
  \caption{「BOX数入力画面.md」を左クリックした直後の\VDM 仕様表示領域}
  \label{fig:T-VDM-2}
\end{figure}

図\ref{fig:T-VDM}より、リスト\ref{lst:T-ScreenList}に示した画面一覧フィールドの内容と、\VDM 仕様内の「types」が対応しているため、「画面一覧.md」と対応した\VDM 仕様を描画できていることが確認できる。
また、図\ref{fig:T-VDM-2}より、リスト\ref{lst:T-Screen}に示したタイムアウト記述フィールドと\VDM 仕様内の「values」が、
有効ボタン記述フィールドと\VDM 仕様内の「types」が
、および、イベント記述フィールドと\VDM 仕様内の「Operations」が、それぞれ対応しているため、「BOX数入力画面.md」と対応した\VDM 仕様を描画できていることが確認できる。
そのため、それぞれの仕様に対応した\VDM 仕様を描画できることが確認できる。

以上より、\VDM 仕様描画機能が正しく動作することを確認した。

\subsection{操作ボタン描画機能}

\begin{figure}[tp]
  \centering
  \includegraphics[width=0.4\linewidth]{./images/T-04.png}
  \caption{「GUI操作による\VDM 仕様編集ページ」へ遷移直後の操作ボタン領域}
  \label{fig:T-04}
\end{figure}
\begin{figure}[tp]
  \centering
  \includegraphics[width=0.4\linewidth]{./images/T-05.png}
  \caption{「FTSフォルダ」選択直後の操作ボタン領域}
  \label{fig:T-05}
\end{figure}\begin{figure}[tp]
  \centering
  \includegraphics[width=0.4\linewidth]{./images/T-06.png}
  \caption{「画面一覧」を左クリックした直後の操作ボタン領域}
  \label{fig:T-06}
\end{figure}\begin{figure}[tp]
  \centering
  \includegraphics[width=0.9\linewidth]{./images/T-07.png}
  \caption{「BOX数入力画面.md」を左クリックした直後の操作ボタン領域}
  \label{fig:T-07}
\end{figure}
操作ボタン描画機能では「表示パターンに対応した操作ボタンの表示切替ができる」ことについて確認する。

まず、「スタートページ」から「GUI操作による\VDM 仕様編集ページ」へ、ページを遷移する。「GUI操作による\VDM 仕様編集ページ」へ遷移直後の操作ボタン領域を、図\ref{fig:T-04}に示す。
次に、図\ref{fig:T-Tree}に示したフォルダツリー表示領域内の「FTSフォルダ」を選択する。「FTSフォルダ」選択直後の操作ボタン領域を、図\ref{fig:T-05}に示す。
次に、図\ref{fig:T-Tree}に示したフォルダツリー表示領域内の「画面一覧.md」を左クリックする。「画面一覧.md」を左クリックした直後の操作ボタン領域を、図\ref{fig:T-06}に示す。
最後に、図\ref{fig:T-Tree-2}に示したフォルダツリー表示領域内の「BOX数入力画面.md」を左クリックする。「BOX数入力画面.md」を左クリックした直後の操作ボタン領域を、図\ref{fig:T-07}に示す。

図\ref{fig:T-04}は表示パターンA(\ref{sec:GUI-operation-control-function}節を参照)に、図\ref{fig:T-05}は表示パターンD(\ref{sec:GUI-operation-control-function}節を参照)に、図\ref{fig:T-06}は表示パターンB(\ref{sec:GUI-operation-control-function}節を参照)に、図\ref{fig:T-07}は表示パターンC(\ref{sec:GUI-operation-control-function}節を参照)に、それぞれ対応していることが確認できる。

以上より、操作ボタン描画機能が正しく動作することを確認した。

また、描画機能の4つの機能が正しく動作したことにより、描画機能が正しく動作することを確認した。


\section{GUI編集による仕様編集機能の確認}
本節では、GUI編集による仕様編集機能が正しく動作することを確認する。
確認する機能は、以下の3つである。
\begin{itemize}
\item メニューバーの操作による仕様生成補助機能
\item 操作ボタン領域のボタン操作によるCTM編集機能
\item CTM領域上でのユーザ操作によるCTM編集機能
\end{itemize}

以降、各機能について確認する。
\subsection{メニューバーの操作による仕様生成補助機能}

メニューバーの操作による仕様生成補助機能では、以下の2つの機能について確認し、メニューバーの操作による仕様生成補助機能が正しく動作することを確認する。
\begin{itemize}
  \item フォルダを選択
  \item 新規ファイルの作成
\end{itemize}

以降、各機能について確認する。
\subsubsection{フォルダを選択}\label{sec:SelectFolder}\label{sec:folfol}
本機能では、ユーザが、メニューバー内の「フォルダを選択」を選択した際に、
拡張後の\tool がフォルダ選択インターフェースを出力し、選択したフォルダに移動できることを確認する。本機能の確認では、図\ref{fig:test-folder-structure}に示した「FTSフォルダ」と、
図\ref{fig:Folder-origin}に示すテスト用に作成した「テストフォルダ」を用いて確認を行う。
\begin{figure}[tb]
\centering
\small % 全体を少し小さく(不要なら消してよい)
\begin{forest}
for tree={
font=\ttfamily,
grow'=0,
child anchor=west,
parent anchor=south,
anchor=west,
calign=first,
s sep=3pt, 
l sep=6pt,
edge path={
\noexpand\path [draw, \forestoption{edge}]
(!u.south west) +(7.5pt,0) |- (.child anchor)\forestoption{edge label};
},
before typesetting nodes={
if n=1
{insert before={[,phantom]}}
{}
},
fit=band,
before computing xy={l=35pt},
}
[テストフォルダ
  [テスト1フォルダ
    [テスト3フォルダ
    [画面30.md]
    [画面30.vdmpp]
    [画面31.md]
    ]
    [画面10.md]
    [画面10.vdmpp]
    [画面11.md]
  ]
  [テスト2フォルダ
    [画面20.md]
    [画面20.vdmpp]
    [画面21.md]
    [画面21.vdmpp]
  ]
  [表示部クラス.vdmpp]
  [画面1.md]
  [画面1.vdmpp]
  [画面管理.md]
  [画面管理.vdmpp]
]
\end{forest}
\caption{テスト用プロジェクトフォルダ構成(テストフォルダ)}
\label{fig:Folder-origin}
\end{figure}

\begin{figure}[tp]
  \centering
  \includegraphics[width=1.0\linewidth]{./images/Select-Folder-interface.png}
  \caption{フォルダ選択インターフェースで「テストフォルダ」を選択時の\tool の外観}
  \label{fig:Select-Folder-interface}
\end{figure}

まず、「FTSフォルダ」を選択し、フォルダツリー表示領域に「FTSフォルダ」を表示する。この時のフォルダツリー表示領域は、図\ref{fig:T-Tree-2}に示したものである。
次に、メニューバーの「フォルダを選択」を選択し、フォルダ選択インターフェース内で「テストフォルダ」を選択する。フォルダ選択インターフェースで「テストフォルダ」を
選択時の\tool の外観を、図\ref{fig:Select-Folder-interface}に示す。
最後に、フォルダ選択インターフェースの「フォルダの選択」を左クリックする。クリック後のフォルダツリー表示領域を、図\ref{fig:T-Tree-3}に示す。

\begin{figure}[tp]
  \centering
  \includegraphics[width=0.4\linewidth]{./images/T-Tree-3.png}
  \caption{「テストフォルダ」選択時のフォルダツリー表示領域}
  \label{fig:T-Tree-3}
\end{figure}

図\ref{fig:Select-Folder-interface}、および、図\ref{fig:T-Tree-3}より、メニューバーの「フォルダを選択」を選択した際に、
フォルダ選択インターフェースを出力し、フォルダ選択インターフェースの選択に応じたフォルダに移動できることを確認できる。

\subsubsection{新規ファイル作成}

本機能では、ユーザが、メニューバー内の「新規ファイル作成」を選択した際に、
拡張後の\tool が新規ファイル作成ダイアログを出力し、入力に則った新規ファイルをプロジェクトフォルダに作成できるかを確認する。
本機能の確認では、図\ref{fig:Folder-origin}に示した「テストフォルダ」を用いて確認する。
まず、「テストフォルダ」を選択し、フォルダツリー表示領域に図\ref{fig:T-Tree-3}に示したフォルダツリーを表示する。
次に、メニューバーの「新規ファイル作成」を選択し、新規ファイル作成ダイアログに「画面2」を入力する。「画面2」を入力した新規ファイル作成ダイアログを、図\ref{fig:T-NewFile}に示す。
最後に、新規ファイル作成ダイアログの「作成」を左クリックする。クリック後のフォルダツリー表示領域を、図\ref{fig:T-Tree-4}に示す。

\begin{figure}[tp]
  \centering
  \includegraphics[width=0.5\linewidth]{./images/T-NewFile.png}
  \caption{「画面2」を入力した新規ファイル作成ダイアログ}
  \label{fig:T-NewFile}
\end{figure}
\begin{figure}[tp]
  \centering
  \includegraphics[width=0.5\linewidth]{./images/T-Tree-4.png}
  \caption{「画面2」を作成後のフォルダツリー領域}
  \label{fig:T-Tree-4}
\end{figure}
図\ref{fig:T-NewFile}、および、図\ref{fig:T-Tree-4}より、ニューバーの「新規ファイル作成」を選択した際に、
新規ファイル作成ダイアログを出力し、入力に応じた新規ファイル「画面2.md」がフォルダツリー内に表示されているため、新規ファイルがプロジェクトフォルダに作成できることを確認できる。

以上より、メニューバーの操作による仕様生成補助機能が正しく動作することを確認した。

\subsection{操作ボタン領域のボタン操作によるCTM編集機能}\label{sec:CTM-edit}

操作ボタン領域のボタン操作によるCTM編集機能では、以下の9つの機能について確認し、操作ボタン領域のボタン操作によるCTM編集機能が正しく動作することを確認する。
\begin{itemize}
  \item クラスの種類選択および追加
  \item 画面の追加
  \item ボタンの追加
  \item タイムアウトの追加
  \item イベントの追加
  \item クラス名(画面名)の変更
  \item 削除
  \item フォルダの選択
  \item スタートページに戻る
\end{itemize}

以降、各機能について確認する。
\subsubsection{クラスの種類選択および追加}
本機能では、ユーザが「クラスの種類選択・追加」ボタンを左クリックした際に、
拡張後の\tool がクラスの種類選択ダイアログを表示すること、クラスの種類選択ダイアログによる選択に応じて、\VDM 仕様に画面管理クラスを追加すること、さらに、
拡張後の\tool が画面クラス追加ダイアログを表示し、\VDM 仕様に、入力に応じたクラス名の画面クラスを追加することを、それぞれ確認する。

まず、画面管理クラスの追加について確認する。
「クラスの種類選択・追加」ボタンを左クリックした際の\tool の外観を、図\ref{fig:T_1}に示す。
\begin{figure}[tp]
  \centering
  \includegraphics[width=1.0\linewidth]{./images/T_1.png}
  \caption{「クラスの種類選択・追加」ボタンを左クリックした際の\tool の外観}
  \label{fig:T_1}
\end{figure}

クラスの種類選択ダイアログで、「画面管理クラスの追加」を選択することで画面管理クラスの追加を確認できる。
クラスの種類選択ダイアログで、「画面管理クラスの追加」を選択した際の\VDM 仕様表示領域を、図\ref{fig:T_2}に示す。
\begin{figure}[tp]
  \centering
  \includegraphics[width=0.4\linewidth]{./images/T_2.png}
  \caption{「画面管理クラスの追加」を選択した際の\VDM 仕様表示領域}
  \label{fig:T_2}
\end{figure}

次に、画面クラスの追加について確認する。
上記で、すでにプロジェクトフォルダ内に画面管理クラスを作成したため、「クラスの種類選択・追加」ボタンをクリックすることで、
画面クラス追加ダイアログを表示する。画面クラス追加ダイアログに、
追加する画面クラス名を入力し、「OK」を選択することで画面クラスを追加できる。
本機能では、「テスト」クラスを追加する。

画面クラス追加ダイアログに「テスト」を入力した\tool の外観を、図\ref{fig:T_3}に示す。
また、「テスト」クラスを追加した\VDM 仕様表示領域を、図\ref{fig:T_4}に示す

\begin{figure}[tp]
  \centering
  \includegraphics[width=1.0\linewidth]{./images/T_3.png}
  \caption{画面クラス追加ダイアログに「テスト」を入力した\tool の外観}
  \label{fig:T_3}
\end{figure}
\begin{figure}[tp]
  \centering
  \includegraphics[width=0.4\linewidth]{./images/T_4.png}
  \caption{「テスト」クラスを追加した\VDM 仕様表示領域}
  \label{fig:T_4}
\end{figure}

図\ref{fig:T_1}、図\ref{fig:T_2}、図\ref{fig:T_3}、および、図\ref{fig:T_4}より、ユーザが「クラスの種類選択・追加」ボタンを左クリックした際に、
クラスの種類選択ダイアログを表示すること、クラスの種類選択ダイアログによる選択に応じて、\VDM 仕様に画面管理クラスを追加すること、さらに、
画面クラス追加ダイアログを表示し、入力に応じたクラス名の画面クラスを\VDM 仕様に追加することを確認できる。

\subsubsection{画面の追加}
本機能では、ユーザが「画面の追加」ボタンを左クリックした際に、拡張後の\tool が画面追加ダイアログを表示し、
入力に応じた画面要素をCTM領域に追加でき、画面に関する情報を\VDM 仕様に追加できることを確認する。

今回は、「1画面」という名前の画面要素を追加する。
まず、「画面の追加」ボタンを左クリックし、表示する画面追加ダイアログに「1画面」を入力する。
画面追加ダイアログに「1画面」を入力した\tool の外観を、図\ref{fig:T_5}に示す。次に「OK」を選択することで、画面要素を追加する。
「1画面」を追加したCTM領域と、追加した画面に関する情報をオレンジの枠線で囲った\VDM 仕様表示領域を、図\ref{fig:T_6}に示す。

\begin{figure}[tp]
  \centering
  \includegraphics[width=1.0\linewidth]{./images/T_5.png}
  \caption{画面追加ダイアログで「1画面」を入力した\tool の外観}
  \label{fig:T_5}
\end{figure}
\begin{figure}[tp]
  \centering
  \includegraphics[width=0.9\linewidth]{./images/T_6.png}
  \caption{「1画面」を追加したCTM領域と追加した画面に関する情報をオレンジの枠線で囲った\VDM 仕様表示領域}
  \label{fig:T_6}
\end{figure}

図\ref{fig:T_5}、および、図\ref{fig:T_6}より、ユーザが「画面の追加」ボタンを左クリックした際に、画面追加ダイアログを表示し、
画面追加ダイアログでの入力に応じた画面要素をCTM領域に追加でき、画面に関する情報を\VDM 仕様に追加できることを確認できる。

\subsubsection{ボタンの追加}
本機能では、ユーザが「ボタンの追加」ボタンを左クリックした際に、拡張後の\tool がボタン追加ダイアログを表示し、
入力に応じたボタン要素をCTM領域に追加でき、ボタンに関する情報を\VDM 仕様に追加できることを確認する。

今回は、「1ボタン」という名前のボタン要素を追加する。
まず、「ボタンの追加」ボタンを左クリックし、表示するボタン追加ダイアログに「1ボタン」を入力する。
ボタン追加ダイアログに「1ボタン」を入力した\tool の外観を、図\ref{fig:T_7}に示す。次に「OK」を選択することで、ボタン要素を追加する。
「1ボタン」を追加したCTM領域と、追加したボタンに関する情報をオレンジの枠線で囲った\VDM 仕様表示領域を、図\ref{fig:T_8}に示す。

\begin{figure}[tp]
  \centering
  \includegraphics[width=1.0\linewidth]{./images/T_7.png}
  \caption{ボタン追加ダイアログで「1ボタン」を入力した\tool の外観}
  \label{fig:T_7}
\end{figure}
\begin{figure}[tp]
  \centering
  \includegraphics[width=1.0\linewidth]{./images/T_8.png}
  \caption{「1ボタン」を追加したCTM領域と追加したボタンに関する情報をオレンジの枠線で囲った\VDM 仕様表示領域}
  \label{fig:T_8}
\end{figure}

図\ref{fig:T_7}、および、図\ref{fig:T_8}より、ユーザが「ボタンの追加」ボタンを左クリックした際に、ボタン追加ダイアログを表示し、
ボタン追加ダイアログでの入力に応じたボタン要素をCTM領域に追加でき、ボタンに関する情報を\VDM 仕様に追加できることを確認できる。

\subsubsection{イベントの追加}
本機能では、ユーザが「イベントの追加」ボタンを左クリックした際の、以下の6つの機能を確認する。
\begin{itemize}
  \item 拡張後の\tool が対象ボタン選択ダイアログを表示できる
  \item 拡張後の\tool が分岐イベント選択ダイアログを表示できる
  \item 拡張後の\tool が分岐イベント入力ダイアログを表示できる
  \item 拡張後の\tool が単一イベント入力ダイアログを表示できる
  \item 拡張後の\tool が別の分岐イベント追加確認ダイアログを表示できる
  \item 入力に則ったイベント要素、または、分岐イベント要素をCTM領域に追加でき、単一イベント、および、分岐イベントに関する情報を\VDM 仕様に追加できる
\end{itemize}

\begin{figure}[tp]
  \centering
  \includegraphics[width=1.0\linewidth]{./images/T_12.png}
  \caption{対象ボタン選択ダイアログを表示した\tool の外観}
  \label{fig:T_12}
\end{figure}
\begin{figure}[tp]
  \centering
  \includegraphics[width=1.0\linewidth]{./images/T_13.png}
  \caption{分岐イベント選択ダイアログを表示した\tool の外観}
  \label{fig:T_13}
\end{figure}
今回は、単一イベント追加時、および、分岐イベント追加時のCTM領域と\VDM 仕様表示領域を確認することで正常に動作することを確認する。
単一イベントは、「1ボタン」を対象として、イベント種別「入力」を選択し、「1」を入力する。
分岐イベントは、「2ボタン」を対象として、条件分岐1つ目に「1を入力している」を入力し、イベント種別「削除」を選択する。そして、別の分岐イベント追加を選択する。
最後に、条件分岐2つ目に「1を入力していない」を入力し、イベント種別「遷移」を選択する。この際、遷移先画面名として「スタート画面」を入力する。

「イベントの追加」ボタンを左クリックし、対象ボタン選択ダイアログを表示する。対象ボタン選択ダイアログを表示した\tool の外観を、図\ref{fig:T_12}に示す。
まず、「1ボタン」を選択する。対象ボタン選択後、分岐イベント選択ダイアログを表示する。分岐イベント選択ダイアログを表示した\tool の外観を、図\ref{fig:T_13}に示す。

分岐選択ダイアログで、「いいえ」を選択することで単一イベント入力ダイアログを表示する。単一イベント入力ダイアログでイベント種別を「入力」にし、遷移先および操作内容入力欄に「1」を入力する。
単一イベント入力ダイアログのイベント種別を「入力」にし、遷移先および操作内容入力欄に「1」を入力した\tool の外観を、図\ref{fig:T_14}に示す。
\begin{figure}[tp]
  \centering
  \includegraphics[width=1.0\linewidth]{./images/T_14.png}
  \caption{単一イベント入力ダイアログのイベント種別を「入力」にし遷移先および操作内容入力欄に「1」を入力した\tool の外観}
  \label{fig:T_14}
\end{figure}
入力を入れた状態で単一イベント入力ダイアログの「OK」を選択することで、
単一イベントを追加する。
単一イベントを追加したCTM領域を図\ref{fig:T_15}に、追加した単一イベントに関する情報をオレンジの枠線で囲った\VDM 仕様表示領域を図\ref{fig:T_16}に、それぞれ示す。

\begin{figure}[tp]
  \centering
  \includegraphics[width=0.6\linewidth]{./images/T_15.png}
  \caption{単一イベントを追加したCTM領域}
  \label{fig:T_15}
\end{figure}
\begin{figure}[tp]
  \centering
  \includegraphics[width=0.7\linewidth]{./images/T_16.png}
  \caption{追加した単一イベントに関する情報をオレンジの枠線で囲った\VDM 仕様表示領域}
  \label{fig:T_16}
\end{figure}

図\ref{fig:T_12}、図\ref{fig:T_13}、図\ref{fig:T_14}、図\ref{fig:T_15}、および、図\ref{fig:T_16}より、正しいダイアログを表示し、CTM領域にイベント要素を追加できていること、および、\VDM 仕様に単一イベントに関する情報を追加できていることを確認できる。

次に、対象ボタン選択ダイアログで、「2ボタン」を選択する。「2ボタン」を選択し、分岐選択ダイアログの「はい」を選択することで分岐イベント入力ダイアログを表示する。分岐イベント入力ダイアログで、
条件入力欄に「1を入力している」を入力し、イベント種別を「削除」にする。分岐イベント入力ダイアログの
条件入力欄に「1を入力している」を入力し、イベント種別を「削除」にした\tool の外観を、図\ref{fig:T_17}に示す。

\begin{figure}[tp]
  \centering
  \includegraphics[width=1.0\linewidth]{./images/T_17.png}
  \caption{分岐イベント入力ダイアログの条件入力欄に「1を入力している」を入力しイベント種別を「削除」にした\tool の外観}
  \label{fig:T_17}
\end{figure}
入力項目を入れた状態で分岐イベント入力ダイアログの「OK」を選択することで、別の分岐イベント追加確認ダイアログを表示する。
別の分岐イベント追加確認ダイアログを表示した\tool の外観を、図\ref{fig:T_18}に示す。
\begin{figure}[tp]
  \centering
  \includegraphics[width=1.0\linewidth]{./images/T_18.png}
  \caption{別の分岐イベント追加確認ダイアログを表示した\tool の外観}
  \label{fig:T_18}
\end{figure}
別の分岐イベント追加確認ダイアログで「はい」を選択することで、再び、分岐イベント入力ダイアログを表示する。
今度は、分岐イベント入力ダイアログの条件入力欄で「1を入力していない」を入力し、イベント種別を「遷移」にし、遷移先および操作内容入力欄に「スタート画面」を入力する。
分岐イベント入力ダイアログの条件入力欄に「1を入力していない」を入力し、イベント種別を「遷移」にし、遷移先および操作内容入力欄に「スタート画面」を入力した\tool の外観を、図\ref{fig:T_19}に示す。
\begin{figure}[tp]
  \centering
  \includegraphics[width=1.0\linewidth]{./images/T_19.png}
  \caption{分岐イベント入力ダイアログの条件入力欄に「1を入力していない」を入力しイベント種別を「遷移」とし遷移先および操作内容入力欄に「スタート画面」を入力した\tool の外観}
  \label{fig:T_19}
\end{figure}
入力項目をすべて入れた状態で分岐イベント入力ダイアログの「OK」を選択することで、別の分岐イベント追加確認ダイアログを表示する。
今回は、別の分岐イベント追加確認ダイアログで「いいえ」を選択することで、CTM領域に分岐イベント要素を追加する。また、\VDM 仕様表示領域に分岐イベントに関する情報を追加する。
分岐イベント要素を追加したCTM領域を図\ref{fig:T_20}に、分岐イベントに関する情報をオレンジの枠線で囲った\VDM 仕様表示領域を図\ref{fig:T_21}に、それぞれ示す。
\begin{figure}[tp]
  \centering
  \includegraphics[width=0.7\linewidth]{./images/T_20.png}
  \caption{分岐イベントを追加したCTM領域}
  \label{fig:T_20}
\end{figure}
\begin{figure}[tp]
  \centering
  \includegraphics[width=0.8\linewidth]{./images/T_21.png}
  \caption{分岐イベントに関する情報をオレンジの枠線で囲った\VDM 仕様表示領域}
  \label{fig:T_21}
\end{figure}

図\ref{fig:T_17}、図\ref{fig:T_18}、図\ref{fig:T_19}、図\ref{fig:T_20}、および、図\ref{fig:T_21}より、
正しいダイアログを表示し、CTM領域に分岐イベント要素の追加、および、\VDM 仕様表示領域に分岐イベントに関する情報を追加できていることを確認できる。

\subsubsection{タイムアウトの追加}

本機能では、ユーザが「タイムアウトの追加」ボタンを左クリックした際に、拡張後の\tool がタイムアウト設定ダイアログを表示し、
入力に応じたタイムアウト要素をCTM領域に追加でき、タイムアウトに関する情報を\VDM 仕様に追加できることを確認する。

今回は、タイムアウト時間に「30秒」、タイムアウト後遷移先に「1画面」を追加する。
まず、「タイムアウトの追加」ボタンを左クリックし、表示するタイムアウト追加ダイアログの秒数を入力する入力欄に「30」、遷移先を入力する入力欄に「1画面」を入力する。
タイムアウト追加ダイアログに「30」と「1画面」を入力した\tool の外観を、図\ref{fig:T_9}に示す。
\begin{figure}[tp]
  \centering
  \includegraphics[width=1.0\linewidth]{./images/T_9.png}
  \caption{タイムアウト設定ダイアログで「30」と「1画面」を入力した\tool の外観}
  \label{fig:T_9}
\end{figure}
次に「OK」を選択することで、タイムアウトを追加する。
「30」と「1画面」を追加したCTM領域を図\ref{fig:T_10}に、追加したタイムアウトに関する情報をオレンジの枠線で囲った\VDM 仕様表示領域を図\ref{fig:T_11}に、それぞれ示す。
\begin{figure}[tp]
  \centering
  \includegraphics[width=0.7\linewidth]{./images/T_10.png}
  \caption{「30」と「1画面」を追加したCTM領域}
  \label{fig:T_10}
\end{figure}
\begin{figure}[tp]
  \centering
  \includegraphics[width=0.8\linewidth]{./images/T_11.png}
  \caption{追加したタイムアウトに関する情報をオレンジの枠線で囲った\VDM 仕様表示領域}
  \label{fig:T_11}
\end{figure}

図\ref{fig:T_9}、図\ref{fig:T_10}、および、図\ref{fig:T_11}より、ユーザが「タイムアウトの追加」ボタンを左クリックした際に、タイムアウト追加ダイアログを表示し、
タイムアウト追加ダイアログでの入力に応じたタイムアウト要素をCTM領域に追加でき、タイムアウトに関する情報を\VDM 仕様に追加できることを確認できる。

\subsubsection{クラス名(画面名)の変更}
本機能では、ユーザが「クラス名(画面名)の変更」ボタンを左クリックした際に、拡張後の\tool がクラス名変更ダイアログを表示し、
入力に応じた変更を\VDM 仕様に追加できることを確認する。

図\ref{fig:T_11}に示した\VDM 仕様に対応した画面仕様に対して「クラス名(画面名)の変更」を行いクラス名を「テスト」から「ログイン画面」へと変更する。
まず、「クラス名(画面名)の変更」ボタンを左クリックし、表示するクラス名変更ダイアログに「ログイン画面」を入力する。
クラス名変更ダイアログに「ログイン画面」を入力した\tool の外観を、図\ref{fig:T_22}に示す。次に「変更」を選択することで、クラス名を変更する。
クラス名を「ログイン画面」に変更した\VDM 仕様表示領域を、図\ref{fig:T_23}に示す。

\begin{figure}[tp]
  \centering
  \includegraphics[width=1.0\linewidth]{./images/T_22.png}
  \caption{クラス名変更ダイアログに「ログイン画面」を入力した\tool の外観}
  \label{fig:T_22}
\end{figure}
\begin{figure}[tp]
  \centering
  \includegraphics[width=0.8\linewidth]{./images/T_23.png}
  \caption{クラス名を「ログイン画面」に変更した\VDM 仕様表示領域}
  \label{fig:T_23}
\end{figure}

図\ref{fig:T_22}、および、図\ref{fig:T_23}より、ユーザが「クラス名(画面名)の変更」ボタンを左クリックした際に、クラス名変更ダイアログを表示し、
\VDM 仕様の「class」に対して入力に応じたクラス名に変更できることを確認できる。


\subsubsection{削除}\label{sec:delete}

本機能では、ユーザが「削除」ボタンを左クリックした際に、拡張後の\tool が選択したCTM要素をCTM領域から削除できること、
および、選択したCTM要素に対応する内容を\VDM 仕様表示領域から削除できることを確認する。

図\ref{fig:T_20}に示したCTM領域、および、図\ref{fig:T_23}に示した\VDM 仕様に対して削除を行う。
今回は「1ボタン」の削除を行う。CTM領域の「1ボタン」を左クリックし、「削除」ボタンを左クリックすることで、選択したCTM要素を削除する。
「1ボタン」を削除した後のCTM領域を図\ref{fig:T_24}に、削除する前と削除した後の\VDM 仕様表示領域を図\ref{fig:T_25}に、それぞれ示す。

\begin{figure}[tp]
  \centering
  \includegraphics[width=0.7\linewidth]{./images/T_24.png}
  \caption{「1ボタン」を削除した後のCTM領域}
  \label{fig:T_24}
\end{figure}
\begin{figure}[tp]
  \centering
  \includegraphics[width=1.0\linewidth]{./images/T_25.png}
  \caption{「1ボタン」を削除する前と削除した後の\VDM 仕様表示領域}
  \label{fig:T_25}
\end{figure}

図\ref{fig:T_24}、および、図\ref{fig:T_25}より、ユーザがCTM領域のCTM要素を左クリックし、「削除」ボタンを左クリックした際に、対象の要素を
CTM領域の「1ボタン」とそれに対応づいていたイベント要素の削除、および、\VDM 仕様の「types」から「1ボタン」に対応するボタンに関する情報、「operations」から「1ボタン」に対応するボタンに関する情報の削除ができることを確認できる。

\subsubsection{フォルダの選択}

本機能では、ユーザが「フォルダの選択」ボタンを左クリックした際に、フォルダ選択インターフェースを表示できるかを確認する。選択したフォルダを開けるかどうかは、メニューバーの操作による仕様生成補助機能の確認機能の1つである「フォルダを選択」(\ref{sec:folfol}節を参照)で確認済みであるため、本機能では確認できているものと判断する。

「フォルダの選択」ボタンを左クリックした際に、表示したフォルダ選択インターフェースを、図\ref{fig:Ilost}に示す。

\begin{figure}[tp]
  \centering
  \includegraphics[width=1.0\linewidth]{./images/Ilost.png}
  \caption{「フォルダの選択」ボタンを左クリックした際に表示したフォルダ選択インターフェース}
  \label{fig:Ilost}
\end{figure}

図\ref{fig:Ilost}より、「フォルダの選択」ボタンを左クリックした際に、フォルダ選択ダイアログを表示できることを確認できる。

\subsubsection{スタートページに戻る}

本機能では、ユーザが「スタートページに戻る」ボタンを左クリックした際に、スタートページを表示できるかを確認する。本機能は、ページ遷移機能(\ref{sec:switch-page}節を参照)の際に確認しているため、すでに確認できているものと判断する。

以上より、操作ボタン領域のボタン操作によるCTM編集機能が正しく動作することを確認した。

\subsection{CTM領域上でのユーザ操作によるCTM編集機能}
CTM領域上でのユーザ操作によるCTM編集機能が正しく動作することを確認するために、以下の2つの機能について確認する。
\begin{itemize}
  \item 右クリック操作
  \item 左クリック操作
\end{itemize}

CTM領域上でのユーザ操作によるCTM編集機能では、図\ref{fig:test-folder-structure}に示した「FTSフォルダ」、リスト\ref{lst:T-ScreenList}に示した「画面一覧.md」、および、リスト\ref{lst:T-Screen}に示した「BOX数入力画面.md」を使用して確認を行う。
以降、各機能についてそれぞれ確認する。
\subsubsection{右クリック操作}

右クリック操作では、コンテキストメニューを表示し、コンテキストメニュー独自の要素である「要素の編集」、および、「コピー\&ペースト」ができることを確認する。
この2つの要素以外は、操作ボタン領域のボタン操作によるCTM編集機能(\ref{sec:CTM-edit}節を参照)で確認済みであるため、本機能では確認できているものと判断する。

まず、「要素の編集」を確認する。
使用する仕様のファイルは「BOX数入力画面.md」である。リスト\ref{lst:T-Screen}に示した「BOX数入力画面.md」に対するCTM領域の「中止押下 → ボックス設定画面へ」に該当するイベント要素を右クリックする。
右クリックした際のコンテキストメニューを表示した\tool の外観を、図\ref{fig:ContextMenue}に示す。
\begin{figure}[tp]
  \centering
  \includegraphics[width=1.0\linewidth]{./images/ContextMenue.png}
  \caption{イベント要素に対するコンテキストメニューを表示した\tool の外観}
  \label{fig:ContextMenue}
\end{figure}
「要素の編集」である「イベント変更」を選択する。「イベント変更」を選択し、編集用ダイアログを表示する。
「イベント変更」で表示する編集用ダイアログを、図\ref{fig:Contextex}に示す。
\begin{figure}[tp]
  \centering
  \includegraphics[width=0.6\linewidth]{./images/Contextex.png}
  \caption{「イベント変更」で表示する編集用ダイアログ}
  \label{fig:Contextex}
\end{figure}
今回は、中止ボタン押下時の遷移先の画面を「設定画面」に変更する。編集用ダイアログの入力欄を「設定画面へ」に変更し、「OK」を選択することで、CTM領域、
および、\VDM 仕様に該当箇所の変更を加える。変更を加えたCTM領域を図\ref{fig:T_26}に、変更を加えた\VDM 仕様表示領域をオレンジ色の枠で囲ったものを図\ref{fig:T_27}に、それぞれ示す。

\begin{figure}[tp]
  \centering
  \includegraphics[width=1.0\linewidth]{./images/T_26.png}
  \caption{変更を加えたCTM領域}
  \label{fig:T_26}
\end{figure}
\begin{figure}[tp]
  \centering
  \includegraphics[width=0.85\linewidth]{./images/T_27.png}
  \caption{変更を加えた\VDM 仕様表示領域}
  \label{fig:T_27}
\end{figure}

次に「コピー\&ペースト」について確認する。
リスト\ref{lst:T-Screen}に示した「BOX数入力画面.md」に対するCTM領域の「確認」に該当するボタン要素を右クリックする。
右クリックした際のコンテキストメニューを表示した\tool の外観を、図\ref{fig:T_28}に示す。
\begin{figure}[tp]
  \centering
  \includegraphics[width=1.0\linewidth]{./images/T_28.png}
  \caption{ボタン要素に対するコンテキストメニューを表示した\tool の外観}
  \label{fig:T_28}
\end{figure}
ボタン要素に対するコンテキストメニューの「コピー」を選択することで対象のボタン要素をコピーする。
その後、「貼り付け」を選択するとボタン貼り付け用のボタン名入力ダイアログを出力する。ボタン貼り付け用のボタン名入力ダイアログを表示した\tool の外観を、図\ref{fig:T_29}に示す。
今回は、「確定」を入力し「OK」を選択することでCTM領域、および、\VDM 仕様にコピーした内容を追加する。貼り付けを行ったCTM領域を図\ref{fig:T_30}に、\VDM 仕様表示領域の変更箇所をオレンジ色の枠で囲ったものを図\ref{fig:T_31}に、それぞれ示す。
\begin{figure}[tp]
  \centering
  \includegraphics[width=1.0\linewidth]{./images/T_29.png}
  \caption{ボタン貼り付け用のボタン名入力ダイアログを表示した\tool の外観}
  \label{fig:T_29}
\end{figure}

\begin{figure}[tp]
  \centering
  \includegraphics[width=1.0\linewidth]{./images/T_30.png}
  \caption{貼り付けを行ったCTM領域}
  \label{fig:T_30}
\end{figure}

\begin{figure}[tp]
  \centering
  \includegraphics[width=0.7\linewidth]{./images/T_31.png}
  \caption{貼り付けを行った\VDM 仕様表示領域}
  \label{fig:T_31}
\end{figure}

以上より、「右クリック操作」が正確に動作することを確認した。

\subsubsection{左クリック操作}

左クリック操作では、「ドラッグ操作」、および、「画面管理クラスから対象画面クラスへの遷移」ができることを確認する。なお、左クリック操作では、「編集対象CTM要素選択」も機能として持っているが、操作ボタン領域のボタン操作によるCTM編集機能の「削除」機能(\ref{sec:CTM-edit}節を参照)で削除対象のCTM要素を選択する際に使用して「編集対象CTM要素選択」が正しく動作することを確認したため、ここでは確認できているものと判断する。

まず、「ドラッグ操作」を確認する。
使用するMarkdown仕様はリスト\ref{lst:T-Screen}に示した「BOX数入力画面.md」である。
図\ref{fig:T-CTM-Screen}に示した「BOX数入力画面.md」に対するCTM領域の「0押下 → 表示部に0を追加」に該当するイベント要素をドラッグ操作する。
対象のイベント要素を下に2要素分ドラッグ操作する。ドラッグ前のCTM領域とドラッグ後のCTM領域の比較を図\ref{fig:T_32}に、
ドラッグ前の\VDM 仕様表示領域とドラッグ後の\VDM 仕様表示領域の比較を図\ref{fig:T_33}に、それぞれ示す。

\begin{figure}[tp]
  \centering
  \includegraphics[width=1.0\linewidth]{./images/T_32.png}
  \caption{ドラッグ前のCTM領域とドラッグ後のCTM領域の比較}
  \label{fig:T_32}
\end{figure}
\begin{figure}[tp]
  \centering
  \includegraphics[width=1.0\linewidth]{./images/T_33.png}
  \caption{ドラッグ前の\VDM 仕様表示領域とドラッグ後の\VDM 仕様表示領域の比較}
  \label{fig:T_33}
\end{figure}

次に、「画面管理クラスから対象画面クラスへの遷移」を確認する。
使用するMarkdown仕様は、リスト\ref{lst:T-ScreenList}に示した「画面管理.md」である。図\ref{fig:T-CTM-ScreenList}に示した「画面管理.md」に対するCTM領域の「BOX数入力画面」に該当する画面要素を左ダブルクリックし、対象画面クラスである「BOX数入力画面.md」
に対応する画面仕様に遷移する。左ダブルクリック直前の\tool の外観と左ダブルクリックによる遷移後の\tool の外観比較を、図\ref{fig:T_34}に示す。

\begin{figure}[tp]
  \centering
  \includegraphics[width=1.0\linewidth]{./images/T_34.png}
  \caption{左ダブルクリック直前の\tool の外観と左ダブルクリックによる遷移後の\tool の外観比較}
  \label{fig:T_34}
\end{figure}

したがって、「左クリック操作」が正確に動作することを確認した。

また、以上より、CTM領域上でのユーザ操作によるCTM編集機能が正しく動作できることを確認した。
さらに、GUI編集による仕様編集機能が正しく動作できることを確認した。