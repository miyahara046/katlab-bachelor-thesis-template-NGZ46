\chapter{考察}\label{cha:Evaluation}
本研究では、画面遷移システムを対象とした\tool の\VDM 仕様の作成にかかる時間の削減を目的として、\tool にGUI操作による\VDM 仕様の作成を可能にする拡張を行った。
本章では、\ref{sec:Evaluation}節で評価実験を行い、拡張後の\tool の有用性を評価する。次に、\ref{sec:Comparison}節で\tool と既存ツールを比較する。最後に、\ref{sec:ToolProblem}節で\tool の問題点とその解決策について述べる。

\section{拡張後の\tool の有用性に関する評価}\label{sec:Evaluation}

拡張後の\tool の有用性の向上を評価するために、評価実験を行う。
評価実験では、画面遷移システムの\VDM 仕様の作成にかかる時間を計測することで、拡張後の\tool の有用性の向上を確認する。
具体的には、拡張後の\tool を使用した場合と既存の\tool を使用した場合で、
画面遷移システムの\VDM 仕様の一部を作成する際にかかる時間を、
それぞれ計測する。
評価実験で作成対象とする画面遷移システムの仕様は、図\ref{fig:FTSScreen}に示した仕様(以降、ケースAと呼ぶ)と、
ケースAとは異なる組込みシステムの画面の仕様(以降、ケースBと呼ぶ)である。
ケースBの外観を、図\ref{fig:FTSScreenB}に示す。
\begin{figure}[tp]
  \centering
  \includegraphics[width=1.0\linewidth]{./images/TestB.png}
  \caption{ケースBの外観}
  \label{fig:FTSScreenB}
\end{figure}

拡張後の\tool では、ケースA、Bの2つの仕様に対応するCTMの作成にかかる時間を、\VDM 仕様の作成にかかる時間として計測する。
既存の\tool では、ケースA、Bの2つの仕様に対応するMarkdown仕様の作成にかかる時間を、\VDM 仕様の作成にかかる時間として計測する。
また、今回使用するケースA、Bの2つの仕様にはその画面の画面名を表記していないため、画面名は既に記入している状態から作成終了までの時間を、\VDM 仕様の作成にかかる時間とする。
以降、被験者の説明、実験方法の説明、および、実験結果とそれに対する考察
を行う。

\subsubsection{被験者の説明}
被験者は、宮崎大学で情報工学を専攻する4人(以降、A~D と呼ぶ) の学生である。
今回の実験では、既存の\tool の使用経験がある者が参加する。そのため、知識の差や経験による実験結果への影響を抑えるために、実験
に必要な知識と操作方法は、事前に十分に説明する。また、作業の慣れによる実験結果への影響を減らすために
、被験者A、Bの2人は既存の\tool を使用した後に拡張後の\tool を使用するのに対して、被験者C、Dの2人は拡張後の\tool を使用した後に既存の\tool を使用する。

\subsubsection{実験方法の説明}

以下に、実験の手順を示す。
\begin{enumerate}
\item 実験者は、被験者に対し、仕様のサンプルと、そのサンプルの仕様に対するCTMとMarkdown仕様を提示し、実験内容を説明する。
\item 実験者は、被験者に対し、Markdown仕様の記述ルール、および、拡張後の\tool の操作方法を説明する。
\item 実験者は、被験者に対し、ケースAまたはケースBの仕様を提示し、時間の計測を始める。
\item 被験者は、CTMまたはMarkdown仕様の作成を開始し、作成完了時点で実験者へ報告を行う。
\item 実験者は時間計測を一時停止し、成果物である\VDM 仕様を検証する。成果物に誤りが無い場合は、計測を終了する。誤りがある場合は、誤りがあることを被験者に伝え、計測を再開した上で手順4.に戻る。
\end{enumerate}

\subsubsection{実験結果とそれに対する考察}
表\ref{tab:ER}に、評価実験による既存の\tool と拡張後の\tool の\VDM 仕様作成にかかる時間の比較を示す。
\begin{table}[tp]
\centering
\caption{\VDM 仕様の作成にかかる時間の比較(分:秒)}
\label{tab:ER}
\begin{tabular}{c|c|c|c|c}
 & \multicolumn{2}{c|}{\textbf{ケースA}} & \multicolumn{2}{c|}{\textbf{ケースB}} \\
\hline
\textbf{被験者}&\textbf{既存の\tool}&\textbf{拡張後の\tool}&\textbf{既存の\tool}&\textbf{拡張後の\tool}\\
\hline\hline
A&10:32&-&-&4:27\\
\hline
B&6:28&-&-&4:04\\
\hline
C&-&5:17&10:11&-\\
\hline
D&-&5:39&8:36&-\\
\hline\hline
平均&8:30&5:28&9:24&4:16\\
\hline
\end{tabular}
\end{table}
表\ref{tab:ER}に示すように、ケースAの\VDM 仕様の作成に要した時間は、拡張後の\tool を使用した場合、既存の\tool を使用した場合より、
平均で3分2秒(約35\%)短いことを確認できた。%35.68\%
また、ケースBの\VDM 仕様の作成に要した時間は、拡張後の\tool を使用した場合、既存の\tool を使用した場合より、
平均で5分8秒(約55\%)短いことを確認できた。%55.14\%
加えて、既存の\tool を使用した場合には、Markdown仕様の記述ルールに則っていない記述をしてしまうヒューマンエラーを確認した。具体的には、タイトルラベルの間違いや、
有効ボタン記述フィールドとイベント記述フィールドのボタンの名称が異なっているといったミスが起きた。さらに、ミスを指摘してもミスに気付くまでに時間を要した。
以上のことから、拡張後の\tool は、GUI操作によって\VDM 仕様を作成できるため、Markdownで記述する既存の\tool よりも、\VDM 仕様の作成にかかる時間を削減できた。
よって、本研究の拡張により、\tool の有用性が向上したといえる。

\section{既存ツールとの比較}\label{sec:Comparison}

本研究で拡張した\tool と、既存ツールとの比較を行う。VDM++仕様を作成可能なツールとして、VDMTools\cite{VDMTools}やOverture IDE\cite{Overture}がある。
これらのツールは、\VDM の記述支援機能を提供している点で\tool と共通している。
VDMTools は、統合開発環境として広く普及している。VDMToolsは、
\VDM 仕様の構文チェック、型チェック、アニメーション実行など、多機能な支援を提供する。
具体的には、\VDM 仕様からC\texttt{++}やJavaなどのコードを生成する機能があり、
形式仕様から実装コードへ変換することが可能である。 
Overture IDE は、Eclipseを基盤としたオープンソースの統合開発環境として提供しているVDMツール群であり、
VDM-SLおよび\VDM の編集、および、検証支援を行う。
具体的には、\VDM の文法チェックや、関数および操作を指定して実行することが可能である。
また、実行中に実行エラーや事前条件違反などが起きた場合、その時点での変数の値を確認したりステップ実行を試みたりすることができる。

これらの既存ツールに対し、拡張後の\tool は\VDM から実装コードを出力する機能や、\VDM 仕様に対してのデバッグ機能は持っていない。
一方、拡張後の\tool は、画面遷移システムの\VDM 仕様をGUI操作で作成できる点において有用である。
具体的には、これらのツールはどちらもテキストベースで\VDM 仕様を作成することを前提としているため、\VDM を記述するための知識の事前習得が必須である。
これに対して、拡張後の\tool は画面遷移システムにおいてGUI操作で\VDM 仕様を作成できるため、\VDM 初学者でも\VDM 仕様を作成できるという点で既存のツールより有用である。

\section{\tool の問題点とその解決策}\label{sec:ToolProblem}
現状の\tool の問題点と、その解決策について述べる。

\begin{itemize}
\item \textbf{画面遷移システムにおける適用範囲の制限}

\tool は、\ref{sec:Specrule}節に示したMarkdown仕様の記述ルールに基づいた
Markdown仕様を入力として処理することを前提としている。
そのため、すべての画面遷移システムに対して適用可能であるとは限らない。
例えば、\tool では有効ボタンに対応する操作を「押下」に限定しており、
「長押し」や「フリック」といった操作には対応していない。
これは、Markdown仕様の記述ルールが定義する操作の種類を限定していることに起因する。
今後、Markdown仕様の記述ルールの定義範囲を拡張し、
それに基づいた処理を\tool に追加することで、
より多様な画面遷移システムへ適用可能になると考える。

\item \textbf{ユーザによる操作量の多さ}

仕様を一から作成する場合、拡張した\tool では、
主に操作ボタン領域のボタン操作を通じて CTM要素を逐次追加していく必要がある。
そのため、「ボタンの追加」や、
追加したボタンに対応する「イベントの追加」を個別に行う必要があり、
ユーザの操作量が多くなるという問題がある。
これは、各ダイアログにおいて
複数の要素を同時に入力、および、設定できる機能を導入することで、
操作回数を削減することにより解決できると考える。

\item \textbf{編集操作の履歴管理を行っていない}

拡張した\tool では、CTM要素の追加や削除、編集を行うことができるが、
これらの操作に対する履歴管理機能は提供していない。
そのため、ユーザが誤操作をした場合に、
直前の状態へ容易に戻ることができないという問題がある。
これは、操作履歴を管理し、
取り消しや再実行を可能とする機能を導入することで、解決できると考える。

\item \textbf{CTMの矢印を操作できない}

拡張した\tool では、CTMの矢印はCTM領域に描画しているため要素間に固定している。
そのため、矢印の付け替えがユーザによってできないため、ユーザがボタン要素に対するイベント要素を他のボタン要素に対応するイベント要素と入れ替えたい場合は、
イベント要素自体を編集する必要があるという問題がある。
これは、矢印自体を操作可能にし、矢印の先端と後端の座標をCTM要素の外接矩形と紐づけることで解決できると考える。
\end{itemize}