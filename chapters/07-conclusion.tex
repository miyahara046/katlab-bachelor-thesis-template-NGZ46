\chapter{おわりに}\label{cha:Conclusion}
本研究では、画面遷移システムを対象とした\VDM 仕様作成支援ツール\tool(2vdm-spec-generator)の\VDM 仕様の作成時間削減を目的として、\tool の拡張を行った。具体的には、既存の\tool は、\VDM 仕様の作成に時間がかかるという課題が存在する。
そこで、\tool の\VDM 仕様の作成にかかる時間の削減を目的として、\tool にGUI操作による\VDM 仕様作成を可能にする拡張を行った。

拡張した\tool は、以下3つの機能を提供する。
\begin{itemize}
    \item ページ遷移機能
    \item 描画機能
    \item GUI操作による仕様編集機能
\end{itemize}

ページ遷移機能は、\tool の起動時に表示する「スタートページ」、「Markdown仕様記述ページ」、および、本研究の拡張で新たに追加する「GUI操作による\VDM 仕様編集ページ」の3つのページ間の遷移を可能にする機能である。

描画機能は、「GUI操作による\VDM 仕様編集ページ」において、「フォルダツリー表示領域」、「CTM領域」、「\VDM 仕様表示領域」、および、「操作ボタン領域」の4つの領域に適した描画を可能にする機能である。

GUI操作による仕様編集機能は、ユーザによるメニューバーの操作、操作ボタンの操作、CTM領域のクリックイベントに応じて、CTMの要素の追加、削除、編集を行う。さらに、CTMの要素の追加、削除、編集を行うことで、編集対象としているMarkdown仕様、および、\VDM 仕様を作成、編集する機能である。

本研究で拡張した\tool は、5章の適用例で示したように、正しく動作し、GUI操作から\VDM 仕様を生成できることを確認した。加えて、拡張後の\tool が\VDM 仕様作成時間を削減できることを確認するために、既存の\tool と拡張後の\tool を用いて、評価実験を行った。
評価実験では、株式会社フルタイムシステム\cite{fts}が開発し、実際に運用している画面遷移システムの画面の仕様の一部を参考に作成した2つの画面の仕様を対象とした\VDM 仕様を作成し、その作成時間を比較する実験を行った。
その結果、拡張後の\tool は、既存の\tool に比べて、\VDM 仕様の作成時間を削減できることを確認した。

以上より、本研究の拡張によって、\VDM 仕様作成支援ツール\tool の\VDM 仕様の作成時間が削減できたと考える。

以下に、今後の課題を示す。

\begin{itemize}
\item \textbf{画面遷移システムにおける適用範囲の拡張}

\tool は、\ref{sec:Specrule}節に示したMarkdown仕様の記述ルールに基づいた
Markdown仕様を入力として処理することを前提としている。
そのため、画面遷移システムにおいても適応範囲が限定的である。
例えば、\tool では有効ボタンに対応する操作を「押下」に限定しており、
「長押し」や「フリック」といった操作には対応していない。
これは、Markdown仕様の記述ルールが定義する操作の種類を限定していることに起因する。
今後、Markdown仕様の記述ルールの適用範囲を拡張し、
それに基づいた処理を\tool に追加することで、
より多様な画面遷移システムへ適用可能にする必要がある。

\item \textbf{ユーザによる操作量の削減}

仕様を一から作成する場合、拡張した\tool では、
主に操作ボタン領域のボタン操作を通じて ユーザがCTM要素を逐次追加していく必要がある。
その追加のたびにダイアログを表示し、ユーザは入力または選択を行う必要があり、
ユーザの操作量が多くなるという問題がある。
これは、各ダイアログにおいて
複数の要素を同時に入力、および、設定できる機能を導入することで、
操作回数を削減する必要がある。

\item \textbf{編集操作の履歴管理機能の追加}

拡張した\tool では、CTM要素の追加や削除、編集を行うことができるが、
これらの操作に対する履歴管理機能は提供していない。
そのため、ユーザが誤操作した場合に、
直前の状態へ容易に戻すことができないという問題がある。
これは、操作履歴を管理し、
取り消しや再実行を可能とする機能を導入する必要がある。

\item \textbf{CTMの矢印を操作可能にする}

拡張した\tool ではCTMの矢印はCTM領域に描画しているため要素間に固定している。
よって矢印の付け替えがユーザによってできないため、ユーザがボタン要素に対するイベント要素を他のボタン要素に対応するイベント要素と入れ替えたい場合、
イベント要素自体を編集する必要が出てくるという問題がある。
これは、矢印自体を操作可能にし、矢印の先端と後端の座標をCTM要素の外接矩形と紐づける必要がある。
\end{itemize}

