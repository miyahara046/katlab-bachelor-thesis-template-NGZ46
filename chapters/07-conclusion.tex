\chapter{おわりに}\label{cha:Conclusion}
本研究では、画面遷移システムを対象とした\VDM 仕様作成支援ツール\tool(2vdm-spec-generator)の\VDM 仕様作成時間削減を目的として、\tool に拡張を行った。既存の\tool には、Markdown仕様を作成する際の支援機能が不十分であり、\VDM 仕様作成に時間がかかるという課題点が存在する。

本研究では、\tool の\VDM 仕様作成時の支援機能を強化し、\VDM 仕様作成時間の削減を目的として、\tool にGUI操作による\VDM 仕様作成を可能にする拡張を行った。拡張後の\tool は、以下3つの機能を提供する。
\begin{itemize}
    \item ページ遷移機能
    \item 描画機能
    \item GUI操作による仕様編集機能
\end{itemize}

ページ遷移機能は、\tool の起動時に表示する「スタートページ」、「Markdown仕様記述ページ」、および、本研究の拡張で新たに追加する「GUI操作による\VDM 仕様編集ページ」の3つのページ間の遷移を可能にする機能である。

描画機能は、「GUI操作による\VDM 仕様編集ページ」において、「フォルダツリー表示領域」、「CTM領域」、「\VDM 仕様表示領域」、および、「操作ボタン領域」の4つの領域に適した描画を可能にする機能である。

GUI操作による仕様編集機能は、ユーザによるメニューバーの操作、操作ボタンの操作、CTM領域のクリックイベントに応じて、CTMの要素の追加、削除、編集、を行う。さらに、CTMの要素の追加、削除、編集、を行うことで、編集対象としているMarkdown仕様、および、\VDM 仕様を作成、編集する機能である。

本研究で拡張した\tool は、5章の適用例で示したように、正しく動作し、CTMから\VDM 仕様を生成することが可能であることを確認した。加えて、拡張後の\tool が\VDM 仕様作成時間を削減できることを確認するために、既存の\tool と拡張後の\tool を用いて、株式会社フルタイムシステム\cite{fts}が開発し、実際に運用している組込みシステムの画面の仕様の一部を参考に作成した同じ画面遷移システムを対象とした\VDM 仕様を作成し、その作成時間を比較する実験を行った。
その結果、拡張後の\tool は、既存の\tool に比べて、\VDM 仕様作成時間を削減できることを確認した。

以上のことより、本研究の拡張によって、\VDM 仕様作成支援ツール\tool の\VDM 仕様作成時の支援機能を強化し、\VDM 仕様作成時間の削減ができたと考える。

以下に今後の課題を示す。

\begin{itemize}
\item 画面遷移システムにおける適用範囲の拡張
\tool は、\ref{sec:Specrule}節に示した自然言語仕様記述ルールに対応した
Markdown仕様を入力として処理することを前提としている。
そのため、すべての画面遷移システムに対して適用可能であるとは限らない。
例えば、\tool では有効ボタンに対応する操作を「押下」に限定しており、
「長押し」や「フリック」といった操作には対応していない。
これは、自然言語仕様記述ルールが定義する操作の種類を限定していることに起因する。
今後、自然言語仕様記述ルールの定義範囲を拡張し、
それに基づいた処理を\tool に追加することで、
より多様な画面遷移システムへ適用可能になると考える。

\item ユーザによる操作量の削減
仕様を一から作成する場合、拡張した\tool では、
主に操作ボタン領域のボタン操作を通じて GUI 要素を逐次追加していく必要がある。
そのため、「ボタンの追加」や、
追加したボタンに対応する「イベントの追加」を個別に行う必要があり、
ユーザの操作量が多くなるという問題がある。
これは、各ダイアグラムにおいて
複数の要素を同時に入力、および、設定できる機能を導入することで、
操作回数を削減することにより解決できると考える。

\item 編集操作の履歴管理機能の追加
拡張した\tool では、要素の追加や削除、編集を行うことができるが、
これらの操作に対する履歴管理機能は提供していない。
そのため、誤操作が発生した場合に、
直前の状態へ容易に戻ることができないという問題がある。
このれは、操作履歴を管理し、
取り消しや再実行を可能とする機能を導入することで、解決できると考える

\item CTMの矢印を操作可能にする
拡張した\tool ではCTMの矢印はCTM領域に描画しているため要素間に固定している。
よって矢印の付け替えがユーザによってできないため、ユーザがボタン要素に対するイベント要素を逆にしてしまった場合、
イベント要素自体を編集する必要が出てくるという問題がある。
これは、矢印自体を操作可能にし、矢印の先端と後端の座標をCTM要素の概説矩形と紐づけることで解決できると考える。
\end{itemize}

